% Mechanics and computational power
The mechanics followed a deterministic approach for many years. These deterministic approaches helped humanity in various engineering and design problems. Over the last 5 decades, engineering problems got increasingly complex as the needs of humanity changed similarly. These complex problems are tackled mostly with the combination of experimental and computer-aided simulations until recently. In this combined endeavor, where experiments are hindered, computer-aided simulations helped significantly eliminate the money and time limitations of the experimental approach. Especially due to the increasing availability and accessibility of computational power. Although their wide spread uses the deterministic high-fidelity numerical solutions, also known as computer-aided simulations, the requirement of time and money is also increasing as the complexity of the problems increases further. It is getting clear that Moore's Law is losing its validity due to physical limitations of the silicone technology \cite{arenas2021}. Thus, if we want to continue solving complex problems in the coming future, the need for another approach that can further reduce the time and money requirements of the current experimental and computer-aided numerical solutions.

% ML in mechanics
Machine learning (ML) applications from computer-vision, to speech recognition, are getting a more and more essential part of our daily life. To this end, it is not a surprise that ML found use cases in mechanics as well. To the authors' knowledge the first mechanics applications of the machine learning paradigm date back to the second-boom (1990's till the 2000s) of ML in the fields of civil and mechanical engineering (\eg \cite{reich1997,reich1995,bishop1993,adeli}). Similar to numerical method applications used in the field of mechanics, machine learning also benefited from the increasing computational power. However, reaching the popularity and widespread application areas can be attributed to the more open and developing aimed attitude of the field. With this increasing open access to ML tools, the applied use cases of ML in the mechanics community started to flourish.

% Recent landscape
The effort put into decreasing the time complexity of the direct numerical solutions, is sprouting. On one hand, a general way to tackle the differential equations resulting in a mechanics problem is being tackled. For instance,  Physics-Informed Neural Networks (PINNs) \cite{raissi2019b} and its variants, where the fields of interest are represented as neural networks and the residual enforced by the differential equations, and initial and boundary conditions are minimized.  Moreover, the operators that are the key parts of differential equations are tried to be learned from the data in \cite{lu2021a}. On the other hand, more specific applications try to tackle more specific sub-problems with the help of various ML techniques. These examples mostly try to accelerate a certain part of a certain problem via data-driven approaches. Due to its nested structure, the time complexity of the multi-scale direct numerical solutions is one of the most laborious endeavors in solid mechanics especially when coupled with other physical-phenomenon, especially for multi-scale problems (\eg \cite{Rocha2021,Rocha2020}. The time complexity of the multi-scale problem is tried to be reduced by replacing the micro-scale model with a machine learning model (\eg Gaussian Process Regression (GPR), Artificial Neural Network (ANN), etc.). The starting point for this multi-scale application was introduced in \cite{bessa2017} where the design space is mapped to the response for a heterogeneous material and is modeled with both GPR and ANNs. In the same vein path and rate-dependent problems for an RVE are tackled with sequence learning strategies (\eg Gated Recurrent Unit (GRU) and Long-Short Term Memory (LSTM) models) in \cite{mozaffar2019,chen2021b}. Other than numerical multi-scale method speed ups ML based methods are used to find the buckling response of plastic tubular structures in \cite{sakaridis2022} from numerical simulations. Although it is quite common to obtain high-fidelity data from numerical simulations experimental data can also be utilized as shown in \cite{li2019} for a temperature and rate-dependent plastic material.

It is observable from most of the literature that ML methods need an abundance of data. For the problems that are hard to get obtain data, another type of ML application in solid mechanics is utilized under the Reduced Order Models (ROMs), In \cite{liu2016,ferreira2021,liu2019b} different ways of modeling the nonlinear behavior of heterogeneous materials can be found with the help of various machine learning strategies in combination with a small number of numerical simulations. 

% Problems with the landscape
Although the acceptance and the usage of ML in mechanics are widening as can be seen from the literature, three possible problems with the current approaches can be identified.
\begin{itemize}
  \item \textit{Risky Extrapolation:} Almost all the applications in the design space are sampled extensively to always keep the generated model in the interpolation regime. This is done due to the inability of conventional ML models to have bad extrapolation capabilities. To keep the ML model in the interpolation regime one can utilize the above-mentioned ROMs can be utilized, when these efficient methods are not available the applicability of the ML methods to problems is hindered.
  \item \textit{Data Scarcity:} Most problems that are tackled recently rely on abundant data. However, for the problems that are harder to obtain numerical or experimental data, again ML applications lose their validity.
  \item \textit{Single Parameter Configuration:} Most of the tackled problems rely on fixing certain parameters of the application. For instance, if the interest is in applying ML on a material that is only of prime interest. This type of application shut out the possibility of its utilization by other groups across the world. 
\end{itemize}

% How can we solve these problems?
  These three problems are an extensive research area in the field of ML and their applications under the transfer learning paradigm. However, since the solid mechanics field is a late adapter of ML, the growing pains of the current field are not allowing a more in-depth dive into the field of ML to tackle bigger problems than seen currently. Thus, there is a lack of investigation or utilization of ML paradigms that are more complex.

% I think we can try to connect transfer learning to continual learning here!
Transfer learning is one of those paradigms. It allows the utilization of the distilled information from the source tasks to a new and unique target task \cite{pan2010a}. One specific problem setting that is relevant in the transfer learning paradigm is the continual learning setting where target tasks are learned in sequence one after the other with no ultimate final task \cite{thrun1998}. 

% What is the current state of the landscape regarding the solution to these problems?
Although limited, the current applications of transfer learning in the context of mechanics show promising directions. Hints transfer learning utilization without mentioning the term "transfer" is present in multi-fidelity works (\eg \cite{perdikaris2015,perdikaris2017a}), where the data from the low-fidelity simulations are used to increase the predictive performance of surrogate models. The more prominent usage of transfer learning is becoming more prevalent due to even simpler transfer learning strategies (\eg using the weights found for one task as the starting point of another task). For instance, in Deep Material Network (DMNs) while creating a database of responses for a heterogeneous material smaller volume fractions are used as a starting point for the higher volume fractions in \cite{liu2019c} to improve the convergence. A similar transfer learning approach (using the weights of the previous task is used for the target task) is presented in \cite{lejeune2020a}. 

% Open-source and transfer learning 
The computer science community flourished with data and software sharing in recent years, unlike mechanics community. Although there are also negative sides to sharing data and software the power of sharing is undeniable as can be seen from the fast-paced improvements in the field. The same trend is taking over the other fields too. The mechanics field is also changing as can be seen by the flourishing open-source software that is available for mechanics (\eg FEniCS \cite{logg2012a}, MoFEM \cite{kaczmarczyk}). These open-source software pave the way for data creation and sharing culture as can be seen from \cite{lejeune2020b}, where an open-source dataset is created from completely open-source software. Predicting this type of information sharing will become more prevalent it is evident that the applications of transfer learning strategies will benefit the community of mechanics for problems of similar nature. 

% Aim of the study!
In this initial study, we aim to introduce continual learning to the field of mechanics. We show that with the help of continual learning we can not only learn the target task with less data but also retain the performance on the previous tasks. This performance preservation sets this paper apart from the other works present in the field of mechanics. In the context of applied ML in mechanics, we aim to show that by utilizing continual learning we can solve similar problems with less effort and still perform well on the previous tasks. This property of the proposed application opens the door for the overall benefit of the field in a continual manner without losing the information gained until that point. We believe that widespread acceptance of continual learning in the mechanics field can benefit the field immensely going forward.

