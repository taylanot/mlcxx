% Mechanics and computational power
The mechanics followed a deterministic approach for many years. This deterministic approaches helped humanity in various engineering and design problems. Over the last 5 decades engineering problems got increasingly complexer as the needs of the humanity changed in a similar fashion. These complex problems are tackled mostly with the combination of experimental and computer aided simulations until recently. In this combined endeavour, where experiments are hindered, computer aided simulations helped significantly eliminate the money and time limitations of the experimental approach. Especially due to increasing availability and accessibility of the computational power. Although, their wide spread use the deterministic high-fidelity numerical solutions, also known as computer aided simulations, requirement of time and money is also increasing as the complexity of the problems increase further. It is getting evidently clear that the Moore's Law is loosing its validity due to physical limitations of the silicone technology \cite{arenas2021}. Thus, if we want to continue solving complex problems in the coming future, the need for another approach that can further reduce the time and money requirements of the current experimental and computer aided numerical solutions.

% ML in mechanics
Machine learning (ML) applications from computer-vision, to speech-recognition are getting more and more essential part of our daily life. To this end, it is not a surprise that ML found use cases in mechanics as well. To the authors knowledge the first mechanics applications of the machine learning paradigm dates back to the second-boom (1990's till 2000's) of ML in the fields of civil and mechanical engineering (\eg \cite{reich1997,reich1995,bishop1993,adeli}). Similar to numerical method applications used in the field of mechanics, machine learning also benefited from the increasing computational power. However, to reach the popularity and wide-spread application areas can be attributed to the more open and development aimed attitude of the field. This increasing open-access to ML tools the applied use cases of the ML in the mechanics community started to flourish.

% Recent landscape
The effort put into decrease the time complexity of the direct numerical solutions, is sprouting. On one hand a general way to tackle the differential equations resulting in a mechanics problem is being tackled. For instance,  Physics-Informed Neural Networks (PINNs) \cite{raissi2019b} and it variants, where the fields of interests are represented as neural networks and the residual enforced by the differential equations, and initial and boundary conditions are minimized.  Moreover, the operators that are the key parts of differential equations are tried to be learn from the data in \cite{lu2021a}. On the other hand, more specific applications try to tackle more specific sub-problems with the help of various ML techniques. These examples mostly try to accelerate a certain part of a certain problem via data-driven approaches. Due to its nested structure the time complexity of the multi-scale direct numerical solutions are one of the most laborious endeavours in solid mechanics espcially when coupled with other physical-phenomenon. In \cite{bessa2017} one such application is shown where a multi-scale problem is tried to be tackled in micro-scale. Due to data hungry ML applications, another type of ML application, which is ML based reduced order models (\qg \cite{liu2016,ferreira2021}), are utilized. The utilization of these data hungry applications 

One of the major application areas involve the multi-scale 


% Problems with the landscape


% How can we solve these problems?


% The aim of the paper?


% Landscape of the current literature wrt our aim! 





