
Throughout this work uppercase bold letters (\eg $\mathbf{X}$), lowercase bold letters (\eg $\mathbf{x}$), and lowercase letters (\eg ${x}$) are used for matrices, vectors, and scalars respectively. Moreover, the vectors are assumed to be stored in columns. Finally, the $\I_{D}$ represents a $D\times D$ identity matrix, the $\ones_{D}$ and $\zeros_{D}$ represents $D\times 1$ vector of ones and zeros respectively.

\subsection{Learning Problems}

In this work, one linear and one nonlinear problem constitute the family of tasks. 

In the linear case, a realization of the slope of a linear function will be used for the definition of a single task. Consider the conventional linear regression problem in $\R^D$ is given by

\begin{equation}\label{eq:linearreg}
  \lab = \trans{\inp}\scaletask+\noise, 
\end{equation}
where $\lab\in\R$, $\inp\in\R^D$, $\scaletask\in\R^D$ and $\noise\sim\normal{0}{\var}$. Each realization of $\scaletask$ corresponds to a task $\task$ and collection of $N$ observations is represented by $\dataset:=(\inp, \lab)_{i=1}^{N}$. 

For the nonlinear problem, a single task is defined by a certain realization of amplitude and phase. Consider a nonlinear problem in $\R^D$ is given by
\begin{equation}\label{eq:nonlinearreg}
  \lab = \trans{\sine(\inp+\phasetask)}\scaletask+\noise, 
\end{equation}
where $\lab\in\R$, $\inp\in\R^D$, $\scaletask\in\R^D$ and $\noise\sim\normal{0}{\var}$. Assuming that the each realization of scale term $\scaletask$ and $\phasetask$ corresponds to a task observed in the environment $\task$ and each set of observed $N$ input ($\inp$) and its corresponding label ($\lab$) is represented by a dataset $\dataset:=(\inp_i, \lab_i)_{i=1}^{N}$.

For both linear and nonlinear problems presented sample distribution is given by $\prob_\dataset$ for a given $\task$ and the task, distribution is represented by $\prob_\task$. A model parameterized by $\param$ is represented by $\model(\inp, \param):\inp\to\lab$. An estimator that is trained with $\dataset$ that is obtained from the $\task$ is represented by $\estim(\inp)$. Noting that $\estim(\inp)$ for a base learner is only exposed to a single task $\task$ and a single dataset $\dataset$, whereas a meta learner, in this case, MAML, is exposed to multiple tasks $T$'s and multiple datasets $\dataset$ in the meta-learning stage and then the adaptation is done as in the case of a base learner with just a single task $\task$ and a single dataset $\dataset$.  The discrepancy between the prediction of the estimator $\estim$ and $\lab$ is measured in terms of squared loss $\loss:=(\estim(x)-\lab)^2$. The main loss that this paper tries to investigate is the \textit{Expected Squared Loss} of an estimator $\estim$ over the $\prob_{\task}$. Then the expected squared loss can be represented as

\begin{equation}\label{eq:ee}
  \EE:= \iiint(\estim(x) - y)^2\prob(\inp, y)\prob_{\dataset}\prob_{\task} d\inp d\lab d\dataset d\task.
\end{equation}

This defined performance measure gives rise to \textit{Bayes Error} to be given by $\sigma^2$ that is coming from the noise term, which represents a model that is the perfect estimator, which is referred to as oracle in some of the meta-learning literature.

For all the problems the input distribution is given by $\prob_\inp\sim\normal{0,k\I}$ where $k$ is a parameter for the variance of the inputs. For the linear problem the $\prob_\task:=\prob(\scaletask)\sim\normal{{m\ones_{D}},{c\I_{D}}}$ and for nonlinear problem the task distribution takes the form of a joint distribution $\prob_\task:=\prob(\scaletask, \phasetask)$ where $\prob_{\scaletask}\sim\normal{\ones_{D},c_1\I_{D}}$ and $\prob_{\phasetask}\sim\normal{\zeros_{D},c_2\I_{D}}$

\begin{figure}[ht!]
  \centering
  \begin{subfigure}[b]{0.49\textwidth}
    \centering
    \includetikz{\textwidth}{Figures_v1/methods/lin_eg.tikz}
    \caption{$\lab = \trans{\inp}\scaletask$}
    \label{fig:lintasks}
  \end{subfigure}
  \begin{subfigure}[b]{0.49\textwidth}
    \centering
    \includetikz{\textwidth}{Figures_v1/methods/nonlin_eg.tikz}
    \caption{$\lab = \trans{\sine(\inp+\phasetask)}\scaletask$}
    \label{fig:nonlintasks}
  \end{subfigure}
  \caption{100 sample tasks drawn from $\prob_\task$ for both linear ($m=0$ and $c=1$) and nonlinear ($c_1=1$ and $c_2=1$) problems.}
\end{figure}

\subsection{Models} 

For the linear problems linear model in the form of $\model(\inp,\slope,\bias):=\inp\slope+\bias$ or $\model(\inp, \param):=\inpbias\param$ with $\inp\in\R^{1\times D}$, $\slope\in\R^{D\times 1}$, $\inpbias\in\R^{1\times D+1}$ and $\param\in\R^{D+1\times}$ where $\param:=\trans{[\slope, \bias]}$ and $\inpbias:=[\inp, 1]$. The optimum parameters ($\opt$) for different linear models are obtained as follows:

\paragraph{Linear Estimator} is given by the least-squares solution, $\opt:=(\inv{\trans{\design}\design)}\trans{\design}\labs$, where $\design\in\R^{N\times D}$ is the design matrix where the observed input data is stored in rows.

\paragraph{Ridge Estimator} is given by $\opt:=(\inv{\trans{\design}\design+\ridge \I_{D})}\trans{\design}\labs$ which is obtained by minimizing the squared loss with the additional term of $\ridge\norm{\param}{2}^2$. Thus, overall loss takes the form $\loss+\ridge\norm{\param}{2}^2$.

In a similar fashion linear models can be used in nonlinear regression problems with the help kernel trick. Then,

\paragraph{Kernel Ridge Estimator} is given by $\opt= \trans{\design}\weight$ where $\weight:=\inv{(\gram+\ridge\I_{N})}\lab$ where $\gram\in\R^{N\times N}$ is the  \textit{Gram Matrix} obtained by replacing $\trans{\design}\design$ inner product by a kernel $\kernel(\design, \design)$. Then, the prediction of the estimator takes the form $\estim(\pred,\param)=\trans{\weight}\kernel(\pred,\design)$ where $\pred\in\R^{D\times 1}$.

For both linear and nonlinear models, gradient descent can be utilized to update the parameters of a model $\model$. Then,

\paragraph{Gradient Descent Estimator} for any given model $\model(\inp, \param)$ and a given number of iterations $\iter$ the gradient descent estimator is given by $\{\param_{j+1}=\param_{j} - \lr\sum_i^{N}\inp_i(\model(\inp,\param_j)-\lab_i)\}_{j=0}^{\iter}$. In other words for any given value of $\param$ one gradient update is given by the gradient with respect to $\param$ with a scaling parameter $\nu$. 

All the models investigated can be divided into two sub-categories the models that have information regarding the task space. The labels 
of the models that have information regarding the task space are as follows:
\begin{itemize}
  \item \textbf{GD}: corresponds to gradient descent with $\iter$ with the adjustable parameters obtained from the mean of the tasks $\mathbb{E}[\prob_{\task}]$. (\eg considering the linear problem the with $\prob_{\scaletask}\sim\normal{\zeros}{\I}$, the initial starting point of the linear model is $\param=\zeros$.)
  \item \textbf{MAML} (for linear problem): corresponds to gradient descent with $\iter$ with the adjustable parameters obtained from the mean of the tasks $\mathbb{E}[\prob_{\task}]$ with small perturbation $\delta\sim\normal{\zeros}{0.1\I}$. This is done to simulate the effect of various learning rates and stopping points during the meta-learning stage.
  \item \textbf{MAML} (for nonlinear problem): MAML algorithm trained with the information given in \cite{finn2017} for the sinusoidal regression problem. It should be noted that network architecture and all the other hyper-parameters are taken from the paper exactly.
  \item \textbf{random GD}: corresponds to a gradient descent of a randomly initialized model.
\end{itemize} 

Finally, the following model labels have information from a single task:
\begin{itemize}
  \item \textbf{Linear}: standard least squares solution.
  \item \textbf{Ridge}: standard least squares solution with $L_2-regularization$.
  \item \textbf{random GD}: gradient descent with $\iter$ with the adjustable parameters starting from random initialization.
  \item \textbf{Kernel Ridge}: kernelized (with Radial Basis Function Kernel) 
\end{itemize}

For the linear problem setting, it should be noted that the Linear problem gives the optimum which means that when the gradient descent with infinitely small learning rate with the infinite number of gradient steps. Thus, allowing us to investigate the difference between taking limited steps or allowing the model to go towards the optimum directly.

It should be noted that the hyper-parameters of the utilized models, if there are any, are not tuned, properly as it would increase the computational burden of the problem to another level. However, a simple grid search is employed with 20 different values and only the one with the lowest mean expected performance over the parameter under investigation is presented for the results. 



