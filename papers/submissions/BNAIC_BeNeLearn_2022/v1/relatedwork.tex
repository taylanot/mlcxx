% MAML and its Extensions (deficiencies and improvements)
After the introduction of the MAML, multiple papers followed its backbone idea, which is finding a warm initialization for all the tasks coming from a certain task distribution. In the follow-up works, the deficiencies of MAML are tried to be tackled. \cite{flennerhag2019,collins2020} try to tackle the task similarity needed for MAML, in other words, make the meta-learner more task family robust. The sensitivity of MAML to architectural details is tried to be circumvented in \cite{antoniou2019}. And, the need for the second order term needed by MAML is questioned and shown that first order approximations can give as good results as MAML in \cite{nichol2018}. Deeper changes to the MAML method are presented in \cite{grant2018} Moreover, an extension for the continual learning where tasks are introduced sequentially is proposed in \cite{finn2019,rajasegaran2020}. 

% Emphasis on Gradient steps 
Besides most of the above-mentioned architectural and problem tweaks, some of the attention went to improving the quick adaptation stage from the learned initialization. In \cite{li2017} not just the initialization but the update direction and the learning rate are tried to be found. Moreover, in \cite{behl2019} the learning rate for the adaptation is tried to be learned with hypergradient descent. Both of the methods aim to limit the adaptation steps needed. 

% Few-shot Supervised Learning Benchmarks
It can be seen from the derivative work of MAML that the extra aim introduced with MAML to meta-learning, which is a quick adaptation, is the common denominator in most of the work that can be found. There seems to be a case for quick adaptation in dynamic problems, in other problems, this seems not to be the case. All of the above-mentioned articles use the Omniglot dataset \cite{lake2019} as a supervised classification benchmark with the MAML as the baseline. Moreover, some of the work use the sinusoidal regression task as shown in the original MAML paper \cite{finn2017}. Both of these settings where MAML and its variants are tested do not have the time or the memory restrictions as a few-shot learning problem. However, quick adaptation methods are still being benchmarked with these methods. Although there is quite a lot of effort into improving MAML it is still not clear the effect of adaptation step limitation in a problem that does not necessarily suffer from the quick adaptation constraint.


