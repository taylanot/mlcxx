Model-Agnostic Meta-Learning is a meta-learning method that achieved state-of-the-art performance on few-shot image classification benchmarks at the time of its introduction. MAML's strength is its ability to quickly adapt to a new task in time-critical settings, while still being general enough for any gradient-based model. Although there is no need for quick adaptation in most of the few-shot learning benchmarks, MAML is still being utilized as a benchmark in this context. We investigate the benefit of limiting the adaptation steps of MAML in settings where quick adaptation is not required by the problem. In this initial study, the expected performance of MAML is compared to some conventional base learners for synthetic linear and nonlinear regression problems. Our experimental results show that considering few gradient steps in the presence of small task variance the limitation of the gradient steps improves generalization performance when tuned properly.
