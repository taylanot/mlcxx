Model-Agnostic Meta-Learning is a meta-learning method that achieved state-of-the-art performance few-shot image classification benchmarks at the time of its introdction. MAML's primary novelties lies with it being applicable to any gradient based model and allowing quick adaptation to a new task in time-critical settings. Although, there is no need for a quick adaptation in most of the few-shot learning benchmarks, MAML is still being utilized as a benchmark for derivative/follow-up work. This raises the question what does MAML add in those settings where the quick adaptation is not bestowed by the problem. In this intial study we aim to investigate the expected performance of the MAML compared to some conventional base learners for synthetic linear and nonlinear regression problems. 


