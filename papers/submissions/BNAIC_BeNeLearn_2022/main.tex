% This is samplepaper.tex, a sample chapter demonstrating the
% LLNCS macro package for Springer Computer Science proceedings;
% Version 2.21 of 2022/01/12
%
\documentclass[runningheads]{llncs}
\usepackage[utf8]{inputenc} % allow utf-8 input
\usepackage[T1]{fontenc}    % use 8-bit T1 fonts
\usepackage{hyperref}       % hyperlinks
\usepackage{url}            % simple URL typesetting
\usepackage{booktabs}       % professional-quality tables
\usepackage{amsfonts}       % blackboard math symbols
\usepackage{amsmath}
\usepackage{nicefrac}       % compact symbols for 1/2, etc.
\usepackage{microtype}      % microtypography
\usepackage{lipsum}
\usepackage{graphicx}
\usepackage{tikz}
\usepackage{caption}
\usepackage{float}
\usepackage{subcaption}
\usepackage{algorithm2e}
\usepackage{algpseudocode}
\usepackage{amsmath}        % blackboard math symbols
\graphicspath{{Figures_v1/}}     % organize your images and other figures 
\usetikzlibrary{external}
\tikzexternalize[prefix=Figures_v1/tikz]
\newcommand{\includetikz}[2]{ \resizebox{#1}{!}{\input{#2}}}

\usepackage{pgfplots}
\usepackage{bm}             % blackboard math symbols
\usepackage[cal=cm]{mathalfa}

%%%% COMMANDS %%%%

% problem
\newcommand{\slope}{\mathbf{w}}
\newcommand{\bias}{b}
\newcommand{\inp}{\mathbf{x}}
\newcommand{\pred}{\mathbf{x}^*}
\newcommand{\task}{\mathcal{T}}
\newcommand{\dataset}{\mathcal{Z}}
\newcommand{\scaletask}{\mathbf{a}}
\newcommand{\inpbias}{\bar{\mathbf{x}}}
\newcommand{\phasetask}{\boldsymbol{\phi}}
\newcommand{\lab}{y}
\newcommand{\labs}{\mathbf{y}}
\newcommand{\EE}{\mathcal{E}}
\newcommand{\prob}{p}
\newcommand{\trainset}{Z}
\newcommand{\model}{\mathcal{M}}
\newcommand{\estim}{\mathcal{\hat{M}}}
\newcommand{\noise}{\varepsilon}
\newcommand{\var}{\sigma^2}
\newcommand{\mean}{\mu}
\newcommand{\ridge}{\lambda}
\newcommand{\kernel}{\kappa}
\newcommand{\weight}{\boldsymbol{\alpha}}
\newcommand{\genridge}{\mathbf{h}}
\newcommand{\normal}[2]{\mathcal{N}(#1, #2)}
\newcommand{\param}{\bar{\mathbf{w}}}
\newcommand{\opt}{\mathbf{\hat{\bar{w}}}}
% GD
\newcommand{\iter}{n_{{iter}}}
\newcommand{\lr}{\eta}
\newcommand{\grad}[1]{\nabla_{#1}}
\newcommand{\loss}{\mathcal{L}}
% math
\newcommand{\trans}[1]{#1^\text{T}}
\newcommand{\inv}[1]{#1^{-1}}
\newcommand{\sine}{\text{sin}}
\newcommand{\norm}[2]{||#1||_#2}
\newcommand{\R}{\mathbb{R}}
% matrix
\newcommand{\design}{{\mathbf{X}}}
\newcommand{\gram}{{\mathbf{K}}}
\newcommand{\ones}{\mathbf{1}}
\newcommand{\zeros}{\mathbf{0}}
\newcommand{\I}{\mathbf{I}}

% Text Related Commands
\newcommand{\eg}{\textit{e.g.}}

\newcommand{\version}{v1/}

%\maketitle


\begin{document}
%
\title{Contribution Title\thanks{Supported by organization x.}}
%
%\titlerunning{Abbreviated paper title}
% If the paper title is too long for the running head, you can set
% an abbreviated paper title here
%
\author{First Author\inst{1}\orcidID{0000-1111-2222-3333} \and
Second Author\inst{2,3}\orcidID{1111-2222-3333-4444} \and
Third Author\inst{3}\orcidID{2222--3333-4444-5555}}
%
\authorrunning{F. Author et al.}
% First names are abbreviated in the running head.
% If there are more than two authors, 'et al.' is used.
%
\institute{Princeton University, Princeton NJ 08544, USA \and
Springer Heidelberg, Tiergartenstr. 17, 69121 Heidelberg, Germany
\email{lncs@springer.com}\\
\url{http://www.springer.com/gp/computer-science/lncs} \and
ABC Institute, Rupert-Karls-University Heidelberg, Heidelberg, Germany\\
\email{\{abc,lncs\}@uni-heidelberg.de}}
%
\maketitle              % typeset the header of the contribution


\begin{abstract}
  asldkfjasdklfj

\end{abstract}


% keywords can be removed
\keywords{meta-learning \and expected performance \and model-agnostic meta-learning}

\section{Introduction}\label{sec:intro}
  
% Meta Learning Definitions
%Learning to learn, also referred to as meta-learning, treats the training of a machine learning model as a learning problem in itself. In the context of this work if a machine learning model's performance on a task is improving with training experiences it is said to be learning. In the light of this definition a machine learning model is said to be learning to learn if the performance on each task improves with training experience obtained from each task and with the number of tasks \cite{thrun1998}. Meta-learning recently is being used to tackle few-shot learning problems, where there is little data available from the learning task that is of prime interest, whereas there is an abundance of data from other similar tasks. 

Learning to learn, also referred to as meta-learning, treats the training of a machine learning model as a learning problem in itself. In this setting, there exist multiple learning problems and the learning problems are treated altogether. Then, a machine learning model is said to be learning to learn if the performance on each task improves with training experience obtained from each task and with the number of tasks \cite{thrun1998}. Meta-learning recently is being used to tackle few-shot learning problems, where there is little data available from the learning task that is of prime interest, whereas there is an abundance of data from other similar tasks. 

% What Makes MAML Different
%Early works of the learning-to-learn paradigm relied upon the one supervisory and one sub-ordinate model that interacts with each other for meta-learning. On one hand, sub-ordinate models try to improve the performance with training examples and on the other hand, the supervisory model tries to increase the performance over the family of tasks. MAML (Model-Agnostic Meta-Learning) \cite{finn2017} is a model that circumvents the need for supervisory and subordinate models. This method tries to tackle meta-learning by training any (As the name suggests MAML applies to all learners that improve performance by SGD (Stochastic Gradient Descent).) machine learning models parameters in a way to maximize the performance on a new learning task with few experiences through one or more gradient steps. 

Early works of the learning-to-learn paradigm relied upon two consecutive models working simultaneously, where one model tries to improve performance on the specific task and the other tries to improve performance over the observed tasks together. Considering this nested structure  MAML (Model-Agnostic Meta-Learning) \cite{finn2017} provides an algorithm that circumvents the need for multiple models. This method tries to tackle meta-learning by training any (all learners that improve performance by SGD (Stochastic Gradient Descent)) machine learning models parameters in a way to maximize the performance on a new learning task with few experiences through one or more gradient steps. 

% Where to use MAML?
%MAML is used in few-shot learning problems in the supervised and reinforcement learning problems, where the losses differ from each other. Due to being model and problem independent MAML finds a wide application area in the context of few-shot meta-learning. Moreover, MAML also aims to improve a specific task performance quickly (with a few gradient steps). This is an additional aspect to our definition of meta-learning. 

Due to being model and problem independent MAML finds a wide application area in the context of few-shot meta-learning. It is used under few-shot learning problems for supervised and reinforcement learning problems, where the losses differ from each other.  Moreover, MAML also aims to improve a specific task performance quickly (with a few gradient steps). This is an additional aspect to the definition of meta-learning. 

% What is the problem?
%As mentioned above, MAML aims to improve the generalization of a model for a certain learning task from a given family of tasks, with little data and minimal training. Minimal training indicates quick adaptation capabilities. This feature can prove useful in certain settings, for instance, in robotics research, where the reaction/adaptation time of the agents to dynamic environments bestow an inherent time limitation. However, this limitation is not present for the supervised learning problems, where MAML or its variants are utilized as a baseline. (\eg \cite{flennerhag2019,nichol2018,rajasegaran2020,collins2020,guiroy2019} etc.) Most of the unsupervised problem benchmark is image detection problem, where $N$-way $K$-shot classification problem ($N$ different classes with $K$ labeled training data) is tried to be tackled. Given the nature of the problem, most of the time memory or time limitation does not constitute a major issue in the given problem setting.

MAML aims to improve the generalization of a model for a certain learning task from a given family of tasks, with little data and minimal training. Minimal training indicates quick adaptation capabilities. This feature can prove useful in certain settings, for instance, in robotics research, where the reaction/adaptation time of the agents to dynamic environments bestow an inherent time limitation. However, this limitation is not present for supervised learning problems, where MAML or its variants are utilized as a baseline. (\eg \cite{flennerhag2019,nichol2018,rajasegaran2020,collins2020,guiroy2019} etc.) Most of the unsupervised problem benchmark is image detection problem, where $N$-way $K$-shot classification problem ($N$ different classes with $K$ labeled training data) is tried to be tackled. Given the nature of the problem, most of the time memory or time limitation does not constitute a major issue in the given problem setting.

% What is our paper about?
The main aim of this paper is to investigate the MAML under the settings where quick adaptation is not needed, and where most of the applications and variants of this method are benchmarked. This will be achieved by looking at the expected performance of the MAML under two synthetic regression scenarios, and comparing its performance to conventional base learners (\eg Linear Regression, Ridge Regression, Kernel Ridge Regression, etc.). By doing this we aim to investigate the effect of the limited adaptation step, and whether or not there is a benefit to this limitation. % In the meantime, the effect of task variance, and noisy observations (unlike the setting presented in \cite{finn2017}) will also be investigated.



\section{Model-Agnostic Meta-Learning (MAML)}\label{sec:maml}
  % MAML Aim -> Intermediate Model
MAML aims to obtain an intermediate model $\model({\param}_{\text{meta}})$ that can generalize well after adaptation with gradient descent to a dataset $\dataset$ observed from a new and unseen task $\task$ drawn from $\prob_\task$ where the number of training points $N$ and the number of iterations $\iter$ is limited.  

% How to obtain the intermediate 
In order to obtain $\param_{\text{meta}}$ first, a batch of tasks $\{\task_i\}_{i=1}^{M}$ from $\prob_{\task}$ is observed with each having a corresponding dataset $\{\dataset_i\}_{i=1}^{M}$. Then, the future gradients concerning each task are observed and gradient descent is utilized to get possible parameters $\param^\prime$ for each task and a gradient descent iteration is made by collecting all the possible parameters from an observed batch of tasks for the real parameter update. The general procedure for supervised learning problems is given in Algorithm \ref{alg:MAML}. Authors of \cite{finn2017} indicate that by using this procedure the model $\model({\param})_{\text{meta}}$ requires few gradient updates from a specific task, achieving good generalization performance. Examples of the intermediate model $\model({\param})_{\text{meta}}$ prediction with the observed tasks on the background can be seen in Figures \ref{fig:lintasks} and \ref{fig:nonlintasks}.

\begin{algorithm}
  \caption{MAML\cite{finn2017} Algorithm}\label{alg:MAML}
  \KwData{$\prob_{\task}$, $\alpha$, $\beta$}
  \KwResult{Intermediate Model $\model(\param_{meta})$}
  initialize $\param$ randomly; \\
  \While{not done}
  {
    sample a batch of tasks $\task_i$ from $\prob_{\task}$\\
    \ForAll{$\task_i$}
    {
      Obtain future gradients: $\grad{\param}\loss_{\task_i}(\model(\param))$ wrt. $\dataset_i$ \\
      Possible future parameters: $\param_i^\prime = \param_i -\alpha\grad{\param}\loss_{\task_i}(\model(\param))$
    }
    Update: $\param \leftarrow \param- \beta\grad{\param}\sum\loss_{\task_i}(\model(\param_i^\prime))$
  }
\end{algorithm}

%\begin{figure}[ht!]
%  \centering
%  \begin{subfigure}[b]{0.49\textwidth}
%    \centering
%    \includetikz{\textwidth}{Figures_v1/methods/lin_maml.tikz}
%    \caption{Linear Problem}
%    \label{fig:lin_maml}
%  \end{subfigure}
%  \begin{subfigure}[b]{0.49\textwidth}
%    \centering
%    \includetikz{\textwidth}{Figures_v1/methods/nonlin_maml.tikz}
%    \caption{Nonlinear Problem}
%    \label{fig:nonlin_maml}
%  \end{subfigure}
%  \caption{Visualizing the MAML intermediate model. 100 sample tasks drawn from $\prob_\task$ for both linear ($m=0$ and $c=1$) and nonlinear ($c_1=2$ and $c_2=2$) problems shown transparent and intermediate model trained (obtained from MAML algorithm) for 1D cases of the experimentation given with solid dark orange line. The linear and nonlinear problems are provided in Section \ref{sec:methods}.}
%\end{figure}



\section{Related Work}\label{sec:rw}
  
Since its publications, MAML \cite{Finn2017} opened a wide range of research path that tries to understand gradient-based meta-learning, which aims to find an intermediate model to be used for adaptation at a later stage. There is both empirical and theoretical work dedicated to the understanding of MAML.

The empirical works try to understand and improve the MAML algorithm. For example, the sensitivity of MAML to architectural details and the difficulty in the training of the MAML algorithm is addressed and tackled in \cite{Antoniou2019} under the name of MAML++. This method tries to provide a much more stable algorithm by overcoming the gradient instability, the absence of batch statistics accumulation, the shared bias per layer, constant learning rates utilized in the algorithm. Although, its training difficulties the generalization capabilities are empirically investigated in \cite{Guiroy2019a} and concluded that the generalization to new tasks can be related to the similarity between the trajectories encountered during meta-training and the adaptation procedures. Moreover, in the same work, $L_2$ regularization like regularization for the inner loop is proposed for improved generalization performance. 

Aside from empirical studies, some theoretical work tries to understand the convergence and the generalization of MAML over the past years. However, since the theoretical investigation of the over-parametrized neural network is quite cumbersome, most of the effort goes into understanding this algorithm in much milder settings with strong assumptions. For instance, in \cite{Khodak2019} various gradient-based meta-learning algorithms including MAML and its derivatives are investigated for convex optimization problems and their usefulness is shown compared to single-task learning. In addition, negative learning rates in the inner loop are shown to be optimal during the meta-training stage, theoretically, in \cite{Bernacchia2021} for a mixed linear regression problem. Moreover, in \cite{Collins2020b} the effect of task distribution is investigated for linear regression problems and show that when tasks seen in training are similar to to the ones seen at adaptation step. Finally, in \cite{Fallah2021} the generalization of MAML is investigated from the algorithmic stability and generalization bounds perspective and concluded that the MAML generalizes well even to an unseen task if the training and test task distributions are sufficiently close. 


Another line of work tries to come up with other meta-learning scenarios involving convex settings by construction, which is a quite rare setting in meta-learning paradigm. In \cite{Denevi2018a, Denevi2019} meta-learning models inspired by the biased regularization works \cite{Kuzborskij2017, Kuzborskij2017a}, where bias is tried to be learned from the task environment by minimizing the \textit{Transfer Risk} (Expected Loss), is proposed. Due to the convex nature of the problem, the theoretical foundation is also provided for the proposed models. For learning-to-learn and continual learning settings. On top of this work, \cite{Bai2020} investigate the need for \textit{train-test} split for the same models and concludes that the \textit{train-train} model achieves strictly better generalization performance for structured tasks in the setting of learning around a common mean problem presented in \cite{Denevi2018a}. However, the comparision of this method to other meta-learning methods (\eg MAML) are left out.


\paragraph{Our Contribution:} The \textit{Expected Loss} investigation for most of the meta-learning methods are left out for various reasons. We try to investigate this for two different meta learning approaches. On one hand, the loss MAML \cite{Finn2017} is trying to optimize is the loss that the model would make given a batch of tasks obtained from the environment of certain task distribution. The model parameters update is done by looking at the possible loss for each task if the model parameters are changed for a certain task. On the other hand, the methods proposed by \cite{Denevi2018a} are trying to optimize for the so-called \textit{Transfer Risk} over for the task distribution. The main contribution of this paper is to investigate in linear and nonlinear settings the average performance of the MAML algorithm by looking at its expected loss, which will then be compared to individual learning task performance where there is no information coming from the task environment. The reason that the expected loss of MAML is not investigated can be speculated to be the computational burden that it bestows upon the problem. Here, for the linear problem case, this problem is tried to be elevated utilizing NUMBA \cite{Lam2015}, which can create compiled code for Python. However, due to the inflexibility of the NUMBA, pure PyTorch \cite{Paszke2019} implementation, which is slower, is used for the nonlinear problem with a lesser degree of fidelity. 


\section{Experimental Setting}\label{sec:methods}
  Throughout this work uppercase bold letters (\eg $\mathbf{X}$), lowercase bold letters (\eg $\mathbf{x}$), and lowercase letters (\eg ${x}$) are used for matrices, vectors and scalars respectively. Moreover, the vectors are assumed to be stored in columns. Finally, the $\I_{D}$ represents a $D\times D$ identity matrix. Moreover, the $\ones_{D}$ and $\zeros_{D}$ represents $D\times 1$ vector of ones and zeros respectively.

\subsection{Learning Problems}

\subsubsection{Linear Problem}\label{sec:Linear}

Consider the conventional linear regression problem in $\R^D$ is given by
\begin{equation}\label{eq:linearreg}
  \lab = \trans{\inp}\scaletask+\noise
\end{equation}
where, $\lab\in\R$, $\inp\in\R^D$, $\scaletask\in\R^D$ and $\noise\sim\normal{0}{\var}$. Assuming that the each realization of a scale term $\scaletask$ corresponds to a task $\task$  observed in the environment and each set of observed $N$ input ($\inp$) and its corresponding label ($\lab$) is represented by a dataset $\dataset_{j}:=(\inp, \lab)_{i=1}^{N}$. 

\subsubsection{Nonlinear Problem}\label{sec:Nonlinear}

Consider a nonlinear regression problem in $\R^D$ is given by
\begin{equation}\label{eq:nonlinearreg}
  \lab = \trans{\sine(\inp+\phasetask)}\scaletask+\noise
\end{equation}
where, $\lab\in\R$, $\inp\in\R^D$, $\scaletask\in\R^D$ and $\noise\sim\normal{0}{\var}$. Assuming that the each realization of scale term $\scaletask$ and $\phasetask$ corresponds to a task observed in the environment $\task$ and each set of observed $N$ input ($\inp$) and its corresponding label ($\lab$) is represented by a dataset $\dataset_{j}:=(\inp_i, \lab_i)_{i=1}^{N}$.

\subsubsection{General Problem Setting}
For both problems presented in Sections \ref{sec:Linear} and \ref{sec:Nonlinear} sample distribution is given by $\prob_\dataset$ for a given $\task_{k}$ and the task distribution is represented by $\prob_\task$. A model parameterized by $\param$\footnote{Bar on top of the parameters are used to indicate the bias terms inclusion.} is represented by $\model(\inp, \param):\inp\to\lab$. An estimator that is trained with $\dataset_{j}$ that is obtained from the $\task_k$ is represented by $\estim(\inp)$. The discrepancy between the prediction of the estimator $\estim$ and $\lab$ is measured in terms of squared loss $\loss:=(\estim(x)-\lab)^2$. The main loss that this paper tries to investigate is the \textit{Expected Error} of an estimator $\estim$ over the $\prob_{\task}$. Then the expected error is represented as

\begin{equation}\label{eq:ee}
  \EE:= \iiint(\estim(x) - y)^2\prob(\inp, y)\prob_{\dataset}\prob_{\task} d\inp d\lab d\dataset d\task.
\end{equation}

For the defined expected error and the problem definitions, the \textit{Bayes Error} is given by $\sigma^2$ that is coming from the noise term, which represents a model that is the perfect estimator.

\subsubsection{Experimental Assumptions}
For all the problems the input distribution is given by $\prob_\inp\sim\normal{0,k\I}$ where $k$ is a parameter for the variance of the inputs. For the linear problem the $\prob_\task:=\prob(\scaletask)\sim\normal{{m\ones_{D}},{c\I_{D}}}$ and for nonlinear problem the task distribution takes the form of a joint distribution $\prob_\task:=\prob(\scaletask, \phasetask)$ where $\prob_{\scaletask}\sim\normal{\ones_{D},c_1\I_{D}}$ and $\prob_{\phasetask}\sim\normal{\zeros_{D},c_2\I_{D}}$


\begin{figure}[ht!]
  \centering
  \begin{subfigure}[b]{0.49\textwidth}
    \centering
    \includetikz{\textwidth}{Figures/methods/lin_eg.tikz}
    \caption{$\lab = \trans{\inp}\scaletask$}
    \label{fig:lintasks}
  \end{subfigure}
  \begin{subfigure}[b]{0.49\textwidth}
    \centering
    \includetikz{\textwidth}{Figures/methods/nonlin_eg.tikz}
    \caption{$\lab = \trans{\sine(\inp+\phasetask)}\scaletask$}
    \label{fig:nonlintasks}
  \end{subfigure}
  \caption{100 sample tasks drawn from $\prob_\task$ for both linear ($m=0$ and $c=1$) and nonlinear ($c_1=1$ and $c_2=1$) problems.}
\end{figure}

\subsection{Models} 

\subsubsection{Single Task Learning Models}

Assuming a linear model in the form of $\model(\inp,\slope,\bias):=\inp\slope+\bias$ or $\model(\inp, \param):=\inpbias\param$ with $\inp\in\R^{1\times D}$, $\slope\in\R^{D\times 1}$, $\inpbias\in\R^{1\times D+1}$ and $\param\in\R^{D+1\times}$ where $\param:=\trans{[\slope, \bias]}$ and $\inpbias:=[\inp, 1]$. The optimum collection of parameters ($\opt$) for different models are obtained as follow:

\paragraph{Linear Estimator} is given by the least-squares solution, $\opt:=(\inv{\trans{\design}\design)}\trans{\design}\labs$, where $\design\in\R^{N\times D}$ is the design matrix where the observed input data is stored in rows.

\paragraph{Ridge Estimator} is given by $\opt:=(\inv{\trans{\design}\design+\ridge \I_{D})}\trans{\design}\labs$ which is obtained by minimizing the squared loss with the additional term of $\ridge\norm{\param}{2}^2$. Thus, overall loss takes the form $\loss+\ridge\norm{\param}{2}^2$.


\paragraph{Kernel Ridge Estimator} is given by $\param = \trans{\design}\weight$ where $\weight:=\inv{(\gram+\ridge\I_{N})}\lab$ where $\gram\in\R^{N\times N}$ is the  \textit{Gram Matrix} obtained by replacing $\trans{\design}\design$ inner product by a kernel $\kernel(\design, \design)$. Then, the prediction of the estimator takes the form $\estim(\pred,\param)=\trans{\weight}\kernel(\pred,\design)$ where $\pred\in\R^{D\times 1}$.

\paragraph{Gradient Descent Estimator for the Linear Model} for a given number of iterations $\iter$ the gradient descent estimator is given by $\slope_{j+1}=\slope_{j} - \lr\sum_i^{N}\inp_i(\estim(\inp,\param_j)-\lab_i)_{j=0}^{\iter-1}$ and $\bias_{j+1}=\bias_{j} - \lr\sum_i^{N}(\estim(\inp,\param_j)-\lab_i)_{j=0}^{\iter-1}$.

\subsubsection{Meta Learning Models}

\paragraph{Model-Agnostic Meta-Learning (MAML)} aims to obtain an intermediate model that can generalize well after adaptation to a dataset $\dataset$ observed from a new and unseen task $\task_i$ drawn from $\prob_\task$ where the number of training points $K$ and the number of iterations $\iter$ is limited. The general procedure for supervised learning problems is given in Algorithm \ref{alg:MAML}.

\begin{algorithm}
  \caption{MAML\cite{Finn2017} Algorithm}\label{alg:MAML}
  \KwData{$\prob_{\task}$, $\alpha$, $\beta$}
  \KwResult{Intermediate Model $\model(\param_{meta})$}
  initialize $\param$ randomly; \\
  \While{not done}
  {
    sample a batch of tasks $\task_i$ from $\prob_{\task}$\\
    \ForAll{$\task_i$}
    {
      Obtain future gradients: $\grad{\param}\loss_{\task_i}(\model(\param))$ wrt. $\dataset_K$ \\
      Possible future parameters: $\param_i^\prime = \param_i -\alpha\grad{\param}\loss_{\task_i}(\model(\param))$
    }
    Update: $\param \leftarrow \param- \beta\grad{\param}\sum\loss_{\task_i}(\model(\param_i^\prime))$
  }
\end{algorithm}

\paragraph{General Ridge Estimator} is a version of \textit{Ridge Estimator} with the capability to penalize towards a given vector instead of a zero vector. It is given by $\opt:=(\inv{\trans{\design}\design+\ridge \I_{D})}(\trans{\design}\labs+\ridge\genridge)$ by minimizing the squared loss with the additional term $\ridge\norm{\param-\genridge}{2}^2$, where $\genridge\in\R^{D\times 1}$. Then, the overall loss takes the form $\loss+\ridge\norm{\param-\genridge}{2}^2$.

\begin{figure}[ht!]
  \centering
  \begin{subfigure}[b]{0.49\textwidth}
    \centering
    \includetikz{\textwidth}{Figures/methods/lin_maml.tikz}
    \caption{Linear Problem}
    \label{fig:lin_maml}
  \end{subfigure}
  \begin{subfigure}[b]{0.49\textwidth}
    \centering
    \includetikz{\textwidth}{Figures/methods/nonlin_maml.tikz}
    \caption{Nonlinear Problem}
    \label{fig:nonlin_maml}
  \end{subfigure}
  \caption{Visualizing the MAML intermediate model. 100 sample tasks drawn from $\prob_\task$ for both linear ($m=0$ and $c=1$) and nonlinear ($c_1=2$ and $c_2=2$) problems shown transparent and intermediate model trained (obtained from MAML algorithm) for 1D cases of the experimentation given with solid dark orange line.}
\end{figure}



\section{Results and Discussion}\label{sec:resdis}
  This section is dedicated to the expected performance results of a meta-learning model after adaptation and conventional base learners (\eg Linear, Ridge, and Kernel Ridge Regression models), to see their performance differences under certain scenarios induced by the experimental assumptions (\eg task variance, input variance, noise, and dimensionality). 
%Experiments conducted have various models some of which have information about the family of tasks (either due to meta-training or the initialization point being selected suitable to the task distribution), and others have only information from a single task at every step.

\subsubsection{Linear Problem}
The linear problem introduced in Section \ref{sec:methods} is characterized by the dimensionality $D$, number of training samples $N$, number of gradient steps $\iter$, the task variance $c$ and task mean $m$, and the variance of the input samples $k$. For the sake of brevity, only some of the parameters are discussed in this section. Unless the parameter in the configuration is under investigation, the default values are utilized. And, the default values for the experimentation $\sigma=1$, $m=0$, $k=1$, $c=1$, $\iter=1$.
Moreover, the number of tasks drawn ($N_{\task}$), and dataset draws ($N_{\dataset})$  for approximating the expected error given in Equation \ref{eq:ee} are taken to be $100$ each. Finally, the test set size is taken as 1000.

%\paragraph{Effect of the Number of Training Samples $N$:} The results of this experiment can be found in Figure \ref{fig:linear-N} for increasing problem dimensionality. It can be seen that the Linear model suffers from singularities, 10r instance in Figure \ref{fig:linear-N-D-10} when the number of samples equals dimensionality $N=D$. However, it can have comparable performance over all the selected problem dimensionalities. As one might expect as the dimensionality increases so do the difference between the models with analytical optimum and gradient descent utilizing methods. Moreover, for the increasing training samples case the Ridge regression variants perform much better as for all the cases they Converge towards the Bayes error. However, the gradient descent variant models, where there exists task information (\eg MAML, GD), are unable to converge towards the Bayes error. This can be attributed to the regularizing effect of the limited gradient steps ($\iter$) allowed for the models. Overall, the improvement that the additional task-related information brings to the gradient-based models, as the random GD model is orders of magnitude higher than expected performance. Although, task information inclusion decreases the expected performance as the number of gradient step limitations hinders the gradient-based models' capability to decrease the expected performance further.

\paragraph{Effect of Number of Gradient Steps $\iter$:} It can be observed from Figure \ref{fig:linear-n_iter-N-1-D-1} that for a low number of training samples the gradient steps taken have little to no influence. But as the number of training samples increases for a given problem dimensionality the effect $\iter$ on the expected performance gets much more prominent. It is evident that MAML decreases the number of gradient steps needed for convergence. Moreover, for $D=N$ case it even improves generalization after convergence too (see Figures \ref{fig:linear-n_iter-N-10-D-10} and \ref{fig:linear-n_iter-N-1-D-1}). Overall, it can be observed that the increasing $\iter$ converges towards the Ridge model variants with the exception of the $D=1$ and $N=1$ cases.
 
\begin{figure}[h!]
  \centering
    \begin{subfigure}{0.3\textwidth}
      \centering
      \includetikz{\textwidth}{Figures_v2/linres/n_iter/n_iter-1-1-x-0.tikz}
      \caption{$D=1$, $N=1$}
      \label{fig:linear-n_iter-N-1-D-1}
    \end{subfigure}
    \begin{subfigure}{0.3\textwidth}
      \centering
      \includetikz{\textwidth}{Figures_v2/linres/n_iter/n_iter-1-10-x-0.tikz}
      \caption{$D=1$, $N=10$}
      \label{fig:linear-n_iter-N-10-D-1}
    \end{subfigure}
    \begin{subfigure}{0.3\textwidth}
      \centering
      \includetikz{\textwidth}{Figures_v2/linres/n_iter/n_iter-1-50-x-0.tikz}
      \caption{$D=1$, $N=50$}
      \label{fig:linear-n_iter-N-50-D-1}
    \end{subfigure}

    \begin{subfigure}{0.3\textwidth}
      \centering
      \includetikz{\textwidth}{Figures_v2/linres/n_iter/n_iter-10-10-x-0.tikz}
      \caption{$D=10$, $N=10$}
      \label{fig:linear-n_iter-N-10-D-10}
    \end{subfigure}
    \begin{subfigure}{0.3\textwidth}
      \centering
      \includetikz{\textwidth}{Figures_v2/linres/n_iter/n_iter-50-10-x-0.tikz}
      \caption{$D=50$, $N=10$}
      \label{fig:linear-n_iter-N-10-D-50}
    \end{subfigure}  

  %\caption{The expected performance for changing the number of gradient steps $\iter$ with various training samples and various dimensional problems. For the given parameter the effect of increasing number of training samples can be seen by looking at Figures \ref{fig:linear-n_iter-N-1-D-1}, \ref{fig:linear-n_iter-N-10-D-1}, \ref{fig:linear-n_iter-N-50-D-1} and the effect of increasing dimensionality can be seen by looking at Figures \ref{fig:linear-n_iter-N-10-D-1}, \ref{fig:linear-n_iter-N-10-D-10}, \ref{fig:linear-n_iter-N-10-D-50}.}
  \caption{The expected error for the increasing number of gradient steps $\iter$ used for adaptation when changing the number of training samples for various problems of different dimensions.}
  \label{fig:linear-n_iter}
\end{figure}


\paragraph{Effect of the Number of Training Samples $N$:} Results of this experiment can be found in Figure \ref{fig:linear-N} for increasing problem dimensionality. It can be seen that the Linear model suffers from singularities. For instance, in Figure \ref{fig:linear-N-D-10} when the number of samples equals dimensionality $N=D$. However, it is able to have comparable performance over all the selected problem dimensionalities. For the increasing training samples case, the Ridge model performs much better as all the cases converge towards the Bayes error. However, MAML is unable to converge towards the Bayes error. This can be attributed to the regularizing effect of the limited gradient steps ($\iter$) allowed for the models. Overall, the improvement that the additional task-related information brings to the gradient-based models is not visible, as the random GD model is orders of magnitude higher than expected performance. Although task information provides gain over the random initialization, the expected performance is hindered for the gradient-based models.

\begin{figure}[!h]
  \centering
    \begin{subfigure}{0.3\textwidth}
      \centering
      \includetikz{\textwidth}{Figures_v2/linres/N/Ntrn-1-1-x-0.tikz}
      \caption{$D=1$}
      \label{fig:linear-N-D-1}
    \end{subfigure}
    \begin{subfigure}{0.3\textwidth}
      \centering
      \includetikz{\textwidth}{Figures_v2/linres/N/Ntrn-10-1-x-0.tikz}
      \caption{$D=10$}
      \label{fig:linear-N-D-10}
    \end{subfigure}
    \begin{subfigure}{0.3\textwidth}
      \centering
      \includetikz{\textwidth}{Figures_v2/linres/N/Ntrn-50-1-x-0.tikz}
      \caption{$D=50$}
      \label{fig:linear-N-D-50}
    \end{subfigure}
  \caption{The expected error for the increasing number of training samples and problem dimensionality.}\label{fig:linear-N}
\end{figure}


%\paragraph{Effect of Dimensionality $D$:} These results are quite similar to the ones obtained with the experiments investigating the effect of training samples. It can be observed from Figure \ref{fig:linear-D} that, aside from singularities the Linear model variants yield lower expected performance compared to the gradient descent variants and this gap increases as the dimensionality of the problem or the number of training samples increases. Looking at Figure \ref{fig:linear-N-D-10} it can be observed that for the number of training samples around the dimensionality of the problem Ridge model performs much better, but as the dimensionality of the problem increases so does the gap between the performances of the Ridge model and the gradient descent variants.
%
%\begin{figure}[!h]
%  \centering
%    \begin{subfigure}{0.3\textwidth}
%      \centering
%      \includetikz{\textwidth}{Figures_v2/linres/dim/dim-1-x-0.tikz}
%      \caption{$N=1$}
%      \label{fig:linear-D-N-1}
%    \end{subfigure}
%    \begin{subfigure}{0.3\textwidth}
%      \centering
%      \includetikz{\textwidth}{Figures_v2/linres/dim/dim-10-x-0.tikz}
%      \caption{$N=10$}
%      \label{fig:linear-D-N-10}
%    \end{subfigure}
%    \begin{subfigure}{0.3\textwidth}
%      \centering
%      \includetikz{\textwidth}{Figures_v2/linres/dim/dim-50-x-0.tikz}
%      \caption{$N=50$}
%      \label{fig:linear-D-N-50}
%    \end{subfigure}
%  \caption{The expected error for increasing problem dimensionality and the number of training points.}\label{fig:linear-D}
%\end{figure}


\paragraph{Effect of Task Variance $c$:} The results of increasing task variance for various problem dimensions and various numbers of training samples can be found in Figure \ref{fig:linear-c}. The most obvious observation is that for all the models that utilize gradient descent, expected performance increases, whereas the Linear model and the Ridge model are only affected by this phenomenon for problem dimensionality $D\geq N$. In light of this observation, another important result is that for $N\geq D$ and for small task variance the gradient descent variants (except the randomly initialized model) have lower expected performance than the Ridge model. However, this performance diminishes with increasing problem dimensionality and the increasing number of training points. It is interesting to see a better performance from just one gradient step. That is why an extra mini-experimentation is done for the GD and MAML models to investigate if there is a performance improvement with the additional gradient steps. The results of this experimentation are given in Table \ref{tab:zoom}. It is observed from the table that there exists a point at which the gradient steps are hurting the expected performance one would get in this range after the second gradient step. Then, it can be conjectured that the number of gradient steps has a regularizing effect on the task distributions with small variance. Despite, the surprising results of MAML-like algorithms, the Ridge model is much more stable and performs better than gradient-based methods for $N\geq D$.

\begin{table}
  \centering
  \caption{Mean expected performance for the range $c:[0,1]$ range with various gradient steps for the MAML and GD models with $\lr=0.334$. Note that only $D=1$, $N=10$ case (see Figure \ref{fig:linear-c-N-10-D-1}) is presented.}\label{tab:zoom}
  \begin{tabular}{c|c|c|c|c|c|c|c|c|c|c|c|}
    \cline{2-11}
     & \multicolumn{10}{|c|}{$\iter$}\\
    \cline{2-11}
     & 1 & 2 & 3 & 4 & 5 & 6 & 7 & 8 & 9 & 10\\
    \hline
    \multicolumn{1}{|c|}{MAML} & 1.2132 & \textbf{1.1938} & 1.2067 & 1.2171 & 1.2318 & 1.2476 & 1.2773 & 1.3330 & 1.4622 & 1.7556  \\
    \hline
    \end{tabular}
\end{table}

\begin{figure}[!h]
  \centering
    \begin{subfigure}{0.33\textwidth}
      \centering
      \includetikz{\textwidth}{Figures_v2/linres/c/c-1-1-x-0.tikz}
      \caption{$D=1$, $N=1$}
      \label{fig:linear-c-N-1-D-1}
    \end{subfigure}
    \begin{subfigure}{0.33\textwidth}
      \centering
      \includetikz{\textwidth}{Figures_v2/linres/c/c-1-10-x-0.tikz}
      \caption{$D=1$, $N=10$}
      \label{fig:linear-c-N-10-D-1}
    \end{subfigure}
    \begin{subfigure}{0.33\textwidth}
      \centering
      \includetikz{\textwidth}{Figures_v2/linres/c/c-1-50-x-0.tikz}
      \caption{$D=1$, $N=50$}
      \label{fig:linear-c-N-50-D-1}
    \end{subfigure}

    \begin{subfigure}{0.33\textwidth}
      \centering
      \includetikz{\textwidth}{Figures_v2/linres/c/c-10-10-x-0.tikz}
      \caption{$D=10$, $N=10$}
      \label{fig:linear-c-N-10-D-10}
    \end{subfigure}
    \begin{subfigure}{0.33\textwidth}
      \centering
      \includetikz{\textwidth}{Figures_v2/linres/c/c-50-10-x-0.tikz}
      \caption{$D=50$, $N=10$}
      \label{fig:linear-c-N-10-D-50}
    \end{subfigure}  

  \caption{The expected error for increasing task variance $c$ when changing the number of training samples for various problems of different dimensions.}
  \label{fig:linear-c}
  %For the given parameter the effect of increasing number of training samples can be seen by looking at Figures \ref{fig:linear-c-N-1-D-1}, \ref{fig:linear-c-N-10-D-1}, \ref{fig:linear-c-N-50-D-1} and the effect of increasing dimensionality can be seen by looking at Figures \ref{fig:linear-c-N-10-D-1}, \ref{fig:linear-c-N-10-D-10}, \ref{fig:linear-c-N-10-D-50}.}
\end{figure}


\paragraph{Effect of Task Mean $m$:} The results can be seen in Figure \ref{fig:linear-m}. The most important observation from this experimentation is that the Ridge model has increasing expected performance for the cases of $N\leq D$ cases (see Figures \ref{fig:linear-m-N-1-D-1}, \ref{fig:linear-m-N-10-D-10} and \ref{fig:linear-m-N-10-D-50}) and mostly the best $\lambda$ from the trials is found to be the lowest value, which makes the Ridge model behave similar to the Linear model. Furthermore, other models which have prior task information do not seem to be affected by the task mean shifting in the task space, as expected. Again, the superiority of including information from the task space is evident as the conventional regularization cannot deal with the changing task distribution mean for $N\leq D$.

\begin{figure}[!h]
  \centering
    \begin{subfigure}{0.3\textwidth}
      \centering
      \includetikz{\textwidth}{Figures_v2/linres/m/m-1-1-x-0.tikz}
      \caption{$D=1$, $N=1$}
      \label{fig:linear-m-N-1-D-1}
    \end{subfigure}
    \begin{subfigure}{0.3\textwidth}
      \centering
      \includetikz{\textwidth}{Figures_v2/linres/m/m-1-10-x-0.tikz}
      \caption{$D=1$, $N=10$}
      \label{fig:linear-m-N-10-D-1}
    \end{subfigure}
    \begin{subfigure}{0.3\textwidth}
      \centering
      \includetikz{\textwidth}{Figures_v2/linres/m/m-1-50-x-0.tikz}
      \caption{$D=1$, $N=50$}
      \label{fig:linear-m-N-50-D-1}
    \end{subfigure}

    \begin{subfigure}{0.3\textwidth}
      \centering
      \includetikz{\textwidth}{Figures_v2/linres/m/m-10-10-x-0.tikz}
      \caption{$D=10$, $N=10$}
      \label{fig:linear-m-N-10-D-10}
    \end{subfigure}
    \begin{subfigure}{0.3\textwidth}
      \centering
      \includetikz{\textwidth}{Figures_v2/linres/m/m-50-10-x-0.tikz}
      \caption{$D=50$, $N=10$}
      \label{fig:linear-m-N-10-D-50}
    \end{subfigure}  

  %\caption{The expected performance for changing task mean $m$ with various training samples and various dimensional problems. For the given parameter the effect of increasing number of training samples can be seen by looking at Figures \ref{fig:linear-m-N-1-D-1}, \ref{fig:linear-m-N-10-D-1}, \ref{fig:linear-m-N-50-D-1} and the effect of increasing dimensionality can be seen by looking at Figures \ref{fig:linear-m-N-10-D-1}, \ref{fig:linear-m-N-10-D-10}, \ref{fig:linear-m-N-10-D-50}.}
  \caption{The expected error for increasing task mean $m$ when changing the number of training samples for various problems of different dimensions.}
  \label{fig:linear-m}
\end{figure}

%%%%%%%%%%%%%%%%%%%%%%%%%%%%%%%%%%%%%%%%%%%%%%%%%%%%%%%%%%%%%%%%%%%%%%%%%%%%%%%
\subsubsection{Nonlinear Problem}

The nonlinear problem introduced in Section \ref{sec:methods} has the parameters controlling the dimensionality $D$, number of training samples $N$, number of gradient steps $\iter$, the task variances and means $m_1$ and $m_2$, $c_1$ and $c_2$, and the variance of the input samples $k$. Note that only some of the parameters are discussed in this section. Unless the parameter in the configuration is under investigation, the default values are utilized. And, the default values are given as $\sigma=1$, $m_1=1$, $m_2=0$, $c_1=2$, $c_2=2$, $k=1$, $\iter=5$. Moreover, the number of tasks drawn ($N_{\task}$), and dataset draws ($N_{\dataset})$  for approximating the expected error given in Equation \ref{eq:ee} are taken to be $50$ each. Finally, the test set size is taken as 1000.

\paragraph{Effect of Number of Gradient Steps $\iter$:} It can be seen from Figure \ref{fig:nonlinear-n_iter-N-1-D-1} that a single realization of the Kernel Ridge model can have a lower expected error for an extreme value of $1$ training sample for the 1-dimensional problem. However, as the number of training samples increases for a given problem dimensionality MAML model starts showing a lower expected error (see Figure \ref{fig:nonlinear-n_iter-N-10-D-1}). Moreover, the further increase in training samples to $50$ lowers the difference in expected performance for given models. Another interesting observation is, that for $N\leq D$ Kernel Ridge model can achieve a lower expected error, and for all the other cases one might find a better MAML model given that sufficient gradient steps are allowed. Moreover, it can be observed that the difference between random GD and MAML is low in terms of expected performance for most of the presented problems. Finally, in cases where MAML outperforms Kernel Ridge the number of gradient steps required to surpass its expected performance is low.

\begin{figure}[!htb]
  \centering
    \begin{subfigure}{0.3\textwidth}
      \centering
      \includetikz{\textwidth}{Figures_v2/nonlinres/n_iter/n_iter-1-1-x-0.tikz}
      \caption{$D=1$, $N=1$}
      \label{fig:nonlinear-n_iter-N-1-D-1}
    \end{subfigure}
    \begin{subfigure}{0.3\textwidth}
      \centering
      \includetikz{\textwidth}{Figures_v2/nonlinres/n_iter/n_iter-1-10-x-0.tikz}
      \caption{$D=1$, $N=10$}
      \label{fig:nonlinear-n_iter-N-10-D-1}
    \end{subfigure}
    \begin{subfigure}{0.3\textwidth}
      \centering
      \includetikz{\textwidth}{Figures_v2/nonlinres/n_iter/n_iter-1-50-x-0.tikz}
      \caption{$D=1$, $N=50$}
      \label{fig:nonlinear-n_iter-N-50-D-1}
    \end{subfigure}

    \begin{subfigure}{0.3\textwidth}
      \centering
      \includetikz{\textwidth}{Figures_v2/nonlinres/n_iter/n_iter-10-10-x-0.tikz}
      \caption{$D=10$, $N=10$}
      \label{fig:nonlinear-n_iter-N-10-D-10}
    \end{subfigure}
    \begin{subfigure}{0.3\textwidth}
      \centering
      \includetikz{\textwidth}{Figures_v2/nonlinres/n_iter/n_iter-50-10-x-0.tikz}
      \caption{$D=50$, $N=10$}
      \label{fig:nonlinear-n_iter-N-10-D-50}
    \end{subfigure}  
%  \caption{The expected performance for changing the number of gradient steps $\iter$ with various training samples and various dimensional problems. For the given parameter the effect of increasing number of training samples can be seen by looking at Figures \ref{fig:nonlinear-n_iter-N-1-D-1}, \ref{fig:nonlinear-n_iter-N-10-D-1}, \ref{fig:nonlinear-n_iter-N-50-D-1} and the effect of increasing dimensionality can be seen by looking at Figures \ref{fig:nonlinear-n_iter-N-10-D-1}, \ref{fig:nonlinear-n_iter-N-10-D-10}, \ref{fig:nonlinear-n_iter-N-10-D-50}.}
  \caption{The expected error for the increasing number of gradient steps $\iter$ used for adaptation when changing the number of training samples for various problems of different dimensions.}
  \label{fig:nonlinear-n_iter}
\end{figure}


\paragraph{Effect of Number of Training Samples $N$:} By looking at Figure \ref{fig:nonlinear-N} it can be seen that for all the given dimensionalities there exists a training sample amount where the expected error of the Kernel Ridge model is higher than MAML. Another observation is that the randomly initialized model average performance is stable overall number of training samples. 

\begin{figure}[!h]
  \centering
    \begin{subfigure}{0.3\textwidth}
      \centering
      \includetikz{\textwidth}{Figures_v2/nonlinres/N/Ntrn-1-1-x-0.tikz}
      \caption{$D=1$}
      \label{fig:nonlinear-N-D-1}
    \end{subfigure}
    \begin{subfigure}{0.3\textwidth}
      \centering
      \includetikz{\textwidth}{Figures_v2/nonlinres/N/Ntrn-10-1-x-0.tikz}
      \caption{$D=10$}
      \label{fig:nonlinear-N-D-10}
    \end{subfigure}
    \begin{subfigure}{0.3\textwidth}
      \centering
      \includetikz{\textwidth}{Figures_v2/nonlinres/N/Ntrn-50-1-x-0.tikz}
      \caption{$D=50$}
      \label{fig:nonlinear-N-D-50}
    \end{subfigure}
  \caption{The expected error for the increasing number of training samples and problem dimensionality.}\label{fig:nonlinear-N}
\end{figure}


\paragraph{Effect of Phase Task Variance $c_2$:} Remembering that the task variance effect for the linear problem had some interesting properties where even a single gradient step resulted in better expected performance. One might wonder if that is the case for the nonlinear problem as well. As can be seen in Figure \ref{fig:nonlinear-c2-N-10-D-1} a similar effect is observed for the nonlinear problem. However, for this case, the increased expected performance of MAML is only better than a randomly initialized model. This indicates that a better performance can be achieved with a regularization-based conventional learner. The only clear advantage of utilizing MAML is seen in the case where there is only $1$ training sample. However, even the randomly initialized model performs better than the Kernel Ridge model for the case and what MAML adds is just a slight improvement. 

\begin{figure}[!h]
  \centering
    \begin{subfigure}{0.33\textwidth}
      \centering
      \includetikz{\textwidth}{Figures_v2/nonlinres/c_phase/c_phase-1-1-x-0.tikz}
      \caption{$D=1$, $N=1$}
      \label{fig:nonlinear-c2-N-1-D-1}
    \end{subfigure}
    \begin{subfigure}{0.33\textwidth}
      \centering
      \includetikz{\textwidth}{Figures_v2/nonlinres/c_phase/c_phase-1-10-x-0.tikz}
      \caption{$D=1$, $N=10$}
      \label{fig:nonlinear-c2-N-10-D-1}
    \end{subfigure}
    \begin{subfigure}{0.33\textwidth}
      \centering
      \includetikz{\textwidth}{Figures_v2/nonlinres/c_phase/c_phase-1-50-x-0.tikz}
      \caption{$D=1$, $N=50$}
      \label{fig:nonlinear-c2-N-50-D-1}
    \end{subfigure}

    \begin{subfigure}{0.33\textwidth}
      \centering
      \includetikz{\textwidth}{Figures_v2/nonlinres/c_phase/c_phase-10-10-x-0.tikz}
      \caption{$D=10$, $N=10$}
      \label{fig:nonlinear-c2-N-10-D-10}
    \end{subfigure}
    \begin{subfigure}{0.33\textwidth}
      \centering
      \includetikz{\textwidth}{Figures_v2/nonlinres/c_phase/c_phase-50-10-x-0.tikz}
      \caption{$D=50$, $N=10$}
      \label{fig:nonlinear-c2-N-10-D-50}
    \end{subfigure}  

  %\caption{The expected performance for changing the number of training samples for various dimensional problems. For the given parameter the effect of increasing number of training samples can be seen by looking at Figures \ref{fig:nonlinear-c2-N-1-D-1}, \ref{fig:nonlinear-c2-N-10-D-1}, \ref{fig:nonlinear-c2-N-50-D-1} and the effect of increasing dimensionality can be seen by looking at Figures \ref{fig:nonlinear-c2-N-10-D-1}, \ref{fig:nonlinear-c2-N-10-D-10}, \ref{fig:nonlinear-c2-N-10-D-50}.}

  \caption{The expected error for increasing task variance for phase $c_2$ used for adaptation when changing the number of training samples for various problems of different dimensions.}
  \label{fig:nonlinear-c2}
\end{figure}

Additional experimentation results for $\sigma$ for linear and nonlinear problems can be found in the Appendix, it is observed that increasing noise has similar behavior with single task learning models. Moreover, the effect of input variance is investigated and found that the gradient descent based methods perform poorer for the linear problem. 


\subsection{Discussion}

% Task variance and limited number of gradient steps 
Upon our investigation, it is found empirically that meta-information about the task space can help the generalization performance in linear and nonlinear problem settings even with limited gradient steps. Increased generalization performance of MAML compared to single task learning models on expectation when the tasks that are in consideration are close to each other is observed, where the same observation is made theoretically in \cite{fallah2021}. This observation suggests that there is a regularizing effect of limiting the gradient steps used for adaptation. We conjecture that after the meta-learning stage intermediate model parameters $\param$ are closer to the test set optimum compared to the proximity of train and test set optimums. This type of behavior is investigated in \cite{nakkiran2020} as well, where the large learning rate in the training phase acts as a regularizer due to the discrepancy between train and test loss landscapes. 

This limitation of adaptation steps is noted in \cite{behl2019,li2017b} that tries to improve the MAML adaptation step so that the adaptation is limited to fewer gradient steps, preferably one. Our findings suggest that the expected performance of these methods should be investigated as well as some of the generalization power of MAML might be coming from the regularization induced by not optimizing the training loss perfectly. This hypothesis is supported by the findings of \cite{raghu2020} which concludes that the performance gain of MAML is about feature reuse instead of rapid learning.


%It is observed that the MAML in linear and nonlinear problem settings single task learning might provide extra generalization performance although the gradient steps are limited. It should be noted that this conjecture is valid only for the setting where the tasks are similar to each other, in other words, task variance is small. For instance, for the linear problem it is found that the MAML variants perform better only when the task distribution variance is small enough and for almost all the other cases, the limitation of the number of gradient steps results in hindered performance in terms of generalization. Furthermore, there exists a clear optimum for the number of gradient steps taken for adaptation ($\iter$). Hence, in some cases, the MAML variant methods can perform worse than single task learning methods on expectation. In the nonlinear problem setting, a single task learning Kernel Ridge model is found to be competitive with a meta-learning approach especially for the increasing dimensionality the difference between the meta-learning algorithm and single task learning algorithm tends to be small in expectation. However, this competitive behavior is only the case when there are enough training points. Finally, both linear and nonlinear conventional regularization can provide competitive results to a meta-learning algorithm with limited gradient steps for the adaptation phase in a supervised regression setting. In the linear setting, the Ridge model suffers due to regularization being towards $\mathbf{0}$, which means the regularization cannot deal with every task distribution. However, the biased regularization strategy, which can be utilized as a meta-learning algorithm too \cite{denevi2018}, will be able to tackle this issue.
%
%It is observed that the generalization improvement for the MAML variants uses a few gradient steps after observing a few data from the task we are interested in. This validates some of the findings in \cite{fallah2021}, where it is found that under strong convexity assumptions if the training and test task distributions are close enough generalization performance improves. Moreover, this finding is also observed in the non-convex setting. Considering the nonlinear problem setting number of gradient steps as there is a clear optimum expected error for $\iter$ and the same case was found to be the case for the linear problem as shown in Table \ref{tab:zoom}. Furthermore, in the linear problem setting the effect of the stopping point is investigated by means of starting from the exact mean of the tasks compared to the start in the region of the optimum. It can be seen from all the linear problems that there is no observable performance gain or loss regarding the starting point of the MAML variant methods. 



\section{Conclusion}\label{sec:conc}
  \begin{itemize}
  \item We can accurately predict learning curves with a data-driven approach.
  \item Talk about the caveats: Limitations of the approach, time complexity, choices that have to be made like, how many components, FPCA limitations, and the drawbacks you might observe during the re-experimentation.
  \item Talk about possible extensions like Gaussian Process extension for adding uncertainty to our predictions, might get over-confidence though from my experience with multi-fidelity regression. One additional remark would be to penalize the $\boldsymbol{\beta}$'s. This might pave the way for using the raw learning curves in the prediction without being affected by the noisiness of it. But an additional burden on the model selection can make things a bit difficult again.
\end{itemize}

%\section*{Acknowledgments}

%Bibliography
\bibliographystyle{splncs04}  
%\bibliographystyle{unsrt}  
\bibliography{/home/taylanot/Dropbox/archive_bib/mylibrary.bib}  
%\newpage
\section{Appendix}
  \begin{figure}[!ht]
  \centering
    \begin{subfigure}{0.3\textwidth}
      \centering
      \includetikz{\textwidth}{Figures_v2/linres/std_y/std_y-1-1-x-0.tikz}
      \caption{$D=1$, $N=1$}
      \label{fig:linear-std_y-N-1-D-1}
    \end{subfigure}
    \begin{subfigure}{0.3\textwidth}
      \centering
      \includetikz{\textwidth}{Figures_v2/linres/std_y/std_y-1-10-x-0.tikz}
      \caption{$D=1$, $N=10$}
      \label{fig:linear-std_y-N-10-D-1}
    \end{subfigure}
    \begin{subfigure}{0.3\textwidth}
      \centering
      \includetikz{\textwidth}{Figures_v2/linres/std_y/std_y-1-50-x-0.tikz}
      \caption{$D=1$, $N=50$}
      \label{fig:linear-std_y-N-50-D-1}
    \end{subfigure}

    \begin{subfigure}{0.3\textwidth}
      \centering
      \includetikz{\textwidth}{Figures_v2/linres/std_y/std_y-10-10-x-0.tikz}
      \caption{$D=10$, $N=10$}
      \label{fig:linear-std_y-N-10-D-10}
    \end{subfigure}
    \begin{subfigure}{0.3\textwidth}
      \centering
      \includetikz{\textwidth}{Figures_v2/linres/std_y/std_y-50-10-x-0.tikz}
      \caption{$D=50$, $N=10$}
      \label{fig:linear-std_y-N-10-D-50}
    \end{subfigure}  

%  \caption{[\textbf{Linear Problem}] The Expected Error for varying noise standard deviation $\sigma$ and increasing training samples and different problem. For the given parameter the effect of increasing number of training samples can be seen by looking at Figures \ref{fig:linear-std_y-N-1-D-1}, \ref{fig:linear-std_y-N-10-D-1}, \ref{fig:linear-std_y-N-50-D-1} and the effect of increasing dimensionality can be seen by looking at Figures \ref{fig:linear-std_y-N-10-D-1}, \ref{fig:linear-std_y-N-10-D-10}, \ref{fig:linear-std_y-N-10-D-50}.}\label{fig:linear-std_y}
  \caption{[\textbf{Linear Problem}]: The expected error for increasing noise standard deviation $\sigma$  when changing the number of training samples for various problems of different dimensions.}
  \label{fig:linear-std_y}
\end{figure}

\begin{figure}[!h]
  \centering
    \begin{subfigure}{0.3\textwidth}
      \centering
      \includetikz{\textwidth}{Figures_v2/linres/b/b-1-1-x-0.tikz}
      \caption{$D=1$, $N=1$}
      \label{fig:linear-b-N-1-D-1}
    \end{subfigure}
    \begin{subfigure}{0.3\textwidth}
      \centering
      \includetikz{\textwidth}{Figures_v2/linres/b/b-1-10-x-0.tikz}
      \caption{$D=1$, $N=10$}
      \label{fig:linear-b-N-10-D-1}
    \end{subfigure}
    \begin{subfigure}{0.3\textwidth}
      \centering
      \includetikz{\textwidth}{Figures_v2/linres/b/b-1-50-x-0.tikz}
      \caption{$D=1$, $N=50$}
      \label{fig:linear-b-N-50-D-1}
    \end{subfigure}

    \begin{subfigure}{0.3\textwidth}
      \centering
      \includetikz{\textwidth}{Figures_v2/linres/b/b-10-10-x-0.tikz}
      \caption{$D=10$, $N=10$}
      \label{fig:linear-b-N-10-D-10}
    \end{subfigure}
    \begin{subfigure}{0.3\textwidth}
      \centering
      \includetikz{\textwidth}{Figures_v2/linres/b/b-50-10-x-0.tikz}
      \caption{$D=50$, $N=10$}
      \label{fig:linear-b-N-10-D-50}
    \end{subfigure}  

  %\caption{[\textbf{Linear Problem}] The Expected Error for changing input variance  $k$ with various training samples and various dimensional problems. For the given parameter the effect of increasing number of training samples can be seen by looking at Figures \ref{fig:linear-b-N-1-D-1}, \ref{fig:linear-b-N-10-D-1}, \ref{fig:linear-b-N-50-D-1} and the effect of increasing dimensionality can be seen by looking at Figures \ref{fig:linear-b-N-10-D-1}, \ref{fig:linear-b-N-10-D-10}, \ref{fig:linear-b-N-10-D-50}.}\label{fig:linear-b}
  \caption{[\textbf{Linear Problem}]: The expected error for increasing input variance $k$  when changing the number of training samples for various problems of different dimensions.}
  \label{fig:linear-b}
\end{figure}



\begin{figure}[!h]
  \centering
    \begin{subfigure}{0.3\textwidth}
      \centering
      \includetikz{\textwidth}{Figures_v2/nonlinres/std_y/std_y-1-1-x-0.tikz}
      \caption{$D=1$, $N=1$}
      \label{fig:nonlinear-std_y-N-1-D-1}
    \end{subfigure}
    \begin{subfigure}{0.3\textwidth}
      \centering
      \includetikz{\textwidth}{Figures_v2/nonlinres/std_y/std_y-1-10-x-0.tikz}
      \caption{$D=1$, $N=10$}
      \label{fig:nonlinear-std_y-N-10-D-1}
    \end{subfigure}
    \begin{subfigure}{0.3\textwidth}
      \centering
      \includetikz{\textwidth}{Figures_v2/nonlinres/std_y/std_y-1-50-x-0.tikz}
      \caption{$D=1$, $N=50$}
      \label{fig:nonlinear-std_y-N-50-D-1}
    \end{subfigure}

    \begin{subfigure}{0.3\textwidth}
      \centering
      \includetikz{\textwidth}{Figures_v2/nonlinres/std_y/std_y-10-10-x-0.tikz}
      \caption{$D=10$, $N=10$}
      \label{fig:nonlinear-std_y-N-10-D-10}
    \end{subfigure}
    \begin{subfigure}{0.3\textwidth}
      \centering
      \includetikz{\textwidth}{Figures_v2/nonlinres/std_y/std_y-50-10-x-0.tikz}
      \caption{$D=50$, $N=10$}
      \label{fig:nonlinear-std_y-N-10-D-50}
    \end{subfigure}  

  %\caption{[\textbf{Nonlinear Problem}] The Expected Error for changing noise standard deviation $\sigma$ with various training samples and various dimensional problems. For the given parameter the effect of increasing number of training samples can be seen by looking at Figures \ref{fig:nonlinear-std_y-N-1-D-1}, \ref{fig:nonlinear-std_y-N-10-D-1}, \ref{fig:nonlinear-std_y-N-50-D-1} and the effect of increasing dimensionality can be seen by looking at Figures \ref{fig:nonlinear-std_y-N-10-D-1}, \ref{fig:nonlinear-std_y-N-10-D-10}, \ref{fig:nonlinear-std_y-N-10-D-50}.}\label{fig:nonlinear-std_y}
  \caption{[\textbf{Nonlinear Problem}]: The expected error for increasing noise standard deviation $\sigma$  when changing the number of training samples for various problems of different dimensions.}
  \label{fig:nonlinear-std_y}
\end{figure}

%\begin{figure}[!h]
%  \centering
%    \begin{subfigure}{0.3\textwidth}
%      \centering
%      \includetikz{\textwidth}{Figures_v2/nonlinres/c_amplitude/c_amplitude-1-1-x-0.tikz}
%      \caption{$D=1$, $N=1$}
%      \label{fig:nonlinear-c1-N-1-D-1}
%    \end{subfigure}
%    \begin{subfigure}{0.3\textwidth}
%      \centering
%      \includetikz{\textwidth}{Figures_v2/nonlinres/c_amplitude/c_amplitude-1-10-x-0.tikz}
%      \caption{$D=1$, $N=10$}
%      \label{fig:nonlinear-c1-N-10-D-1}
%    \end{subfigure}
%    \begin{subfigure}{0.3\textwidth}
%      \centering
%      \includetikz{\textwidth}{Figures_v2/nonlinres/c_amplitude/c_amplitude-1-50-x-0.tikz}
%      \caption{$D=1$, $N=50$}
%      \label{fig:nonlinear-c1-N-50-D-1}
%    \end{subfigure}
%
%    \begin{subfigure}{0.3\textwidth}
%      \centering
%      \includetikz{\textwidth}{Figures_v2/nonlinres/c_amplitude/c_amplitude-10-10-x-0.tikz}
%      \caption{$D=10$, $N=10$}
%      \label{fig:nonlinear-c1-N-10-D-10}
%    \end{subfigure}
%    \begin{subfigure}{0.3\textwidth}
%      \centering
%      \includetikz{\textwidth}{Figures_v2/nonlinres/c_amplitude/c_amplitude-50-10-x-0.tikz}
%      \caption{$D=50$, $N=10$}
%      \label{fig:nonlinear-c1-N-10-D-50}
%    \end{subfigure}  
%
%  \caption{[\textbf{Nonlinear Problem}] The Expected Error for changing variance of amplitude of the task $c_1$ with various training samples and various dimensional problems. For the given parameter the effect of increasing number of training samples can be seen by looking at Figures \ref{fig:nonlinear-c1-N-1-D-1}, \ref{fig:nonlinear-c1-N-10-D-1}, \ref{fig:nonlinear-c1-N-50-D-1} and the effect of increasing dimensionality can be seen by looking at Figures \ref{fig:nonlinear-c1-N-10-D-1}, \ref{fig:nonlinear-c1-N-10-D-10}, \ref{fig:nonlinear-c1-N-10-D-50}.}\label{fig:nonlinear-c1}
%\end{figure}

\end{document}
