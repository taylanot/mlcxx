\begin{figure}[!ht]
  \centering
    \begin{subfigure}{0.3\textwidth}
      \centering
      \includetikz{\textwidth}{Figures_v2/linres/std_y/std_y-1-1-x-0.tikz}
      \caption{$D=1$, $N=1$}
      \label{fig:linear-std_y-N-1-D-1}
    \end{subfigure}
    \begin{subfigure}{0.3\textwidth}
      \centering
      \includetikz{\textwidth}{Figures_v2/linres/std_y/std_y-1-10-x-0.tikz}
      \caption{$D=1$, $N=10$}
      \label{fig:linear-std_y-N-10-D-1}
    \end{subfigure}
    \begin{subfigure}{0.3\textwidth}
      \centering
      \includetikz{\textwidth}{Figures_v2/linres/std_y/std_y-1-50-x-0.tikz}
      \caption{$D=1$, $N=50$}
      \label{fig:linear-std_y-N-50-D-1}
    \end{subfigure}

    \begin{subfigure}{0.3\textwidth}
      \centering
      \includetikz{\textwidth}{Figures_v2/linres/std_y/std_y-10-10-x-0.tikz}
      \caption{$D=10$, $N=10$}
      \label{fig:linear-std_y-N-10-D-10}
    \end{subfigure}
    \begin{subfigure}{0.3\textwidth}
      \centering
      \includetikz{\textwidth}{Figures_v2/linres/std_y/std_y-50-10-x-0.tikz}
      \caption{$D=50$, $N=10$}
      \label{fig:linear-std_y-N-10-D-50}
    \end{subfigure}  

%  \caption{[\textbf{Linear Problem}] The Expected Error for varying noise standard deviation $\sigma$ and increasing training samples and different problem. For the given parameter the effect of increasing number of training samples can be seen by looking at Figures \ref{fig:linear-std_y-N-1-D-1}, \ref{fig:linear-std_y-N-10-D-1}, \ref{fig:linear-std_y-N-50-D-1} and the effect of increasing dimensionality can be seen by looking at Figures \ref{fig:linear-std_y-N-10-D-1}, \ref{fig:linear-std_y-N-10-D-10}, \ref{fig:linear-std_y-N-10-D-50}.}\label{fig:linear-std_y}
  \caption{[\textbf{Linear Problem}]: The expected error for increasing noise standard deviation $\sigma$  when changing the number of training samples for various problems of different dimensions.}
  \label{fig:linear-std_y}
\end{figure}

\begin{figure}[!h]
  \centering
    \begin{subfigure}{0.3\textwidth}
      \centering
      \includetikz{\textwidth}{Figures_v2/linres/b/b-1-1-x-0.tikz}
      \caption{$D=1$, $N=1$}
      \label{fig:linear-b-N-1-D-1}
    \end{subfigure}
    \begin{subfigure}{0.3\textwidth}
      \centering
      \includetikz{\textwidth}{Figures_v2/linres/b/b-1-10-x-0.tikz}
      \caption{$D=1$, $N=10$}
      \label{fig:linear-b-N-10-D-1}
    \end{subfigure}
    \begin{subfigure}{0.3\textwidth}
      \centering
      \includetikz{\textwidth}{Figures_v2/linres/b/b-1-50-x-0.tikz}
      \caption{$D=1$, $N=50$}
      \label{fig:linear-b-N-50-D-1}
    \end{subfigure}

    \begin{subfigure}{0.3\textwidth}
      \centering
      \includetikz{\textwidth}{Figures_v2/linres/b/b-10-10-x-0.tikz}
      \caption{$D=10$, $N=10$}
      \label{fig:linear-b-N-10-D-10}
    \end{subfigure}
    \begin{subfigure}{0.3\textwidth}
      \centering
      \includetikz{\textwidth}{Figures_v2/linres/b/b-50-10-x-0.tikz}
      \caption{$D=50$, $N=10$}
      \label{fig:linear-b-N-10-D-50}
    \end{subfigure}  

  %\caption{[\textbf{Linear Problem}] The Expected Error for changing input variance  $k$ with various training samples and various dimensional problems. For the given parameter the effect of increasing number of training samples can be seen by looking at Figures \ref{fig:linear-b-N-1-D-1}, \ref{fig:linear-b-N-10-D-1}, \ref{fig:linear-b-N-50-D-1} and the effect of increasing dimensionality can be seen by looking at Figures \ref{fig:linear-b-N-10-D-1}, \ref{fig:linear-b-N-10-D-10}, \ref{fig:linear-b-N-10-D-50}.}\label{fig:linear-b}
  \caption{[\textbf{Linear Problem}]: The expected error for increasing input variance $k$  when changing the number of training samples for various problems of different dimensions.}
  \label{fig:linear-b}
\end{figure}



\begin{figure}[!h]
  \centering
    \begin{subfigure}{0.3\textwidth}
      \centering
      \includetikz{\textwidth}{Figures_v2/nonlinres/std_y/std_y-1-1-x-0.tikz}
      \caption{$D=1$, $N=1$}
      \label{fig:nonlinear-std_y-N-1-D-1}
    \end{subfigure}
    \begin{subfigure}{0.3\textwidth}
      \centering
      \includetikz{\textwidth}{Figures_v2/nonlinres/std_y/std_y-1-10-x-0.tikz}
      \caption{$D=1$, $N=10$}
      \label{fig:nonlinear-std_y-N-10-D-1}
    \end{subfigure}
    \begin{subfigure}{0.3\textwidth}
      \centering
      \includetikz{\textwidth}{Figures_v2/nonlinres/std_y/std_y-1-50-x-0.tikz}
      \caption{$D=1$, $N=50$}
      \label{fig:nonlinear-std_y-N-50-D-1}
    \end{subfigure}

    \begin{subfigure}{0.3\textwidth}
      \centering
      \includetikz{\textwidth}{Figures_v2/nonlinres/std_y/std_y-10-10-x-0.tikz}
      \caption{$D=10$, $N=10$}
      \label{fig:nonlinear-std_y-N-10-D-10}
    \end{subfigure}
    \begin{subfigure}{0.3\textwidth}
      \centering
      \includetikz{\textwidth}{Figures_v2/nonlinres/std_y/std_y-50-10-x-0.tikz}
      \caption{$D=50$, $N=10$}
      \label{fig:nonlinear-std_y-N-10-D-50}
    \end{subfigure}  

  %\caption{[\textbf{Nonlinear Problem}] The Expected Error for changing noise standard deviation $\sigma$ with various training samples and various dimensional problems. For the given parameter the effect of increasing number of training samples can be seen by looking at Figures \ref{fig:nonlinear-std_y-N-1-D-1}, \ref{fig:nonlinear-std_y-N-10-D-1}, \ref{fig:nonlinear-std_y-N-50-D-1} and the effect of increasing dimensionality can be seen by looking at Figures \ref{fig:nonlinear-std_y-N-10-D-1}, \ref{fig:nonlinear-std_y-N-10-D-10}, \ref{fig:nonlinear-std_y-N-10-D-50}.}\label{fig:nonlinear-std_y}
  \caption{[\textbf{Nonlinear Problem}]: The expected error for increasing noise standard deviation $\sigma$  when changing the number of training samples for various problems of different dimensions.}
  \label{fig:nonlinear-std_y}
\end{figure}

%\begin{figure}[!h]
%  \centering
%    \begin{subfigure}{0.3\textwidth}
%      \centering
%      \includetikz{\textwidth}{Figures_v2/nonlinres/c_amplitude/c_amplitude-1-1-x-0.tikz}
%      \caption{$D=1$, $N=1$}
%      \label{fig:nonlinear-c1-N-1-D-1}
%    \end{subfigure}
%    \begin{subfigure}{0.3\textwidth}
%      \centering
%      \includetikz{\textwidth}{Figures_v2/nonlinres/c_amplitude/c_amplitude-1-10-x-0.tikz}
%      \caption{$D=1$, $N=10$}
%      \label{fig:nonlinear-c1-N-10-D-1}
%    \end{subfigure}
%    \begin{subfigure}{0.3\textwidth}
%      \centering
%      \includetikz{\textwidth}{Figures_v2/nonlinres/c_amplitude/c_amplitude-1-50-x-0.tikz}
%      \caption{$D=1$, $N=50$}
%      \label{fig:nonlinear-c1-N-50-D-1}
%    \end{subfigure}
%
%    \begin{subfigure}{0.3\textwidth}
%      \centering
%      \includetikz{\textwidth}{Figures_v2/nonlinres/c_amplitude/c_amplitude-10-10-x-0.tikz}
%      \caption{$D=10$, $N=10$}
%      \label{fig:nonlinear-c1-N-10-D-10}
%    \end{subfigure}
%    \begin{subfigure}{0.3\textwidth}
%      \centering
%      \includetikz{\textwidth}{Figures_v2/nonlinres/c_amplitude/c_amplitude-50-10-x-0.tikz}
%      \caption{$D=50$, $N=10$}
%      \label{fig:nonlinear-c1-N-10-D-50}
%    \end{subfigure}  
%
%  \caption{[\textbf{Nonlinear Problem}] The Expected Error for changing variance of amplitude of the task $c_1$ with various training samples and various dimensional problems. For the given parameter the effect of increasing number of training samples can be seen by looking at Figures \ref{fig:nonlinear-c1-N-1-D-1}, \ref{fig:nonlinear-c1-N-10-D-1}, \ref{fig:nonlinear-c1-N-50-D-1} and the effect of increasing dimensionality can be seen by looking at Figures \ref{fig:nonlinear-c1-N-10-D-1}, \ref{fig:nonlinear-c1-N-10-D-10}, \ref{fig:nonlinear-c1-N-10-D-50}.}\label{fig:nonlinear-c1}
%\end{figure}
