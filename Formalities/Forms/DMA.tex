\documentclass{report}
\usepackage[p,osf]{cochineal}
\usepackage[scale=.95,type1]{cabin}
\usepackage[cochineal,bigdelims,cmintegrals,vvarbb]{newtxmath}
\usepackage[zerostyle=c,scaled=.94]{newtxtt}
\usepackage[cal=boondoxo]{mathalfa}
\begin{document}
	\section*{ \centering New Machine Learning Strategies for Data Scarce Materials Science Problems \\\vspace{1cm} \centering Ozgur Taylan Turan \vspace{1cm}}
	\subsection*{Research description}
	
	Data scarcity is one of the main issues while creating surrogate constitutive manifolds for complex solid mechanics applications. This makes existing data immensely important or requires the new data collection scheme to be as efficient and smart as possible. In this context, this research aims to investigate the possible usage of existing data and ways of efficient sampling strategies. This plan falls onto the ground for transfer and active learning settings in the machine learning context. Transfer learning is planned to be probed for utilizing the knowledge gained by existing data to the targeted application to reduce the amount of data needed for data-hungry learning algorithms. Moreover, the active learning environment is planned to be investigated for efficient sampling schemes where the feature space can be actively queried to gain surrogates with desirable accuracy within a given data budget.
	
	\subsection*{Research Plan}
	
	\subsubsection*{Year-1}
	
	\begin{itemize}
		\item Improve the knowledge base regarding machine learning strategies.
		\item Creation of base for solid mechanics tools in a structured manner to enable easy data collection for future utilization.
		\item Literature review regarding machine learning applications in solid mechanics field and transfer learning.
		\item Preparation of a more specific research plan for the coming years.
	\end{itemize}
	
	
	

	
	
\end{document}