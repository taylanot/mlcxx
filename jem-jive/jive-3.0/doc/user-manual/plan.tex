%%%%%%%%%%%%%%%%%%%%%%%%%%%%%%%%%%%%%%%%%%%%%%%%%%%%%%%%%%%%%%%%%%%%%%%
% Preamble
%%%%%%%%%%%%%%%%%%%%%%%%%%%%%%%%%%%%%%%%%%%%%%%%%%%%%%%%%%%%%%%%%%%%%%%

\documentclass[a4paper,12pt]{article}

\usepackage {palatino}
\pagestyle  {plain}


\topmargin        0cm
\headheight       0cm
\headsep          0cm
\textheight      24cm
\textwidth       15cm
\evensidemargin   0cm
\oddsidemargin    0cm


%%%%%%%%%%%%%%%%%%%%%%%%%%%%%%%%%%%%%%%%%%%%%%%%%%%%%%%%%%%%%%%%%%%%%%%
% Macros
%%%%%%%%%%%%%%%%%%%%%%%%%%%%%%%%%%%%%%%%%%%%%%%%%%%%%%%%%%%%%%%%%%%%%%%


\newcommand{\BlankLine}{\vspace{1.5ex} \noindent}


%%%%%%%%%%%%%%%%%%%%%%%%%%%%%%%%%%%%%%%%%%%%%%%%%%%%%%%%%%%%%%%%%%%%%%%
% Title
%%%%%%%%%%%%%%%%%%%%%%%%%%%%%%%%%%%%%%%%%%%%%%%%%%%%%%%%%%%%%%%%%%%%%%%


\title{ Jive user manual }

\author{ }

\date{\today }



%%%%%%%%%%%%%%%%%%%%%%%%%%%%%%%%%%%%%%%%%%%%%%%%%%%%%%%%%%%%%%%%%%%%%%%
% document
%%%%%%%%%%%%%%%%%%%%%%%%%%%%%%%%%%%%%%%%%%%%%%%%%%%%%%%%%%%%%%%%%%%%%%%


\begin{document}

\maketitle


%======================================================================
% Project description
%======================================================================

\section{Project description}

\textbf{Objective}: write a manual for the Jive programming toolkit.

\BlankLine
\textbf{Audience}: the readers of the manual have a university degree
in a technical science; they have knowledge of common numerical
methods -- the finite difference method and the finite element method,
for instance -- for solving partial differential equations; and they have
knowledge of the C++ programming language.

\BlankLine
\textbf{Requirements}:
\begin{itemize}

\item The manual should be written in English.

\item The manual should help people getting started with Jive. In
  particular, it should:

  \begin{itemize}

  \item explain what one can and can not do with Jive;

  \item explain how Jive is structured (files, packages);

  \item show how to compile and link a program; 

  \item provide an overview of the main software components, or classes;

  \item explain what these classes can be used for;

  \item explain how these classes are related to each other;
    
  \item explain how these classes can be used to built numerical
    applications.
    
  \end{itemize}
  
\item The manual should primarily be read from front to back. It
  should also provide easy access to those sections that discuss a
  specific feature of Jive.
  
\item The manual should \emph{not} be used as a detailed reference
  guide.

\end{itemize}


%======================================================================
% Main question
%======================================================================

\section{Main question}

How can one use Jive to write a program that solves a particular
partial differential equation (PDE)?


%======================================================================
% Background questions
%======================================================================

\section{Background questions}

\begin{itemize}

\item \emph{What is Jive?}

  \begin{itemize}
    
  \item A C++ library for implementing programs that solve one or more
    partial differential equations.
  
  \end{itemize}


\item \emph{Should I use Jive?}

  \begin{itemize}
    
  \item You should consider using Jive if you have knowledge of C++;
    if you have knowledge of general numerical methods for solving
    PDEs; and if you want to implement a specific method for solving
    one or more PDEs.

  \item You should \emph{not} use Jive if you want to have a
    ready-to-run program.

  \end{itemize}


\item \emph{What does Jive provide?}

  \begin{itemize}
    
  \item Data structures for storing and manipulating unstructured
    grids.

  \item Data structures for selecting sets of grid components such as
    nodes, elements, boundaries, etc.

  \item Data structures for storing dense and sparse matrices.
    
  \item Algorithms for building (sparse) matrices, and for performing
    (sparse) matrix and vector operations.

  \item Algorithms for solving linear systems of equations that may be
    subjected to a set of linear constraints.
    
  \item Facilities for computing element shape functions and their
    gradients, and for evaluating integrals over the domain of an
    element.
    
  \item Facilities for reading and writing commonly used data
    structures such as grids, sets, vectors, (sparse) matrices, etc.

  \end{itemize}


\item \emph{What other libraries are required when using Jive?}

  \begin{itemize}

  \item The Jem library. To be precise, the core Jem library together
    with the packages \texttt{numeric} and \texttt{xml}.

  \end{itemize}


\item \emph{How can I start using Jive?}

  \begin{itemize}

  \item Read this manual.
    
  \item Read the manual of Jem, as Jive is built on top of the Jem C++
    library.

  \item Study the example programs provided with Jive.

  \item Read the HTML reference manuals of Jive and Jem.

  \end{itemize}

\end{itemize}


%======================================================================
% Key questions
%======================================================================

\section{Key questions}

\subsection*{Part I: getting started}

\begin{itemize}

\item \emph{How do I start developing a program with Jive?}

  \begin{itemize}
  
  \item \emph{What is Jive from a programmer's point of view?}
    
  \item \emph{How is Jive organized?}
    
  \item \emph{How do I use components from Jive in my program?}
    
  \item \emph{Do I also need to use components from Jem?}
    
  \item \emph{If so, how do I use components from Jem?}
    
  \item \emph{How do I compile and link my programs?}

  \end{itemize}


\item \emph{How do I solve a PDE with Jive?}

  \begin{itemize}

  \item \emph{What PDE is considered?}

  \item \emph{Which numerical method will be used?}

  \item \emph{What are the main steps in the solution procedure?}

  \item \emph{Which components from Jive and Jem do I need?}

  \item \emph{How do I create a mesh/grid?}

  \item \emph{How do I compute the element matrices and vectors?}

  \item \emph{How do I assemble the global system of equations?}

  \item \emph{How do I apply constraints?}

  \item \emph{How do I solve the global system of equations?}

  \end{itemize}

\end{itemize}


\subsection*{Part II: description of Jive components}

\begin{itemize}

\item \emph{What general utilities are provided by Jive?}

\item \emph{How do I store and manipulate matrices and vectors?}
    
\item \emph{How can I evaluate the geometrical properties of lines,
    triangles, and other basic shapes?}

\item \emph{Which components are available for finite element
    computations?}

\item \emph{How can I solve a linear system of equations?}

\item \emph{How can I read and write data from and to files?}

\end{itemize}


\subsection*{Part III: solution techniques}


\end{document}
