\documentclass[a4paper]{article}

\usepackage{listings}
\usepackage{palatino}
\usepackage{parskip}
\usepackage{color}
\usepackage{url}

\lstset{language={},backgroundcolor=\color[gray]{0.95},
  basicstyle=\ttfamily}


%========================================================================
% Definitions
%========================================================================


\newcommand{\Code}[1]{\texttt{#1}}


%========================================================================
% Document
%========================================================================


\title {Jem/Jive 2.0\\[1ex]
        Release Notes}
\author{Dynaflow Research Group}
\date  {24-10-2011}


\begin{document}

\maketitle

\section{Overview}

Version 2.0 of Jem/Jive is a new major release with many new features and
improvements. Although this new version is largely source compatible with
the previous version (1.5.1), it might be necessary to adapt existing
programs before they can be compiled with version 2.0.

The following list provides an overview of the new features and
improvements. More detailed information can be found in the following
sections.

\begin{itemize}

  \item Jive now provides native support for the Windows operating system
    in addition to Linux and MacOS~X.

  \item The reference manual has been set up in a completely new way. It
    now lists all classes and functions, and not only those with
    documentation.

  \item A new type \Code{idx\_t} is used to represent indices and sizes
    of most data structures. By default this type is equal to \Code{int}
    but it can be configured to be a long integer.

  \item The \Code{Array} class provides support for mixed-type
    expressions and irregular slices using the subscript and
    function-call operators.

  \item The implementation of the visualization module has been largely
    revised. It provides a better user interface and implements a much
    better font-rendering method.

  \item The \Code{geom} and \Code{fem} packages in Jive provide support
    for so-called isogeometric elements that are based on NURBS and
    T-splines.

  \item The \Code{solver} package in Jive implements new and more
    efficient (parallel) preconditioners and solvers. It also supports a
    new option to re-use a preconditioner multiple times during a
    non-linear and/or transient computation.

  \item The \Code{implict} package in Jive provides a new and more
    flexible framework for implementing implicit time integration
    procedures.

  \item The syntax of a properties file has been extended so that the
    type of an object (such as a model) can be specified more directly. A
    sub-properties set named \Code{params} can be used to simplify
    the specification of user-defined functions.

  \item The \Code{Properties} class is much more efficient when storing
    simple values such as integers and floating point numbers.

  \item The \Code{UserFunc} class provides support for computing the
    derivative of a user-defined function.

  \item The output operators have been implemented in a better way. The
    type of the output stream is preserved when chaining multiple output
    operations.

  \item Various existing classes have been improved and several new
    classes have been added to Jem and Jive. These minor improvements and
    extensions make it bit easier to develop programs with Jem/Jive.

\end{itemize}

%------------------------------------------------------------------------

\section{Windows support}

Jive 2.0 can be used to develop native Windows programs; there is no need
anymore for a compatibility layer like Cygwin. Two compilers are
supported: the Windows version of the GNU compiler (MinGW) or the Visual
C++ compiler from Microsoft, possibly in combination with Visual
Studio. When using the GNU compiler or when using the Microsoft compiler
without Visual Studio, it is necessary to install the so-called MSYS
package that provides GNU Make and some other tools that are necessary to
make use of the Makefiles provided by Jive.

%------------------------------------------------------------------------

\section{Reference manual}

The reference manual has been restructured so that it provides a complete
overview of all classes and functions that are available in Jem and Jive.
A new search function can help finding the documentation of a specific
class or function. Although the manual is far from complete, it provides
an efficient way to find the definition of a class or the declaration of
a (member) function; it is not necessary any more to look into header
files. The manual also makes it clear how classes are related to each
other. The latest version of the manual is available online at:
\url{www.dynaflow.com/manuals/jive-reference-manual/}.

%------------------------------------------------------------------------

\section{The \Code{idx\_t} type}

A new type \Code{idx\_t} (declared in the \Code{jem} namespace) is used
to store indices and sizes of most container classes. By default, the
\Code{idx\_t} type is equal to \Code{int} so that existing code is not
affected by this change. The \Code{idx\_t} type can be configured to be
equal to long integer on 64-bit platforms. This makes it possible to
store very large data sets of which the size exceeds the capacity of
32-bit integers.

The types \Code{IdxVector} and \Code{IdxMatrix} (declared in the
namespace \Code{jive}) are used to store arrays of indices. By default,
they are equal to the types \Code{IntVector} and \Code{IntMatrix}.

Current users of Jive are advised to use the new types \Code{idx\_t},
\Code{IdxVector} and \Code{IdxMatrix} in new code. Existing code need not
to be updated, unless 32-bit integers are not large enough.

%------------------------------------------------------------------------

\section{Isogeometric elements}

Jive 2.0 provides support for so-called isogeometric elements that are
based on Bezier extraction. This feature makes it possible to use element
shape functions that are based on splines, including NURBS and T-splines.
The new classes \Code{BezierShape} and \Code{BezierElement} are the main
building blocks for implementing models that are based in isogeometric
elements. These are defines in the packages \Code{geom} and \Code{fem},
respectively.

Because an isogeometric element has no nodes from a traditional finite
element point of view (it has control points instead), many modifications
have been made to the \Code{geom} package. Although most of these
modifications are not visible, some may be incompatible with existing
code. To be precise, the following member functions have been renamed:
\begin{itemize}

  \item \Code{getLocalNodeCoords} has been renamed to
    \Code{getVertexCoords};

  \item \Code{getNodeGradients} has been renamed to
    \Code{getVertexGradients};

  \item \Code{getNodeNormals} has been renamed to
    \Code{getVertexNormals}.

\end{itemize}
In addition, the \Code{params} parameter has been removed from the
following member functions:
\begin{itemize}
  \item \Code{evalShapeFunctions};
  \item \Code{evalShapeGradients};
  \item \Code{evalNormal}.
\end{itemize}

%------------------------------------------------------------------------

\section{New preconditioners}

The \Code{solver} package has been extended with a number of new
preconditioners: \Code{ILUn}, \Code{ILUd} and \Code{Coarse}. The first
two implement efficient and robust incomplete factorization algorithms.
The \Code{ILUn} preconditioner is less sophisticated than \Code{ILUd} but
will work quite well for a wide range of problems. The \Code{ILUd} can be
much more effective, but not for all kinds of problems. The
\Code{Coarse} preconditioner implements an algebraic coarse grid
preconditioner that must be combined with a conventional preconditioner
like \Code{ILUd}. The \Code{Coarse} preconditioner is especially meant to
be used in parallel computations.

The new \Code{Schur} solver is also meant to be used in parallel
computations. It transforms the global system of equations into a smaller
one that involves only the degrees of freedom in the interfaces of the
sub-domains. This smaller system can be solved with any of the available
iterative solvers. The \Code{Schur} solver can be very effective in
combination with the \Code{Coarse} preconditioner, especially for systems
of equations that are obtained from solid mechanics models.

The iterative solvers have a new property named \Code{updatePolicy} that
controls when a preconditioner is to be updated. Its default value is
\Code{"Auto"} which indicates that a preconditioner should only be
updated when the performance of the solver degrades too much. This
procedure can significantly lower the time that is needed to update the
preconditioner in non-linear and time-dependent computations.

%------------------------------------------------------------------------

\section{New time integration framework}

The \Code{implict} package provides a new and flexible time integration
framework. This framework is based on the assumption that the time
integration procedure is implemented by a model and not by a module. That
is, a model must combine the stiffness matrix, the damping matrix and the
mass matrix into one matrix that is then passed to a linear or non-linear
solver module. In this way it is straightforward to combine any time
integration scheme with any linear or non-linear solution procedure.

Three classes form the basis of the time integration framework:
\Code{TimeStepper}, \Code{TransientMatrix} and \Code{TransientModel}. The
\Code{TimeStepper} class is an abstract base class that represents a time
integration method. The \Code{TransientMatrix} class represents a matrix
that is a linear combination of the stiffness matrix, the damping matrix
and the mass matrix. It uses the \Code{TimeStepper} class to determine
the coefficients with which the three matrices are to be multiplied. The
\Code{TransientModel} class couples the \Code{TimeStepper} and the
\Code{TransientMatrix} classes and takes care of handling the actions
that are requested by a solver module. A \Code{TransienModel} is
typically the top-most model in the model tree when running an implicit
time-dependent simulation.

Implementing a new time integration procedure involves implementing a new
class that is derived from the \Code{TimeStepper} class. There is no need
to implement a new model, unless the time integration procedure has a
form that does not fit into the interface defined by the
\Code{TimeStepper} class.

%------------------------------------------------------------------------

\section{Extended properties file syntax}

The syntax of a properties file has been slightly extended to simplify
the specification of a model, module or some other object. The type of
the object does not have to be specified as a separate property. Instead,
it can be specified between the `\Code{=}' and `\Code{\{}' tokens. For
instance:
\begin{lstlisting}
  model = "Transient"
  {
    // ...
  };
\end{lstlisting}
This is equivalent with:
\begin{lstlisting}
  model =
  {
    type = "Transient";
    // ...
  };
\end{lstlisting}

Properties that are specified in a properties set named \Code{params} can
be used in function expressions elsewhere in the properties file, without
having to specify the \Code{params} name as a prefix. This is best
explained with a simple example:
\begin{lstlisting}
  params =
  {
    tau = 0.1;
  };

  model = "Example"
  {
    load = "exp( - t / tau )";
  };
\end{lstlisting}
The property \Code{model.load} is a function expression involving the
parameter \Code{tau} that has been defined in the \Code{params}
properties set.

%------------------------------------------------------------------------

\section{Other new features}

The output operators (overloaded left-shift operators) have been
succeeded by overloaded \Code{print} functions that result in more
readable code. For instance:
\begin{lstlisting}
  print ( System::out(), "Hello, the value of x = ", x );
\end{lstlisting}
The output operators still exist but have been re-implemented in terms of
the \Code{print} functions.

A new \Code{StringBuffer} class (in the \Code{base} package of Jem) can
be used to efficiently construct strings. Here is an example:
\begin{lstlisting}
  StringBuffer  buf;
  String        str;

  print ( buf, "x = ", x, ', y = ', y );

  str = buf.toString ();
\end{lstlisting}
The \Code{StringBuffer} class can pass its allocated data directly to a
\Code{String} object so no data needs to be copied.

The \Code{Array} class now supports expressions involving different types
of arrays. For instance, it is possible to write an expression involving
integer and floating point arrays. It is also possible to assign an array
to another array of a different types, as long as the array elements can
be converted. Here is a simple example:
\begin{lstlisting}
  Array<int>     a ( 10 );
  Array<double>  b ( 10 );
  Array<float>   c ( 10 );

  c = a + 3 * b;
\end{lstlisting}
An integer array can be used as an index to obtain a non-regular slice;
there is no need to use the \Code{select} function. For instance:
\begin{lstlisting}
  Array<int>       a ( 4 );
  Array<double,2>  b ( 10, 10 );

  // Initialize a ...

  b(a,3) = 1.0;
\end{lstlisting}
Note that the \Code{select} function is still needed when mixing integer
arrays and slices.

A new function \Code{find} can be used to obtain the indices of the array
elements for which a given expression is true. For instance:
\begin{lstlisting}
  Array<double>  a ( 10 );
  Array<int>     b;

  find ( b, a > 10.0 );

  a[b] = 10.0;
\end{lstlisting}

The \Code{Table} class has been extended with functions for getting
values from a row or column. The \Code{XTable} class has been extended
with functions for setting the values in a row or column.

The \Code{DofSpace} class has been extended with overloaded versions of
the \Code{getDofIndices} function that return the DOFs for a single DOF
type or a single row item. There is no need any more to use an array with
a single element.

Function expressions can refer to a special variable named \Code{rand}
that evaluates to a random uniform number between 0.0 and 1.0. This
variable is stored in the global database \Code{globdat} with the name
\Code{var.rand}.

The \Code{Globdat} class has been extended with a number of functions for
updating and reverting the time and time step that are stored in the
global database \Code{globdat}.

Two new macros, named \Code{JEM\_ASSERT2} and \Code{JEM\_PRECHECK2}, can
be used to check that a given expression is true. These are similar to
the existing macros \Code{JEM\_ASSERT} and \Code{JEM\_PRECHECK}, except
that they have two parameters: an expression and a string that will be
printed when the expression is not true. The latter string can clarify
what kind of error occurred.

%------------------------------------------------------------------------

\section{Incompatibilities with version 1.5.1}

\begin{itemize}

  \item The \Code{isNull} member function has been deleted from the
    following classes: \Code{NodeSet} \Code{ElementSet},
    \Code{NodeGroup} and \Code{ElementGroup}. An instance of one of these
    classes can be compared with the constant \Code{NIL} instead.

  \item The \Code{ModuleFactory} class requires a module construction
    function with an extra \Code{Properties} parameter. This parameter
    will be set to the global database \Code{globdat}.

  \item The \Code{DOF\_SPACE} member has been deleted from the class
    \Code{ActionParams} because it was never really used.

  \item The class \Code{SolverRunData} in the \Code{implict} package has
    been modified. This should not be a problem because this class is not
    meant to be used outside Jive.

  \item The function \Code{newSolver} in the \Code{implict} package has a
    different set of parameters. To be precise, it requires a
    \Code{Properties} object containing the system matrix, the
    \Code{DofSpace} and a few other optional objects. This
    \Code{Properties} can be created with the function
    \Code{newSolverParams}.

  \item The \Code{PrintWriter} class has no member function named
    \Code{print} anymore. The global function \Code{print} should be used
    instead.

  \item The function \Code{declareParallelSolvers} in the \Code{solver}
    package no longer exists. Use the function \Code{declareSolvers}
    instead.

\end{itemize}

\end{document}

