\documentclass[a4paper]{article}

\usepackage[text={14cm,24cm},centering]{geometry}
\usepackage{listings}
\usepackage{palatino}
\usepackage{parskip}
\usepackage{color}
\usepackage{url}

\lstset{language={},backgroundcolor=\color[gray]{0.95},
  basicstyle=\ttfamily}


%========================================================================
% Definitions
%========================================================================


\newcommand{\Code}[1]{\texttt{#1}}


%========================================================================
% Document
%========================================================================


\title {Jem/Jive 3.0\\[1ex]
        Release Notes}
\author{Dynaflow Research Group}
\date  {20-12-2019}


\begin{document}

\maketitle

\section{Overview}

Version 3.0 of Jem/Jive is a major release with some new features, bug
fixes, lots of improvements, and, last but not least, an update to the
C++11 standard. This version should be largely source compatible with
version 2.2, although you might need to make some small changes to your
Jem/Jive programs.

The following list provides an overview of the new features and
improvements. More detailed information can be found in the following
sections.

\begin{itemize}

  \item Extension of the \Code{NonlinModule} with support for solving
    non-linear minimisation problems with box constraints. These kind of
    problems occur, for instance, when using a phase field to describe
    the propagation of a crack in a solid material.

  \item Extension of the format of properties files; you can now use the
    \Code{+=} operator to extend a previously defined property.

  \item Improvement of the \Code{Ref} class so that it behaves more like
    a regular pointer. You no longer need to use the \Code{NIL} constant,
    but can use \Code{nullptr} instead.

  \item Support for move semantics where that makes sense.

  \item Support for initializer lists for arrays and other sequence
    types.

  \item Support for the GLFW OpenGL toolkit in addition to the GLUT
    toolkit.

  \item Complete overhaul of the implementation of the \Code{gl} package.
    It can now make use of modern shaders to achieve better and faster
    graphics.

  \item Improved parallel performance of the linear solvers and
    preconditioners, and that of the coarse grid preconditioner in
    particular. Improved handling of simulations in which degrees of
    freedom are added/removed dynamically only in parts of a model.

  \item New class \Code{jem::Flags} for handling bitwise flags in a
    better way.

  \item Change of the \Code{jem::idx\_t} type from a typedef into a class
    type. This should be mostly transparent to Jem/Jive applications if
    you used the \Code{idx\_t} type properly. You may have to change a
    few \Code{int} variable declarations to \Code{idx\_t} declarations.

  \item New header files with forward declarations. All Jem/Jive packages
    now provide a header file \Code{<forward.h>} that contains forward
    declarations for the classes provided by that package.

  \item New Jem package \Code{crypto} for performing
    encryption/decryption operations. This requires the OpenSSL
    library.

  \item New Jem package \Code{net} for performing network operations,
    such as sending and receiving data through sockets.

  \item New Jem package \Code{ssl} for performing network operations over
    secure sockets. This requires the OpenSSL library.

\end{itemize}

%------------------------------------------------------------------------

\section{Update to the C++11 standard}

Starting with version 3.0, a C++ compiler with support for the C++11
standard is required to compile and use Jem/Jive. Jem/Jive now support
move semantics (where that makes sense), initializer lists, the
\Code{nullptr\_t} type, literal operators (for the \Code{idx\_t} and
\Code{Time} classes, among others), and the \Code{noexcept} and
\Code{override} function specifiers, among others.

A consequence of this change is that the \Code{Ref} class now more
closely resembles a standard pointer. Here are some examples:

\begin{lstlisting}
  Ref<Object>  obj = nullptr;  // No need to use NIL.

  if ( obj == nullptr ) ...

  if ( obj ) ...

  if ( ! obj ) ...
\end{lstlisting}

This example shows how to use the literal operators provided by Jem:

\begin{lstlisting}
  using namespace jem::literals;

  Time   t = 24_hour;
  idx_t  i = 34_idx;

  t = t + 60_min;
  i = i + 128_idx;
\end{lstlisting}

The following example shows how you can use initializer lists:

\begin{lstlisting}
  Tuple<String,3>  s = { "one", "two", "three" };
  Array<int>       a = { 1, 2, 3, 4, 5 };
\end{lstlisting}

The next example demonstrates how you can make use of move semantics:

\begin{lstlisting}
  Flex<MyClass>  list;  // Assume MyClass supports move semantics.

  while ( condition )
  {
    MyClass  tmp;

    // Initialize tmp ...

    list.pushBack ( std::move( tmp ) );
  }
\end{lstlisting}

%------------------------------------------------------------------------

\section{Extension of the \Code{Properties} file format}

The \Code{Properties} file format now supports the use of the \Code{+=}
operator to extend previously defines properties This is best explained
through an example:

\begin{lstlisting}
  settings =
  {
    list = [ "one", "two", "three" ];
    word = "Hello";
  };

  settings +=
  {
    more  = 10;          // Add a new member to "settings".
    list += [ "four" ];  // Extend the "list" array.
    word += " World!";   // Extend the "word" string.
  };
\end{lstlisting}

The \Code{+=} operator is particularly useful if you want to modify an
existing properties file; you only need to add a section that modifies
previously defined properties.

\end{document}

