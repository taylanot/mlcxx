
\chapter{Essential Jem components}
\label{chapter:essential-jem-components}

As Jive uses components from the Jem library, some knowledge of Jem is
required before you can start using Jive. This chapter therefore provides
a summary of the most frequently used classes and functions from Jem.
After reading this chapter you should be able to understand the example
programs presented in the next two chapters, and to make modifications to
these programs. Before you start implementing your own programs, however,
you are advised to consult the Jem user manual and reference manual. You
can skip this chapter if you are already familiar with Jem.

The classes and functions discussed in this chapter are divided into four
categories: core classes that form the foundation of Jem; utility classes
that implement various common data structures and algorithms; I/O classes
and functions that provide facilities for reading and writing data; and
numerical classes and functions that can be used to implement numerical
algorithms. Each category is dealt with in one of the following four
sections.

%========================================================================

\section{Core components}

The core classes and functions are located in the package \Code{base}.
They implement various basic data structures such as strings and arrays;
they provide support for memory management, multi-threading and
scripting; and they enable you to access various system-level services
in a portable way. The core classes that are used frequently in Jive are
summarized in \autoref{table:jem-core}. These classes are described in
more detail below.

\begin{table}

  \caption{Summary of the essential core classes.}
  \label{table:jem-core}

  \begin{center}

    \begin{tabular}{|>{\ttfamily}lp{10cm}|}
      \hline
      Name       & Description \\
      \hline \hline
      String      & Represents a character string. \\
      Slice       & Selects parts of array-like objects. \\
      System      & Provides access to the standard I/O streams
                    and other system facilities. \\
      Array       & Implements a multi-dimensional array that
                    supports slicing and array expressions. \\
      Collectable & Provides support for automatic garbage
                    collection. \\
      Ref         & Implements a `smart' pointer that points to a
                    collectable object. \\
      Object      & Defines a common base class for many
                    classes in Jem. \\
      Throwable   & Defines a base class for all exceptions thrown
                    by Jem and Jive. \\
      \hline
    \end{tabular}

  \end{center}

\end{table}


%------------------------------------------------------------------------

\subsection*{The \Code{String} class}

The \Code{String} class represents character arrays. It provides a set of
functions for accessing individual characters, for comparing strings, for
finding sub-strings, for transforming strings, and for creating string
slices (more about those in the next section). The \Code{String} class
also overloads the \Code{+} operator so that one can easily concatenate
two strings.

Strings are \emph{immutable}. This means that the individual characters
in a string can not be modified. The only way to change the contents of a
string is to call one of the assignment operators. You will have to use
an \Code{Array}, or a similar type of object, if you need to modify the
individual characters in a string. The \Code{StringBuffer} class can be
used to create strings with dynamic contents.

Note that the character array stored in a \Code{String} object is
\emph{not} null-terminated. Use the \Code{CString} class and the
\Code{makeCString} function to convert a \Code{String} object to a
null-terminated character array.

Below is a small example involving the \Code{String} class. Note that
some details, such as include directives and namespace declarations, have
been omitted to reduce the amount of code.

\IncludeSource{C++}{examples/String.cpp}

%------------------------------------------------------------------------

\subsection*{The \Code{Slice} class}

\Code{Slice} objects are used to select parts of arrays, strings, and
other array-like objects. Such a selection, or \emph{slice}, is defined
by three integers: a begin index, an end index and a stride. The begin
index is a non-negative integer that points to the first element in the
slice. The end index is an integer that points one position past the last
element in the slice. It is always larger than or equal to the begin
index. The stride is a positive integer that specifies the distance
between two consecutive elements in the slice. For instance, if the
stride is three, a slice contains only each third element of an
array-like object. If the stride equals one, the slice is said to be
contiguous.

Slices are created by invoking the so-called \emph{slice operator} of an
array-like object. This operator is a member function -- usually an
overloaded subscript or function call operator -- that accepts a
\Code{Slice} object and returns an object representing part of the
array-like object. The \Code{String} class, for instance, provides an
overloaded version of the subscript operator that returns a sub-string
for a given \Code{Slice} object.

\Code{Slice} objects are normally constructed by calling the non-member
function function \Code{slice()}, that is overloaded for several types of
slices (contiguous and non-contiguous slices, for instance). These
functions may be passed instances of the special classes
\Code{Begin}, \Code{End} and \Code{All} to create slices that start at
the beginning of an array, extend to the end of an array, or include all
elements of an array. For convenience, there are three pre-defined
instances of these classes which are named \Code{BEGIN}, \Code{END} and
\Code{ALL}, respectively. Note that the class \Code{All} is derived from
the \Code{Slice} class so an instance of this class can be passed
directly to a slice operator.

The following code fragment demonstrates how the \Code{Slice} class and
the \Code{slice()} function can be used to select slices from a
string.

\IncludeSource{C++}{examples/Slice.cpp}

%------------------------------------------------------------------------

\subsection*{The \Code{System} class}

The \Code{System} class provides access to various system-level
facilities. It can be used to get and set the standard input and output
streams; get information about the system; get the contents of
environment variables; and to catch and print exceptions thrown by
the main function of a program. Since it has only static members, it can
be viewed as a sort of mini namespace.

The \Code{System} class provides one input stream and four output
streams: one for writing error messages, one for writing warnings, one
for writing normal output, and one for writing log messages. The input
and output streams are initialized when they are accessed for the first
time.

The example code below prints a message to the standard output stream and
to the debug output stream (which is routed through the log output
stream). It then throws an exception that will be caught by the member
function \Code{System::exec}.

\IncludeSource{C++}{examples/System.cpp}

%------------------------------------------------------------------------

\subsection*{The \Code{Array} class}

The \Code{Array} class implements a multi-dimensional array of objects.
It is a class template with two template parameters. The first parameter,
say \Code{T}, specifies the type of the objects stored in the array. This
type should define at least a default constructor and a copy
constructor. The second parameter, say \Code{N}, specifies the
\emph{rank} -- i.e. the number of dimensions -- of the array. If the rank
parameter is omitted, a default value of one is assumed.

Each array has a \emph{shape}: an integer tuple (vector) of length equal
to the rank of the array. The elements of this tuple specify the sizes of
the array in each dimension. For instance, a three-dimensional array with
shape $(10,3,5)$ has size~10 in the first dimension, size~3 in the second
dimension and size~5 in the third dimension.

The elements in an array are identified in the usual way by a set of
integer indices ranging from zero to the corresponding array size minus
one. Thus, the index pair $(1,0)$ refers to the second element in the
first column of a two-dimensional array. The elements can be accessed by
means of the overloaded subscript and function-call operators.

An array stores its elements in a data block that may be shared with one
or more other arrays. As a consequence, multiple arrays may provide
different views to parts of the same data. For instance, a
one-dimensional array may provide access to a particular column of a
two-dimensional array. Another consequence of the shared storage scheme
is that the elements of one array may be modified through another
array.

The \Code{Array} class defines a set of slice operators -- overloaded
subscript and function-call operators -- that can be used to select
regular, rectangular sections of an array. These operators return an
\Code{Array} object that points to a sub-set of the data block pointed to
by the \Code{Array} object on which the slice operators are invoked.
Because the returned array shares its data with the original array, all
modifications to the returned array will be visible in the original
one, and the other way around.

Jem provides a set of overloaded operators and functions -- that are not
members of the \Code{Array} class -- for writing expressions that operate
on all elements of one or more arrays. These operators and functions
return special array objects, called \emph{array expressions}, that can
be assigned to regular \Code{Array} objects. Array expressions can also
be passed to the overloaded operators and functions to create new array
expressions. The use of array expressions is often more efficient than
writing a series of equivalent (nested) loops. One reason is that the
implementation of array expressions is based on `expression templates', a
technique that makes it possible to avoid creating temporary array
objects (see the papers available at \url{http://oonumerics.org} for the
details). Another reason is that array expressions collapse multiple
nested loops into a single loop in the common case that all the operands
are stored contiguously in memory.

The code fragment below demonstrates how to create an array, how to
access the elements of an array, how to create array slices and how to
make use of array expressions.

\IncludeSource{C++}{examples/Array.cpp}

%------------------------------------------------------------------------

\subsection*{The \Code{Collectable} and \Code{Ref} classes}

The \Code{Collectable} class provides support for automatic garbage
collection. That is, objects derived from the \Code{Collectable}, called
\emph{collectable objects}, are created on the heap and are automatically
deleted when they are no longer used. It is not possible to delete
collectable objects explicitly through a \Code{delete} expression.

Collectable objects are normally accessed through instances of the class
template \Code{Ref} that mimics the behavior of a pointer. As long as a
collectable object can be reached through a \Code{Ref} instance, or a
sequence of \Code{Ref} instances, that object will not be deleted.
Although one can also access collectable objects through normal pointers,
it is unspecified whether the existence of a route from a live pointer
(i.e. a pointer that can be reached from the program) to a
collectable object prevents the deletion of that object.

Collectable objects must be created with the function template
\Code{newInstance} that mimics a new expression but returns a \Code{Ref}
object instead of a normal pointer. The \Code{newInstance} function can
be supplied a variable number of arguments that will be passed on to the
matching constructor of the collectable class. A collectable object must
not be created on the stack. To prevent this from accidentally happening,
a collectable class should declare its destructor as a protected or
private member.

The example program below shows how to use the \Code{Collectable} and
\Code{Ref} classes. It defines and instantiates a trivial collectable
class, named \Code{Tracer}, that prints a message in its constructor and
destructor.

\IncludeSource{C++}{examples/Collectable.cpp}

This is the output of the program:

\begin{Verbatim}
  Tracer::Tracer  ( 1 )
  Tracer::Tracer  ( 2 )
  Tracer::~Tracer ( 1 )
  Tracer::~Tracer ( 2 )
\end{Verbatim}

When the \Code{Ref} object \Code{t2} is assigned to \Code{t1}, the first
\Code{Tracer} object is destroyed because no \Code{Ref} is pointing to it
any more; both \Code{t1} and \Code{t2} point to the second \Code{Tracer}
object. This object is destroyed when the function \Code{run} exits so
that both \Code{t1} and \Code{t2} are destroyed.

%------------------------------------------------------------------------

\subsection*{The \Code{Object} class}

Many classes in Jem are directly or indirectly derived from the class
\Code{Object}. It can therefore be used to implement heterogeneous
data structures and generic algorithms. For instance, a list
containing \Code{Object} instances can be used to store instances of
all classes derived from the \Code{Object} class. The \Code{Object}
class is derived from \Code{Collectable}, so that all \Code{Object}
instances are subjected to automatic garbage collection.

Jem provides two non-member functions that can be used to convert
non-\Code{Object} instances to \Code{Object} instances, and the other way
around. These two functions are named \Code{toObject()} and
\Code{toValue()}, respectively. This is illustrated in the following
example.

\IncludeSource{C++}{examples/Object.cpp}

%------------------------------------------------------------------------

\subsection*{The \Code{Throwable} class}

The \Code{Throwable} class defines a single base class for all exceptons
that are thrown by Jem and Jive. By convention, classes derived from the
\Code{Throwable} class provide a constructor with two string parameters:
one that describes the context in which the exception occured, and one
that describes the cause of the exception. These strings can be obtained
by calling the member functions \Code{where()} and \Code{what()},
respectively.

There are two families of exception classes: one derived from the
\Code{Exception} class and one derived from the \Code{RuntimeException}
class. The first family represents error conditions caused by external
events, such as wrong input, that a program should handle. The second
family represents error conditions caused by programming errors that
should not occur in an error-free program. Here is a brief example
demonstrating how to handle exceptions.

\IncludeSource{C++}{examples/Exception.cpp}

%========================================================================

\section{Utility components}

The package \Code{util} in Jem provides classes that implement various
types of containers such as lists, sets, and hash tables; that implement
an event framework with which you can connect independent software
components; that enable you to read, write and store configuration data
in a flexible way; and that provide various handy facilities.
\autoref{table:jem-util} lists the utility classes that are used
frequently in Jive. These classes are described in more detail in the
next few sub-sections.

\begin{table}

  \caption{Summary of the essential utility classes.}
  \label{table:jem-util}

  \begin{center}

    \begin{tabular}{|>{\ttfamily}lp{10cm}|}
      \hline
      Name        & Description \\
      \hline \hline
      ArrayBuffer & Implements an expandable array. \\
      SparseArray & Implements a multi-dimensional sparse array. \\
      Event       & Encapsulates a list of call-back functions
                    that are invoked when a specific event occurs. \\
      Properties  & Implements a hierarchical database containing
                    name/value pairs. \\
      \hline
    \end{tabular}

  \end{center}

\end{table}

%------------------------------------------------------------------------

\subsection*{The \Code{ArrayBuffer} class}

The class template \Code{ArrayBuffer} represents a one-dimensional array
that can grow and shrink. It is meant to be a helper class for creating
array objects of type \Code{Array<T,1>} of which the size is not known
beforehand. Here is an example:

\IncludeSource{C++}{examples/ArrayBuffer.cpp}

An \Code{ArrayBuffer} encapsulates a one-dimensional \Code{Array} object
that stores the elements in the \Code{ArrayBuffer}. The conversion of an
\Code{ArrayBuffer} to an \Code{Array} is therefore an efficient
operation.

%------------------------------------------------------------------------

\subsection*{The \Code{SparseArray} class}

The class template \Code{SparseArray} represents a multi-dimensional
sparse array. That is, an array that contains mostly zeroes, where the
definition of the term ``zero'' depends on the type of the array
elements. Although the interface of the \Code{SparseArray} class is
somewhat similar to that of the \Code{Array} class, it is intended to be
used as a storage container and not as a direct building block for
numerical algorithms. The \Code{SparseArray} class therefore does not
implement slice operators and does not provide support for sparse array
expressions.

Here is an example involving the \Code{SparseArray} class.

\IncludeSource{C++}{examples/SparseArray.cpp}

The output of the program will be like this:

\begin{Verbatim}
  1 0 0 0
  0 0 0 0
  0 0 0 2
  3 0 0 0
\end{Verbatim}

%------------------------------------------------------------------------

\subsection*{The \Code{Event} class}

The class template \Code{Event} encapsulates a list of call-back
functions, named \emph{delegates}, that are invoked when the
\Code{emit()} member function is called. Events can be used, among
others, to connect independent software components and to implement the
model-view-controller pattern. The template parameters of the
\Code{Event} class determine the parameter types of the \Code{emit()}
member function. These parameters are said to be the parameters of an
event. The number of parameters can be varied from zero to three, as
illustrated below:

\IncludeSource{C++}{examples/Event-1.cpp}

To be useful, an event should be connected to one or more delegates
by calling the non-member function \Code{connect()}. A delegate can be a
regular function or a non-static member function -- called a method
-- of a collectable class (that is, a class derived from the
\Code{Collectable} class). The parameter types of the delegate must be
compatible with the parameter types of the event. Thus, one can not
connect an event with a parameter of type \Code{int} to a delegate with
a parameter of type \Code{int*}. One can, however, connect an event to a
delegate that has less parameters than the event. In this case, the
last few parameters of the event will be discarded when the delegate
is called.

The example below illustrates the power of events. It involves two
independent classes that are connected to each other through an event.
The first class, named \Code{Graph} represents a graph that dynamically
displays a value as function of time. The second class, named
\Code{Sensor}, represents a device that measures some quantity at a given
location. Each time it has taken a new measurement, it calls the
\Code{emit()} member of its event named \Code{newValueEvent}. The
measurement is passed as an argument to the \Code{emit()} function that
passes it on to the member function \Code{addPoint()} of the \Code{Graph}
object that is connected to the event.

\IncludeSource{C++}{examples/Event-2.cpp}

%------------------------------------------------------------------------

\subsection*{The \Code{Properties} class}

The \Code{Properties} class implements an hierarchical database
containing a set of name/value pairs called \emph{properties}. It is
typically used as a repository for configuration data, but it can also be
used as a flexible in-memory data base.

The name of a property is represented by a \Code{String} instance. The
value of a property is a variable that has one of the following types:
\Code{bool}, \Code{int}, \Code{double}, \Code{String},
\Code{Array<bool>}, \Code{Array<int>}, \Code{Array<double>},
\Code{Array<String>}, or \Code{Object}. The value of a property can also
be a \Code{Properties} object so that you can build hierarchical sets of
properties. A \Code{Properties} object that is contained by another
\Code{Properties} object is called a \emph{nested} \Code{Properties}
object.

A \Code{Properties} object can be filled by calling the member functions
\Code{parseFile()} and \Code{parseFrom()}. The former reads a set of
properties from a file, while the latter reads a set of properties from
any text input stream. The contents of a \Code{Properties} object can be
printed by calling the member function \Code{printTo()} or by calling the
\Code{print} function.

Individual properties can be inserted into a \Code{Properties} object by
calling one of the overloaded versions of the \Code{set()} member
function. They can be extracted from a \Code{Properties} object by
calling one of the overloaded versions of the \Code{find()} or
\Code{get()} member functions. The difference between these two functions
is that the former does nothing if a property does not exist, while the
latter will throw an exception in that situation. Optionally, these
functions can check if the value of a property lies within a valid range.
The following code fragment illustrates how to set and get properties.
The last statement will throw an exception because the property named
``x'' does not exist.

A property can be inserted into or extracted from a nested
\Code{Properties} object by calling the \Code{set()}, \Code{find()} and
\Code{get()} functions with a \emph{compound property name}. Such a name
consists of a series of simple names concatenated by periods characters.
Here is an example:

\IncludeSource{C++}{examples/Properties-2.cpp}

The first statement inserts the integer property ``age'' into the nested
\Code{Properties} object named ``jones''. If this object does not exist
it is created on the fly. The second statement reads the double property
``length'' from the nested \Code{Properties} object named ``brown''. An
exception is thrown if this object does not exist.

%======================================================================

\section{I/O components}

The \Code{io} package exports classes for performing operations on
streams of text and binary data, and for getting information about files
and directories. \autoref{table:jem-io} lists the essential classes that
will be useful when developing applications with Jive.

\begin{table}

  \caption{Summary of the essential I/O classes.}
  \label{table:jem-io}

  \begin{center}

    \begin{tabular}{|>{\ttfamily}lp{10cm}|}
      \hline
      Name        & Description \\
      \hline \hline
      TextInput   & Defines an interface for reading text-based
                    input. \\
      TextOutput  & Defines an interface for writing text-based
                    output. \\
      Reader      & Represents a text input stream. \\
      Writer      & Represents a text output stream. \\
      PrintWriter & Formats a text output stream. \\
      \hline
    \end{tabular}

  \end{center}

\end{table}

%------------------------------------------------------------------------

\subsection*{The \Code{TextInput} class}

The \Code{TextInput} class defines an interface for reading data from a
text input stream. It is typically used to implement functions that parse
text input and that do not care about the source of the input. The
\Code{TextInput} class declares only abstract member functions; these
must be implemented by derived classes.

The non-member function \Code{read()} should be used to read objects from
a text input stream. Overloaded versions of this function are available
for reading fundamental objects such as integers and floating point
numbers, and for reading class instances, such as \Code{String} objects.
Multiple objects can be read in one statement by calling an overloaded
template version of the \Code{read()} function that has more than two
parameters. This is illustrated in the following example.

\IncludeSource{C++}{examples/TextInput.cpp}

%------------------------------------------------------------------------

\subsection*{The \Code{TextOutput} class}

The \Code{TextInput} class defines an interface for writing data to a
text output stream. It is essentially the opposite of the
\Code{TextInput} class.

The non-member function \Code{print()} should be used to write objects to
a text output stream. Overloaded versions of this function are available
for writing fundamental objects such as integers and floating point
numbers, and for writing class instances. Multiple objects can be written
in one statement by calling an overloaded template version of the
\Code{print()} function that has more than two parameters. This is
illustrated in the following example. Note that the special object
\Code{endl} is used to print a newline character.

\IncludeSource{C++}{examples/TextOutput.cpp}

%------------------------------------------------------------------------

\subsection*{The \Code{Reader} class}

The \Code{Reader} class implements the \Code{TextInput} interface and
provides various additional member functions for operating on text input
streams. Since the \Code{Reader} class is also derived from the
\Code{Object} class, \Code{Reader} objects are automatically destroyed
when they are no longer used.

The \Code{Reader} class does not specify where the input is coming from;
this is done by various derived classes, including:

\begin{Description}

  \item[StringReader] reads data from a \Code{String} object.

  \item[FileReader] reads data from a text file.

  \item[GzipFileReader] reads data from a compressed text file.

\end{Description}

%------------------------------------------------------------------------

\subsection*{The \Code{Writer} class}

The \Code{Writer} class implements the \Code{TextOutput} interface and
provides various additional member functions for operating on text output
streams. It is the opposite of the \Code{Reader} class. In fact, many
classes derived from the \Code{Reader} class have a corresponding
class that is derived from the \Code{Writer} class.

The example below shows how to convert a text file into a compressed text
file using the \Code{FileReader} and \Code{GzipFileWriter} classes. Note
that an exception will be thrown if the input file does not exist.

\IncludeSource{C++}{examples/Writer.cpp}

%------------------------------------------------------------------------

\subsection*{The \Code{PrintWriter} class}

The class \Code{PrintWriter} can be used for writing formatted text
output. It provides control over the maximum width of each line, the
indentation level and the way that numbers are formatted. It is to be
used as a wrapper around a \Code{Writer} object the specifies where the
formatted output should go, as illustrated in the following example. Note
that the public member object named \Code{nformat} contains the settings
for formatting numbers.

\IncludeSource{C++}{examples/PrintWriter.cpp}

The output of the above code will look like this:

\begin{Verbatim}
  Thank you for visiting ShopUntilYouDrop! We hope that you will
  enjoy your new goodies and that we will see you again soon.

  Here is a summary of your purchase:

    Number of items  ......      4
    Price per item   ......  42.24
    Total amount     ......  169.0

  Bye bye!
\end{Verbatim}

%======================================================================

\section{Numerical components}

The \Code{numeric} package contains classes and functions for building
numerical algorithms. They can be used to multiply matrices and vectors;
to solve linear systems of equations; to define and evaluate functions;
and to perform various other types of operations. The essential functions
and classes are listed in \autoref{table:jem-numeric}.

\begin{table}

  \caption{Summary of the essential numerical classes and functions.}
  \label{table:jem-numeric}

  \begin{center}

    \begin{tabular}{|>{\ttfamily}lp{10cm}|}
      \hline
      Name         & Description \\
      \hline \hline
      matmul       & \\
      MatmulChain  & \\
      LUSolver     & \\
      SparseMatrix & \\
      Function     & \\
      \hline
    \end{tabular}

  \end{center}

\end{table}


