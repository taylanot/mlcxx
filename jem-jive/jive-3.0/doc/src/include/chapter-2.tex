
\chapter{The basics; writing a program with Jive and Jem}
\label{chapter:the-basics}

This chapter focuses on the technical aspects of using Jive and Jem.
After reading this chapter you will know how to include components from
Jive and Jem into your source code and how to translate your source code
into an executable program.

Section~\ref{section:the-basics:using-components} starts this chapter by
explaining how to use the classes and functions provided by Jive and Jem.
It explains how Jive and Jem are organized; where to to find the header
file containing the declaration of a particular class or function; and
how to refer to the name of a class or function.
Section~\ref{section:the-basics:compilation} then shows how to compile
your program and how to link the resulting object file with the Jive and
Jem libraries. This section also shows how to simplify the compilation
and linkage procedure by using portable makefiles. Finally,
Section~\ref{section:the-basics:sample-program} presents a simple Jive
program together with a makefile.

%========================================================================

\section{Using classes and functions}
\label{section:the-basics:using-components}

From a programmer's point of view, Jive -- and Jem too -- consists of a
collection of C++ header files containing class and function declarations
and definitions. As usual, you include the contents of a header file by
putting an \Code{\#include} directive somewhere at the top of your source
file. To limit the header file sizes and to speed up the compilation of a
program, a header file typically contains only a single class definition
and/or a limited set of related functions. This implies that any
non-trivial program will have to include several header files, and that
you will have to know which header file contains which class/function
declaration.

%------------------------------------------------------------------------

\subsection*{Organization of Jive and Jem; working with packages}

Both Jive and Jem bundle related classes and functions into modular units
called \emph{packages}. For instance, the \Code{fem} package in Jive
contains all classes and functions that are aimed at developing finite
element programs. Except for a several core packages, most packages can
be installed and removed on an individual basis\footnote{This is not
entirely true because some packages depend on other packages.} so that
you can easily adapt Jive and Jem to your specific needs.

Here is an overview of the packages that are available in Jive:
\begin{Description}[\Code]

\item[util]    contains general utility classes, including classes for
  storing and manipulating unstructured grid.

\item[graph]   contains classes for partitioning graphs and for
  performing other operations on graphs.

\item[mp]      contains classes for exchanging data between parallel
  threads and processes.

\item[algebra] contains classes for storing and building sparse matrices,
  and for performing matrix/vector operations.

\item[geom]    contains classes for evaluating the geometrical properties
  of basic shapes and for computing element shape functions.

\item[model]   contains classes for implementing numerical models.

\item[app]     contains a modular, high-level application framework.

\item[gl]      contains classes for real-time data visualization.

\item[solver]  contains classes for solving linear systems of
  equations.

\item[fem]     contains classes that provide specific support for finite
  element applications.

\item[femodel] contains classes that implement finite element models.

\item[implict] contains classes for solving non-linear and time-dependent
  systems of equations.

\end{Description}

Jem provides the following packages that are used by Jive:
\begin{Description}[\Code]

\item[base]    contains foundation classes and functions.

\item[io]      contains classes for performing basic input and output
  operations.

\item[util]    contains general utility classes and containers such as
  hash tables.

\item[numeric] contains classes that provide support for numerical
  applications.

\item[mp]      contains classes for implementing parallel programs that
  are based on the message passing programming model.

\item[gl]      contains classes for managing and displaying scene
  graphs.

\item[xml]     contains classes for parsing and printing XML-formatted
  files.

\item[xutil]   contains special-purpose container classes.

\end{Description}

To use a class or function from a package you have to include the header
file containing the class or function declaration. By convention, all
header files provided by a Jive package are located in the directory
\Code{jive/}\Symbol{package-name} (assuming that the header file search
path of the compiler has been set up correctly; see the next section for
more information). Likewise, the header files provided by a Jem package
are located in the directory \Code{jem/}\Symbol{package-name}. The name
of the header file containing a particular class declaration can
generally be obtained by appending the suffix `\Code{.h}' to the name of
the class. Thus, the full pathname of the header file containing the
declaration of the class \Code{Constraints} in the package \Code{util} in
Jive is \Code{jive/util/Constraints.h}. Note that this rule is not
applicable to all classes; consult the reference manual to be sure.

%------------------------------------------------------------------------

\subsection*{Namespaces}

To avoid name collisions, all members (classes, functions, etc.) of a
Jive package are declared in the namespace
\Code{jive::}\Symbol{package-name}. The fully qualified name of the class
\Code{Constraints} in the Jive package \Code{util} is therefore
\Code{jive::util::Constraints}. Jem uses a similar naming scheme: all
members in a Jem package are declared in the namespace
\Code{jem::}\Symbol{package-name}. The members of the Jem package
\Code{base}, however, are declared in the namespace \Code{jem}. Thus, the
fully qualified name of the class \Code{String} from the \Code{base}
package is \Code{jem::String} and \emph{not}
\Code{jem::base::String}.

There are three ways to use names from a namespace: write the fully
qualified names; write one or more \emph{using declarations}; and write
one or more \emph{using directives}. The code fragment below shows an
example in which all three methods are used.

\IncludeSource{C++}{examples/namespaces.cpp}

Writing fully qualified names is the safest method because changes in one
part of your code can not lead to unexpected name collisions in another
part of your code. However, writing fully qualified names becomes tedious
pretty soon and also leads to code that is harder to read.

Using directives are at the other end of the spectrum: they minimize the
required typing effort, but they can also easily lead to name collisions
when you make small changes to your code. What is worse, changes in
external code, such as Jem, can also lead to name collisions in your own
code. Using directives should therefore only be used in small programs
with a limited lifespan.

Using declarations allow you to strike a balance between convenience
(less typing) and robustness (less chances for name collisions). They are
best used within functions, but for convenience you can also put them at
the top of your source files. You can also put them in header files, but
this increases the chances of name collisions. A better approach is to
create one header file -- named \Code{import.h}, for instance -- that
contains forward declarations and using declarations for names that you
use frequently throughout your program; see the example below.

\IncludeSource[title=\Code{import.h}]{C++}{examples/import.h}

Note that the header file \Code{import.h} uses forward class declarations
instead of including the header files containing the full class
definitions. This setup minimizes the compilation time for the source
files that include \Code{import.h} and use only some of those
classes.

%========================================================================

\section{Compiling and linking}
\label{section:the-basics:compilation}

Before you can run a Jive program -- or any C++ program -- you will have
to transform the source code into an executable program. This involves
two phases: compilation and linking. First, each source file is compiled
into a binary \emph{object file}. All the object files are then linked
with each other, with the Jive and Jem libraries, and with all other
required system libraries. The result of this linking process is an
executable program.

The next two sub-sections provide a more detailed description of the
compilation and linking processes when not using an integrated
development environment. The sub-section after that explains how you can
simplify the compilation and linking processes by using portable
makefiles. It is recommended that you read these three sub-sections, even
if you already are familiar with the compilation and linking
processes.

Some of the instructions given below involve the pathnames of the Jive
and Jem libraries. Since these libraries may be installed anywhere in a
filesystem, it is assumed that the environment variables \Code{JIVEDIR}
and \Code{JEMDIR} have been set to the full pathname of the Jive library
and Jem library, respectively. The notation \Code{\$JEMDIR} will be used
to refer to the value of the environment variable \Code{JEMDIR}.

%------------------------------------------------------------------------

\subsection*{The compilation procedure}

To compile a source file that includes components from Jive and/or Jem
you have to:
\begin{enumerate}

\item append the directory \Code{\$JIVEDIR/include} to the header file
  search path of the compiler;

\item append the directory \Code{\$JEMDIR/include} to the header file
  search path of the compiler;

\item append the necessary flags to the argument list of the compiler;

\item invoke the compiler.

\end{enumerate}
The command to compile a source file named \Code{source.cpp} typically
looks something like this:
\begin{verbatim}
  CC -I$JIVEDIR/include -I$JEMDIR/include -O -c source.cpp
\end{verbatim}
with \Code{CC} the name of the C++ compiler. The flag \Code{-O} indicates
that the compiler should turn on optimizations, and the flag \Code{-c}
indicates that the compiler should only produce an object file.

Although the above command will work with many compilers, some require
different and/or additional arguments. Consult the manual of your
compiler to make sure that you are using the correct arguments.

%------------------------------------------------------------------------

\subsection*{The linking procedure}

To create an executable program from one or more object files you have
to:
\begin{enumerate}

\item append the directories \Code{\$JIVEDIR/lib} and
  \Code{\$JEMDIR/lib} to the directory search path of the
  compiler;

\item append the directories of any non-standard libraries that are used
  to the directory search path of the compiler.

\item link your program with all your object files;

\item link your program with the required package-specific Jive
  libraries;

\item link your program with the required package-specific Jem
  libraries;

\item and link your program with the required system-specific libraries.

\end{enumerate}
Each package in Jem and Jive provides its own library containing all
object files related to that package. .When your program uses a class or
function from a package, you must link the program with the
package-specific library which is named \Code{jem}\Symbol{package} or
\Code{jive}\Symbol{package}. There are three exceptions: the packages
\Code{base}, \Code{io} and \Code{util} from Jem are combined in a single
library named \Code{jem}.

There are two ways to determine which system libraries are required. The
first one is to keep adding libraries until all symbols have been
resolved. Of course, you need to be fairly familiar with the system
libraries on your operating system for this scheme to work. Another,
and much more elegant way is to create a portable make file as is
explained in the next sub-section.

Note that the order in which the libraries are linked with the program is
significant with most linkers. In general, you should first list the Jive
libraries, then the Jem libraries, and then the system-specific
libraries.

%------------------------------------------------------------------------

\subsection*{Using portable makefiles}

To reduce the complexity of the compilation and linking process, each
package in Jive (and Jem too) provides a special makefile that specifies
which Jem/Jive libraries and which system libraries are required by that
package. They also specify which directories must be appended to the
header file search path and the library search path of the compiler. By
convention, a package-specific makefile is named
\Symbol{package-name}\Code{.mk} and is located in the directory
\Code{\$JIVEDIR/\-makefiles/\-packages} (or
\Code{\$JEMDIR/\-makefiles/\-packages}). The package-specific makefiles
rely on several features that are specific to GNU make. If GNU make
is not yet installed on your system, you can download a free copy from
the GNU website \url{www.gnu.org}.

You can use the package-specific makefiles by including them into your
own makefile. For instance, if your program uses the packages
\Code{solver} and \Code{fem} from Jive, you should put the following into
your makefile:
\begin{Source}{make}
  include $(JIVEDIR)/makefiles/packages/solver.mk
  include $(JIVEDIR)/makefiles/packages/fem.mk
\end{Source}

Note that the order of the include statements is not important. The
package-specific makefiles automatically resolve inter-package
dependencies, so you do not have to know that the package \Symbol{solver}
internally makes use of the package \Symbol{util}. This also works
between Jive and Jem packages. That is, a package-specific makefile from
Jive will automatically include the makefiles from the Jem packages on
which it depends. In the case that you are lazy and do not want to keep
track of which packages your program actually uses, you can simply put
this into your makefile:
\begin{Source}{make}
  include $(JIVEDIR)/makefiles/packages/*.mk
\end{Source}
But laziness has its price, and this construction may result in a
bunch of warnings from your compiler that some libraries are not used
to resolve any symbols.

When you include one or more package-specific makefiles into your
makefile, you still need some rules to actually build an executable (or
multiple executables) from your source files. Jive can help you out here,
provided that you only need to build a single executable. In this
case you only have to define the name of the executable, include all
required package-specific makefiles, and then include the makefile
\Code{\$JIVEDIR/\-makefiles/\-prog.mk}. The last makefile defines the
following rules:
\begin{Description}[\Code]

\item[\$(program)] builds an executable named \Makexp{program} by
  compiling and linking all C++ source files in the current directory.
  The variable \Code{program} must have been defined before the makefile
  \Code{\$JIVEDIR/\-makefiles/\-prog.mk} is included. This is the
  default make rule.

\item[opt] builds an optimized executable named
  \Makexp{program}\Code{-opt}. This executable will run faster than the
  standard one because it will perform less runtime checks.

\item[debug] builds an executable names \Makexp{program}\Code{-dbg} that
  contains debugging information.

\item[clean] deletes all object files and core dumps in the current
  directory. It also deletes any temporary files that may have been
  generated by the compiler.

\item[clean-all] invokes the \Code{clean} rule, and then deletes all
  executable programs.

\end{Description}

Your makefile may define a variable \Code{subdirs} that lists all
sub-directories containing source files that are part of your program.
You must define this variable before including the makefile
\Code{prog.mk} from Jive. Note that the names of your source files must
be unique; you can not have two source files with the same names that are
located in different sub-directories.

The package-specific makefiles define a (large) collection of variables
that can be used to write your own rules for compiling and linking
programs. The Jem reference manual provides a description of these
variables and explains how you can tune a makefile to your specific
needs.

%========================================================================

\section{A sample program}
\label{section:the-basics:sample-program}

The listing below contains the source code of a simple Jive program,
named \Code{linear}, that reads a data file, assembles and solves a
linear system of equations, and writes an output file. Do not worry if
you do not understand the details; the purpose of this program is not to
show what you can do with Jive but to show what a typical Jive program
looks like.

\IncludeSource[title=\Code{linear.cpp}]{C++}{examples/linear.cpp}

A makefile for this program is listed below. Note that only the makefile
from the \Code{implict} package needs to be included because this
makefile will include the makefiles from all other packages that are
needed.

\begin{Source}{make}
  program = linear
  include $(JIVEDIR)/makefiles/packages/implict.mk
  include $(JIVEDIR)/makefiles/prog.mk
\end{Source}
