\documentclass[a4paper,10pt]{article}

\usepackage[text={16cm,24cm},centering]{geometry}
\usepackage{calc}
\usepackage{ifthen}
\usepackage{epsfig}
\usepackage{palatino}
\usepackage{listings}
\usepackage{parskip}
\usepackage{array}
\usepackage{longtable}
\usepackage{hyperref}
\usepackage{parskip}
\usepackage{underscore}

\lstloadlanguages{C++,make,sh,csh}
\graphicspath{{images/}}
\hypersetup{colorlinks,raiselinks}

%------------------------------------------------------------------------
%   Simple commands
%------------------------------------------------------------------------


\renewcommand{\chapterautorefname}{Chapter}
\renewcommand{\sectionautorefname}{Section}
\renewcommand{\subsectionautorefname}{Subsection}

\newcommand{\Code}[1]{\texttt{#1}}
\newcommand{\Symbol}[1]{\textsl{#1}}
\newcommand{\Makexp}[1]{\texttt{\$(#1)}}
\newcommand{\JiveWebsite}{\url{www.dynaflow.com/software/jive-sdk}}


%------------------------------------------------------------------------
%   Listings commands and environments
%------------------------------------------------------------------------


\lstdefinestyle{default}
{
  basicstyle=\ttfamily, numbers=none, numberstyle=\tiny,%
  showstringspaces=false,
}

\lstdefinestyle{C++}
{
  style=default, language=C++
}

\lstdefinestyle{make}
{
  style=default, language=[gnu]make, showtabs=true
}

\lstdefinestyle{sh}
{
  style=default, language=sh
}

\lstdefinestyle{csh}
{
  style=default, language=csh
}

\lstnewenvironment{Verbatim}[1][]
{
  \lstset{style=default, xleftmargin=5mm, gobble=2, #1}
}
{}

\lstnewenvironment{Source}[2][]
{
  \lstset{style=#2, xleftmargin=5mm, gobble=2, #1}
}
{}

\newcommand{\IncludeSource}[3][]
{
  \lstinputlisting[style=#2, frame=lines,%
    aboveskip=\baselineskip,basicstyle=\small\ttfamily, #1]{#3}
}


%------------------------------------------------------------------------
%   Description environment
%------------------------------------------------------------------------


\newlength{\DescrLength}
\newcounter{DescrCounter}

\newcommand{\DescrFormat}[1]{#1}

\newcommand{\DescrLabel}[1]
{%
  \settowidth{\DescrLength}{\DescrFormat{#1}}%
  \ifthenelse{\lengthtest{\DescrLength > \labelwidth}}%
  {%
    \parbox[b]{\labelwidth}%
    {%
      \makebox[0pt][l]{\DescrFormat{#1}}\\\mbox{}%
    }%
  }{%
    {\DescrFormat{#1}}%
  }%
  \hfil\relax
  \refstepcounter{DescrCounter}
}


\newenvironment{Description}[1][\textnormal]
{
  \begin{list}{h}
  {
    \renewcommand{\DescrFormat}{#1}
    \renewcommand{\makelabel}{\DescrLabel}
    \setlength{\labelwidth}{30pt}
    \setlength{\itemindent}{0pt}
    \setlength{\leftmargin}{\labelwidth + \labelsep + \labelsep}
    \setlength{\rightmargin}{0pt}
  }
}
{ \end{list} }


%------------------------------------------------------------------------
%   MakevarTable environment
%------------------------------------------------------------------------


\newenvironment{MakevarTable}[1]
{
  \begin{longtable}[c]{|>{\ttfamily}lp{9.2cm}|}

    \caption*{#1}\\
    \hline
    \endhead

    \hline
    \endfoot

    JEM\_CXX\_OPT\_FLAGS1 & Rubbish \kill
}
{
  \end{longtable}
}



%%%%%%%%%%%%%%%%%%%%%%%%%%%%%%%%%%%%%%%%%%%%%%%%%%%%%%%%%%%%%%%%%%%%%%%%%

\begin{document}

\begin{titlepage}

  \begin{center}

    \vspace*{2cm}

    {\Huge \textsc{Jem user manual}}

    \vspace{1cm}

    {\large \textsc{version 2.0}}

    \vspace{2cm}

    {\large \textsc{Dynaflow Research Group}}

  \end{center}

\end{titlepage}

\tableofcontents

\newpage

%========================================================================

\section{Introduction}

Jem is an object oriented programming toolkit written in C++. It combines
the efficiency and versatility of C++ with several powerful concepts
provided by Fortran~90, Java and C\# that facilitate the development of
robust, modular software. These concepts include:
\begin{Description}[\textit]

\item[Automatic memory management] allocate objects that are
  automatically deleted when they are no longer being referenced.

\item[Multi-dimensional array syntax] write expressions that operate
  on multi-dimensional arrays.

\item[Events] build completely independent software components and
  connect them with events.

\item[Object serialization] read/write graphs of objects from/to a
  file in a machine-independent format.

\item[Parallel programming] write concurrent programs that run
  efficiently on multi-processor machines.

\item[Reflection] discover the capabilities of objects at run-time.

\end{Description}

In addition to these general concepts, Jem provides a collection of
classes and functions that hide system-dependent programming interfaces.
Consequently, an application built on top of Jem can easily be ported
from one operating system to another.

This manual briefly explains how to get started with Jem. It does not
provide a thorough description of the classes and functions available in
Jem. For that, you should consult the online reference manual, located in
the directory \Code{doc} of your Jem distribution.

The remainder of this manual is organized as follows. First,
\autoref{section:packages} explains how Jem is divided into packages and
how these packages can be used. Next, \autoref{section:namespaces}
explains briefly how to deal with namespaces. After that,
\autoref{section:compiling-and-linking} explains how to compile and link
programs based on Jem, and \autoref{section:portable-makefiles} then
shows how you can simplify the compilation and linking process by writing
portable makefiles. \autoref{section:sample-program} then presents a
simple Jem program and makefile.

The last three sections provide some additional information that is not
immediately required to get started with Jem.
\autoref{section:package-overview} provides an overview of the packages
in Jem. \autoref{section:environment-variables} lists the environment
variables with which the runtime behavior of a Jem program can be tuned.
\autoref{section:make-variables} provides information that may be useful
when building advanced makefiles.

%========================================================================

\section{Working with packages \label{section:packages}}

Like the Java runtime library, Jem bundles groups of related classes and
functions into modular units called \emph{packages}. Except for the three
core packages \Code{base}, \Code{io} and \Code{util}, most packages can
be installed and removed on an individual basis. This means that you can
easily adapt Jem to your specific needs.

To avoid naming conflicts, all members of a package are declared in the
namespace \Code{jem::}\Symbol{package-name}. All members of the package
\Code{base}, however, are declared in the namespace \Code{jem}. The full
name of the class \Code{String} in the package \Code{base} is therefore
\Code{jem::\-String}, while the full name of the class
\Code{Input\-Stream} in the package \Code{io} is
\Code{jem::\-io::\-Input\-Stream}.

To use a class from a package you have to include the header file
containing the class definition. By convention, all header files
provided by a package are located in the directory
\Code{jem/}\-\Symbol{package-name} (assuming that the header file
search path of the compiler has been set up correctly; see the
\autoref{section:compiling-and-linking} for more information). The
name of the header file defining a particular class can generally be
obtained by appending the suffix `\Code{.h}' to the name of the class.
Thus, the full pathname of the header file defining the class
\Code{Input\-Stream} in the package \Code{io} is
\Code{jem/\-io/\-Input\-Stream.h}. Note that this rule is not
applicable to all classes; consult the reference manual to be sure.

A package comprises two directories that are located in the directories
\Code{\$JEMDIR/\-include/\-jem} and \Code{\$JEMDIR/\-packages}, where
\Code{\$JEMDIR} denotes the name of the Jem installation directory. The
package \Code{io}, for instance, comprises the directories
\Code{\$JEMDIR/\-include/\-jem/\-io} and \Code{\$JEMDIR/\-packages/\-io}.
The first one contains the header files and the second one contains the
source code, the documentation and example programs. Although the layout
and location of a package is not really relevant when writing code, it
can be useful to know where to look if you would like to check out an
header file or an example program.

%========================================================================

\section{Dealing with namespaces \label{section:namespaces}}

To use a class, function, or variable that has been declared in a
namespace, you will have to tell the compiler in which namespace to
search. Although this is not a Jem-specific issue, experience has shown
that not all new users of Jem are familiar with namespaces. If that
applies to you, you may benefit from the remainder of this section.

There are three ways to tell the compiler which namespace to search:
write fully qualified names; write one or more \emph{using declarations};
and write one or more \emph{using directives}. The code fragment below
shows an example in which all three methods are used.

\IncludeSource{C++}{examples/namespaces.cpp}

Writing fully qualified names is the safest method because changes in one
part of your code can not lead to unexpected name collisions in another
part of your code. However, writing fully qualified names becomes tedious
pretty soon and also leads to code that is harder to read.

Using directives are at the other end of the spectrum: they minimize the
required typing effort, but they can also easily lead to name collisions
when you make small changes to your code. What is worse, changes in
external code, such as Jem, can also lead to name collisions in your own
code. Using directives should therefore only be used in small programs
with a limited lifespan.

Using declarations allow you to strike a balance between convenience
(less typing) and robustness (less chances for name collisions). They are
best used within functions, but for convenience you can also put them at
the top of your source files. You can also put them in header files, but
this increases the chances of name collisions. A better approach is to
create one header file -- named \Code{import.h}, for instance -- that
contains forward declarations and using declarations for names that you
use frequently throughout your program; see the example below.

\IncludeSource[title=\Code{import.h}]{C++}{examples/import.h}

Note that the header file \Code{import.h} uses forward class declarations
instead of including the header files containing the full class
definitions. This setup minimizes the compilation time for the source
files that include \Code{import.h} and use only some of those
classes.

%========================================================================

\section{Compiling and linking a Jem program
  \label{section:compiling-and-linking}}

To compile and link a program based on Jem you have to:
\begin{enumerate}

\item set the environment variable \Code{JEMDIR} to the directory where
  Jem has been installed;

\item append the directory \Code{\$JEMDIR/include} to the header file
  search path of your compiler;

\item append the directory \Code{\$JEMDIR/lib} to the library search path
  of your compiler;

\item link the program with the appropriate package-specific libraries;

\item link the program with the core Jem library;

\item link the program with all system libraries that are used by Jem;

\item link the program with the standard math library.

\end{enumerate}

Each package provides its own library -- typically named
\Code{jem}\-\Symbol{package-name} -- containing all object files related
to that package. When your program uses a class or function from a
package, you must link the program with the package-specific library. For
instance, if your program uses the packages \Code{xml} and
\Code{numeric}, you must link the program with the libraries
\Code{jem\-xml} and \Code{jem\-numeric}, both of which are located in the
directory \Code{\$JEMDIR/lib}.

The core Jem library, named \Code{jem}, contains all object files from
the core packages \Code{base}, \Code{io} and \Code{util}. There are no
package-specific libraries for these three packages.

There are two ways to determine which system libraries are required. The
first one is to keep adding libraries until all symbols have been
resolved. Of course, you need to be fairly familiar with the system
libraries on your operating system for this scheme to work. Another, and
much more elegant way is to create a portable makefile as is explained in
the \hyperref[section:portable-makefiles]{next} section.

Note that the order in which the libraries are linked with the program
is significant with most linkers. In general, you should first list
all package-specific libraries, then the core Jem library, then all
system libraries used by Jem, and finally the standard math library.

Here is a typical command that compiles and links a Jem program named
\Code{parser} on a Linux system:
\begin{Source}{sh}
  g++ -I$JEMDIR/include -L$JEMDIR/lib parser.cpp \
      -o parser -ljemxml -ljem -lpthread -lm
\end{Source}
The option \Code{-lpthread} specifies that the program should be
linked with the POSIX thread library. This is one of the system
libraries used by Jem.

Experience has shown that one often wants to compile a program with extra
checks during the development stage, but that those checks should be
disabled when the program is going to be deployed. Jem therefore provides
two sets of libraries: one that has been compiled with many checks turned
on, and one that has been compiled with most critical checks turned off
and with aggressive compiler optimization switches turned on. The first
set of libraries is located in the directory \Code{\$JEMDIR/\-lib}, while
the second one is located in \Code{\$JEMDIR/\-lib-opt}. Thus, by simply
changing the library search path for the compiler, one can easily switch
from development to deployment mode.

%========================================================================

\section{Creating portable makefiles
  \label{section:portable-makefiles}}

To reduce the complexity of the compilation and linking process, each
package in Jem provides a special makefile that specifies which Jem
libraries and which system libraries are required by that package. They
also specify which directories must be appended to the header file search
path and the library search path of the compiler. By convention, a
package-specific makefile is named \Symbol{package-name}\Code{.mk} and is
located in the directory \Code{\$JEMDIR/\-makefiles/\-packages}. The
package-specific makefiles rely on several features that are specific to
GNU make. If GNU make is not yet installed on your system, you can
download a free copy from the GNU website \url{www.gnu.org}.

You can use the package-specific makefiles by including them into your
own makefile. For instance, if your program uses the packages
\Code{mp} and \Code{xml}, you should put the following into your
makefile:
\begin{Source}{make}
  include $(JEMDIR)/makefiles/packages/mp.mk
  include $(JEMDIR)/makefiles/packages/xml.mk
\end{Source}
Note that the order of the include statements is not important. The
package-specific makefiles automatically resolve inter-package
dependencies, so you do not have to know that the package \Symbol{foo}
internally makes use of \Symbol{bar}. Since all packages depend on
the core package \Code{base}, \Code{io} and \Code{util}, their
package-specific makefiles do not have to be included explicitly when you
include any other package-specific makefile. In the case that you are
lazy and do not want to keep track of which packages your program
actually uses, you can simply put this into your makefile:
\begin{Source}{make}
  include $(JEMDIR)/makefiles/packages/*.mk
\end{Source}
But laziness has its price, and this construction may result in a
bunch of warnings from your compiler that some libraries are not used
to resolve any symbols.

Before including a package-specific makefile, make sure you have set the
environment variable \Code{JEMDIR} to the name of the Jem directory. You
can do this on Windows by using the control panel, and on Linux and MacOS
X by using a terminal. If, for instance, Jem is located in the
directory \Code{/usr/local/lib/DRG/jem}, then execute
\begin{Source}{sh}
  export JEMDIR=/usr/local/lib/DRG/jem
\end{Source}
when using a \Code{bash}-like shell, or execute
\begin{Source}{csh}
  setenv JEMDIR /usr/local/lib/DRG/jem
\end{Source}
when using a \Code{csh}-like shell. Of course, you can also put the
appropriate command into your shell start-up script (typically
\Code{.bashrc} or \Code{.cshrc} in your home directory).

The package-specific makefiles define a number of variables that you can
use in your own makefile. The ones that you are most likely to need are
briefly described below.
\begin{Description}[\Code]

\item[LIBS] the libraries and object files that need to be linked with
  your program. This list is sorted according to the dependencies between
  the libraries.

\item[LIBDIRS] the directories to be searched for the libraries.

\item[LIBDIRS_OPT] the directories to be searched for optimized
  libraries.

\item[LIBDIRS_DBG] the directories to be searched for libraries with
  debugging information.

\item[INCDIRS] the directories to be searched for the header files.

\item[CXX] the name of the C++ compiler that has been used to compile the
  Jem libraries.

\item[CXX_FLAGS] the compiler options for producing normal object
  files.

\item[CXX_FLAGS_OPT] the compiler options for producing optimized object
  files. These options also disable some runtime checks.

\item[CXX_FLAGS_DBG] the compiler options for producing object files that
  contain debugging information.

\item[Compile] a sequence of commands for creating an executable program
  (the current target, \Code{\$@}) from a single C++ source file (the
  current prerequisite, \Code{\$<}).

\item[CompileExe] a sequence of commands for creating an executable
  program (the current target, \Code{\$@}) from a single C++ source file
  and a set of object files (the current prerequisites, \Code{\$\^}).

\item[LinkExe] a sequence of commands for linking a set of object files
  (the current prerequisites, \Code{\$\^}) into an executable program
  (the current target, \Code{\$@}).

\item[LinkDLL] a sequence of commands for linking a set of object files
  (the current prerequisites, \Code{\$\^}) into a dynamically shared
  library (the current target, \Code{\$@}).

\end{Description}
See \autoref{section:make-variables} for a complete list of make
variables.

It is possible that your program requires a library that is not located
in one of the directories listed by \Makexp{LIBDIRS}. You could, of
course, add the required directory to the \Code{LIBDIRS} variable, but
this may break some of the make rules that you have defined previously.
A better solution is to assign your list of additional directories to the
``customization'' variable \Code{MY\_LIBDIRS}. Since this variable is part
of the definition of the variable \Code{LIBDIRS}, your list of
directories will automatically be pre-pended to the existing list of
directories. The same method can be uses to customize the variables
\Code{LIBDIRS\_OPT}, \Code{LIBDIRS\_DBG} and \Code{INCDIRS}. The compiler
options can also be customized, but in a slightly different way. See
\autoref{section:make-variables} for the details.

When you include one or more package-specific makefiles into your
makefile, you still need some rules to actually build an executable (or
multiple executables) from your source files. Jem can help you out here,
provided that you only need to build a single executable. In this case
you only have to define the name of the executable; include all required
package-specific makefiles; and then include the makefile
\Code{\$JEMDIR/\-makefiles/\-prog.mk}. The last makefile defines the
following rules:
\begin{Description}[\Code]

\item[\$(program)] builds an executable named \Makexp{program} by
  compiling and linking all C++ source files in the current directory.
  The variable \Code{program} must have been defined before the makefile
  \Code{\$JEMDIR/\-makefiles/\-prog.mk} is included. This is the default
  make rule.

\item[opt] builds an optimized executable named
  \Makexp{program}\Code{-opt}. This executable will run faster than the
  standard one because it will perform less runtime checks.

\item[debug] builds an executable names \Makexp{program}\Code{-dbg} that
  contains debugging information.

\item[clean] deletes all object files and core dumps in the current
  directory. It also deletes any temporary files that may have been
  generated by the compiler.

\item[clean-all] invokes the \Code{clean} rule, and then deletes all
  executable programs.

\end{Description}
The next section shows an example makefile that includes the makefile
\Code{prog.mk}.

Your makefile may define a variable \Code{subdirs} that lists all
sub-directories containing source files that are part of your program.
You must define this variable before including the makefile
\Code{prog.mk} from Jem. Note that the names of your source files must
be unique; you can not have two source files with the same names that are
located in different sub-directories.

If you want to write your own rules to compile and/or link a program, you
are advised to use the variables \Code{Compile}, \Code{CompileExe} and
\Code{LinkExe}. You can also use the \Code{CXX} to invoke the compiler
directly, but this can lead to problems with library and header file
search paths. A better solution is to use the compiler wrapper script
\Code{\$JEMDIR/\-bin/\-CC} that accepts the same set of command-line
options on all systems.

%========================================================================

\section{A sample Jem program and makefile
  \label{section:sample-program}}

The example program \Code{unzip.cpp} below shows how to use various Jem
classes to ``unzip'' a series of compressed gzip-files (see also
\Code{www.\-gzip.\-org}). The source code starts with a few
\Code{\#include} directives that import all required class
definitions. It then continues with two using directives that tell the
compiler to search the namespaces \Code{jem} and \Code{jem::io} during
name resolution. These directives save some typing and makes the source
code easier to read. The real action starts in the function \Code{unzip}
that executes a loop over all files that have been specified on the
command line. Each file that has the extension \Code{.gz} is opened with
a \Code{Gzip\-File\-Reader} object that handles the decompression of the
data stored in the file. The decompressed data are then written to
another file that has the same name as the original one minus the
\Code{.gz} suffix.

\IncludeSource[title=\Code{unzip.cpp}]{C++}{examples/unzip.cpp}

The entry point of the program, the function \Code{main}, simply calls
the static member function \Code{exec} of the \Code{System} class,
specifying the address of the \Code{unzip} function as one of the
parameters. This may seem a bit overly complicated, especially if you
know that the \Code{exec} function does not much more than calling the
\Code{unzip} function. However, the \Code{exec} function also catches and
prints any exception that is thrown by the \Code{unzip} function and the
functions that it calls. It therefore saves you from writing the same
exception handling code over and over again.

Creating a makefile for the example program is really straightforward.
In fact, you only need to define the name of the program, include the
makefiles from the packages \Code{base} and \Code{io}, and include the
makefile \Code{prog.mk}. To make things a bit more interesting, let us
assume that the program is to be linked statically so that the
executable can be run on a system with incompatible system libraries.
This can be achieved by setting the customization make variable
\Code{MY\_CXX\_LNK\_FLAGS} so that the option
\Code{-static}\footnote{this works for the GNU linker; other linkers
  may require a different flag.} is appended to the list of linker
options. The complete makefile is shown below.

\IncludeSource[title=\Code{Makefile}]{make}{examples/Makefile}

\appendix

%========================================================================

\section{Package overview \label{section:package-overview}}

The current version of Jem comprises the following packages:
\begin{Description}[\Code]

\item[base]  exports fundamental Jem classes.

\item[io]    exports classes dealing with I/O.

\item[util]  exports container classes and utility functions.

\item[xutil] exports special-purpose container classes.

\item[gl]    exports classes for data visualization.

\item[mp]    exports classes for building parallel programs.

\item[mt]    exports classes for writing multi-threaded programs.

\item[numeric]
             exports classes for developing numerical programs.

\item[xml]   exports classes for parsing XML-formatted text files.

\end{Description}
These packages are described briefly in the following sub-sections. For a
detailed description of the classes and functions in a package you should
consult the reference manual for that package.

Jem defines three integral types -- named \Code{byte}, \Code{lint} and
\Code{idx\_t} -- that are not part of any package. They are declared in
the namespace \Code{jem} and are available through any Jem header file.
The \Code{byte} type is just an alias for the \Code{unsigned char} type.
The \Code{lint} type is a large, signed integer type. It is guaranteed to
be a 64-bit integer type on 64-bit systems. The \Code{idx\_t} type is a
signed integer type that is used for representing sizes of container
classes and for accessing items in a container by index. This type is
equal to the \Code{int} type, but it can be set equal to the \Code{lint}
type when configuring Jem.

\subsection{The \Code{base} package}

The package \Code{base} contains the fundamental classes and functions on
top of which the rest of Jem has been built. The most important of these
classes are listed below.
\begin{Description}[\Code]

\item[String] represents a character array. It provides functions for
  searching, slicing, and concatenation, among others. It also provides
  all the usual string comparison functions.

\item[Array] represents a multi-dimensional array. It provides functions
  for creating multi-dimensional array slices, and supports efficient
  Fortran~90-style array expressions. It also provides support for
  writing array expressions using a kind of tensor notation.

\item[Collectable] implements an automatic memory management scheme. All
  objects derived from \Code{Collectable} are automatically deleted when
  they are no longer used.

\item[Object] is the super class of many other classes in Jem. It can
  therefore be used to implement heterogeneous data structures and
  generic algorithms.

\item[System] provides access to various system-level facilities. It can
  be used, among others, to get and set the standard input and output
  streams; to get the contents of environment variables; and to catch and
  print exceptions thrown by the main function of a program.

\item[Thread] represents a thread of execution. It provides functions for
  starting and joining threads, and for accessing thread-local data. In
  addition to the \Code{Thread} class, the \Code{base} package also
  provides various synchronization data types.

\item[Throwable] provides a common base for all exception classes in Jem.
  A \Code{Throwable} object encapsulates a context string, describing the
  place where the exception was thrown, an error string that indicates
  why the exception was thrown, and possibly a stack trace that describes
  the function-call path at the time that the exception was thrown.

\end{Description}
The \Code{base} package also exports a reflection framework that can
be used to query objects for their fields (data members) and methods
(member functions).


\subsection{The \Code{io} package}

The \Code{io} package exports classes for reading and writing both text
and binary files. The main classes are:
\begin{Description}[\Code]

\item[Reader] \label{class:Reader} represents a text input stream and
  provides methods for reading character data. The \Code{Reader} class
  does not specify where the characters come from; that must be specified
  by a derived class.

\item[Writer] \label{class:Writer} represents a text output stream and
  provides functions for writing character data and for printing
  fundamental data types (integers, floating point numbers, etc.). Again,
  the \Code{Writer} class itself does not specify where the characters go
  to.

\item[InputStream] represents a binary input stream and provides
  functions for reading arrays of bytes. Where the bytes come from must
  be specified by a derived class.

\item[OutputStream] represents a binary output stream that can be used to
  write arrays of bytes. The \Code{Output\-Stream} class itself does not
  specify where the bytes go to.

\end{Description}
Note that the classes listed above are not thread safe. The \Code{mt}
package provides additional classes that make it possible to share input
and output streams between multiple threads.

The \Code{io} package also contains classes for getting and setting file
properties; for formatting numbers; for dissecting file names; for
reading and writing compressed files; and for many other I/O-related
operations.


\subsection{The \Code{util} package}

The \Code{util} package exports various container classes, an event
framework, and a set of general utility functions. Here is an overview of
the main classes in this package.
\begin{Description}[\Code]

\item[ArrayBuffer] represents a 1-D array that can grow and shrink. It is
  meant to be a helper class for creating \Code{Array} objects of which
  the size is not known beforehand.

\item[SparseArray] implements a multi-dimensional sparse array. It
  behaves more or less like a conventional array, except that it only
  stores those array elements that have actually been set.

\item[HashMap] represents an associative array based on a hash table.
  That is, instead of using integers as indices, it enables you to use
  any type as an index.

\item[Dictionary] \label{class:Dictionary} represents a heterogeneous
  associative container that maps strings to objects.

\item[Properties] \label{class:Properties} implements an hierarchical
  database containing a set of name-value pairs called properties. It is
  typically used as a repository for configuration data. It can also be
  used to pass a dynamically varying set of parameters to a function.

\item[Event] \label{class:Event} encapsulates a list of call-back
  functions that are invoked when a particular method is called. Events
  can be used, among others, to connect independent software components
  and to implement the model-view-controller pattern.

\end{Description}

\subsection{The \Code{xutil} package}

The \Code{xutil} package exports special-purpose container classes,
including:
\begin{Description}[\Code]

\item[SnoopDictionary] implements a
  \hyperref[class:Dictionary]{\Code{Dictionary}} that records which
  objects have been accessed.

\item[IListArray] implements an array of integer arrays with dynamic
  sizes. A special storage scheme is used to lower the memory management
  overhead.

\end{Description}

\subsection{The \Code{gl} package}

The \Code{gl} package implements a extensible framework for building and
managing 2-D and 3-D scene graphs that are rendered with OpenGL. It also
implements classes for managing windows, for creating screen shots, for
drawing text, for mappping data values to colors and the other way
around, and for a number of other useful things. Below is a overview of
the available classes.
\begin{Description}[\Code]

\item[Color] represents an RGBA color. It can be used, among others, for
  passing colors around, for setting and getting the current color, and
  for translating a string to a \Code{Color} object and the other way
  around.

\item[Transform] encapsulates a $4 \times 4$ transformation matrix. It
  provides support for creating transformation matrices from rotations
  and translations, for concatenating transformations, for inverting a
  transformation, and for reading a transformation from a
  \hyperref[class:Properties]{\Code{Properties}} object.

\item[VertexSet] contains a set of points. It is meant to be a building
  block for complex geometric objects -- like line sets and triangle sets
  -- that may share the same set of points.

\item[GfxNode] represents a node in a scene graph. It declares a number
  of virtual methods, called action handlers, that are responsible for
  handling events such as redraw requests, resize requests, mouse clicks
  and key presses. A node typically handles only the events it is
  interested in, and passes all other events to its children, if it has
  any. This design makes it very easy to define new nodes and event types
  without having to modify the existing nodes.

\item[EventNode] passes scene graph events to objects that are not part
  of the scene graph. An \Code{Event\-Node} has a number of
  \hyperref[class:Event]{\Code{Event}} objects that are emitted whenever
  an actions handler of that \Code{Event\-Node} is invoked. An
  \Code{Event\-Node} can therefore be viewed as a kind of probe that
  enables you to intercept scene graph events without having to implement
  a new class that is derived from \Code{Gfx\-Node}.

\item[TriangleSet] implements a scene graph node that renders a set of
  triangles. It also implements a method for finding the triangle that is
  nearest to a given point in space. The \Code{Triangle\-Set} class
  supports three color modes: a single color for all triangles; a
  different color per triangle; and a color per vertex. It can generate
  the triangle normals automatically, but it also allows you to specify a
  normal per triangle or a normal per vertex.

\item[XYPlot] implements a scene graph node that displays one or more
  data sets in a 2-D graph. It has member functions for adding and
  deleting data sets; for changing the way in which each data set is
  displayed; for setting the x-range and y-range; and for setting the
  labels along the axes. Although the graph is displayed on a rectangular
  plane, this plane can be part of a 3-D model. Thus, it would be
  possible to attach six \Code{XYPlot} objects to the sides of a 3-D
  cube.

\item[FrameGrabber] saves a snapshot of a scene graph to an output
  stream. It is an abstract class that defines a common interface for
  different types of frame grabbers. The \Code{gl} package contains two
  derived classes: one that saves a snapshot in the PPM format, and one
  that saves a snapshot in the PNG format.

\item[Controller] enables a user to interact with a scene graph. When a
  \Code{Controller} is created, it inserts an \Code{Event\-Node} into a
  given scene graph. By intercepting input events, it then enables the
  user to select an object; to center that object; to rotate and
  translate that object; to zoom in and out; to save a snapshot; and
  more.

\item[Viewer] encapsulates a window showing a scene graph. A
  \Code{Viewer} manages an OpenGL context and translates window-level
  events -- like key presses, mouse clicks, resize events -- into scene
  graph events. These events are then passed to the appropriate action
  handler of the top-most node in the scene graph.

\item[Display] implements functions for creating and deleting
  \Code{Viewer} objects in a portable way; it can be viewed as a uniform
  interface to different window systems. The \Code{Display} class
  provides the option to spawn a separate thread that handles all
  window-level events and executes all OpenGL calls. In this way you can
  run a computation and display the results at the same time without
  having to check for user events at regular intervals.

\end{Description}

\subsection{The \Code{mp} package}

The \Code{mp} package contains classes for building parallel programs
that are based on the message passing programming model. These classes
can be used to start a set of parallel processes; to exchange data
between processes; and to open files that are read from or written to by
multiple processes. The fundamental classes in this package are:
\begin{Description}[\Code]

\item[Context] represents a communication context. That is, a
  \Code{Context} object acts like a virtual network over which processes
  can communicate.

\item[Request] represents an data transfer. A \Code{Request} object can
  be used to start a transfer; check whether a transfer has been
  completed; and wait until a transfer has been completed.

\item[RequestList] encapsulates an array of \Code{Request} objects. It
  provides functions for starting and ending multiple data transfers.

\item[BcastStream] is an \Code{InputStream} that multiplexes an input
  stream opened by one process to all other processes. It can be used to
  open an input file that is read by multiple processes concurrently.

\item[GatherStream] gathers output generated on different processes and
  writes it to an output stream that has been opened on one particular
  process. It is primarily used to write data produced by multiple
  processes to a single output file.

\end{Description}
The \Code{Context} class is an abstract class that declares various
member functions for exchanging data, but it does not provide an
implementation for these functions. These implementations are provided by
a number of derived classes that use different mechanisms for
transferring data. The \Code{MTContext} class, for instance, is based in
threads and transfers data directly through shared memory, while the
\Code{MPIContext} class uses the MPI library to transfer data. In this
way an application can choose at run time which data transport mechanism
is to be used; see also the description of the environment variable
\hyperref[envar:JEM-MP-DRIVER]{\Code{JEM\_MP\_DRIVER}} in
\autoref{section:environment-variables}.

\subsection{The \Code{mt} package}

The \Code{mt} package provides special-purpose classes for writing
parallel programs that are based in the multi-threading programming
model. These classes include:
\begin{Description}[\Code]

\item[Barrier] synchronizes the execution of a group of threads. That is,
  a thread will wait on a \Code{Barrier} object until the last thread in
  the group has called the member function \Code{wait} for that object.

\item[IOMutex] implements a recursive mutex that can be shared between
  multiple input and output streams. It can be used to provide exclusive
  access to an input or output stream.

\item[ThreadSafeReader] implements a text input stream that can be used
  by multiple threads concurrently. It is essentially a wrapper around a
  \Code{IOMutex} and a \hyperref[class:Reader]{\Code{Reader}} object.

\item[ThreadSafeWriter] implements a text output stream that can be used
  by multiple threads concurrently. It is essentially a wrapper around a
  \Code{IOMutex} and a \hyperref[class:Writer]{\Code{Writer}} object.

\end{Description}

\subsection{The \Code{numeric} package}

The \Code{numeric} package exports classes and functions that can be used
to implement numerical algorithms. In particular, these classes and
functions can be used to execute matrix/vector multiplications; to
determine the inverse of a matrix; to compute the eigenvalues of a
matrix; and to perform various sparse matrix operations. Some of of these
classes and functions are described below.
\begin{Description}[\Code]

\item[matmul] computes the product of two matrices, a matrix and a
  vector, a vector and a matrix, or two vectors. Note that \Code{matmul}
  is a non-member function, and not a class, of which multiple overloaded
  versions exist.

\item[MatmulChain] computes the product of up to four matrices and/or
  vectors. The \Code{Matmul\-Chain} class has been designed to minimize
  the number of allocations that are required to store intermediate
  matrices or vectors. It is therefore ideal for multiplying series of
  matrices within a loop.

\item[LUSolver] provides (static) functions for decomposing a matrix into
  a strictly lower triangular matrix and an upper triangular matrix. It
  also provides functions for performing a pair of forward and backward
  substitutions; for inverting a matrix; and for computing the
  determinant of a matrix.

\item[SparseMatrix] stores a sparse matrix using the compressed row
  storage scheme. It also provides a set of overloaded non-member
  functions and operators for executing common operations involving
  sparse matrices and vectors.

\item[SparseLU] computes the $LU$-decomposition of a sparse matrix. It
  does partial pivoting for stability, and it can use a row/column
  permutation to minimize the number of fill-ins. The \Code{Sparse\-LU}
  class is also able to compute the incomplete $LU$-decomposition of a
  sparse matrix by discarding fill-ins that are smaller than a specified
  threshold.

\end{Description}
The \Code{numeric} uses the \Code{Array} class from the package
\Code{base} to represent dense matrices and vectors; it does not provide
its own matrix and vector classes.

\subsection{The \Code{xml} package}

The \Code{xml} package exports a framework for parsing (simple)
XML-formatted files. This framework is based on the concept of a stack
containing parser objects. The idea is that you associate a parser object
with one or more XML elements. Whenever such an element is encountered in
a file, the corresponding parser object is pushed onto the stack. When
the parser has finished parsing the contents of the element, it is popped
from the stack and control is returned to the previously active
parser.

The main classes making up the parser framework are listed below.
\begin{Description}[\Code]

\item[Tokenizer] converts a XML input file into a set of tokens. A token
  is usually an XML tag, but one can also configure a \Code{Tokenizer} in
  such a way that it marks strings, symbols and numbers as separate
  tokens. This is particularly handy if you need to parse the contents of
  an XML element and store the data directly in some type of object,
  instead of storing the data as a string first.

\item[Parser] provides a common base class for all parser objects. It
  declares a virtual method called \Code{parse} that is invoked each time
  the parser should process an XML element. It also provides some utility
  methods for initializing a parser stack (see below) and for reporting
  parse errors.

\item[ParserStack] implements a stack of \Code{Parser} objects. Each time
  a new parser needs to become active, it is pushed onto the stack.

\item[ParserState] groups various objects, including a \Code{Tokenizer}
  and a \Code{Par\-ser\-Stack}, that are required when parsing an XML
  file. There is typically one \Code{Par\-ser\-State} object that is
  passed as an argument to the \Code{parse} method of the \Code{Parser}
  class.

\end{Description}
In addition to these classes, the \Code{xml} package also exports a
collection of utility classes that can be used, among others, to parse
XML attributes; to map XML tags to \Code{Parser} objects; to log parse
events; and to write data to an XML-formatted output file. It also
contains classes for processing XML files according to the Document
Object Model.

%========================================================================

\section{Environment variables
  \label{section:environment-variables}}

The runtime behavior of a Jem program can be tuned with a number of
environment variables. These are described below.
\begin{Description}[\Code]

\item[JEM_ABORT_ON_EXCEPTION] indicates that a program should be aborted
  whenever a runtime exception is thrown. If this variable is set, then
  all constructors of the class \Code{Runtime\-Exception} will terminate
  the current program by calling \Code{abort}. On Linux and MacOS X this
  can produce a core dump file that can be analyzed with a debugger. Note
  that you may have to explicitly enable core dumps in your shell by
  executing
  \begin{Source}[gobble=4]{sh}
    ulimit -c unlimited
  \end{Source}
  when using a \Code{bash}-like shell, or
  \begin{Source}[gobble=4]{csh}
    limit coredumpsize unlimited
  \end{Source}
  when using a \Code{csh}-like shell.

\item[JEM_CACHE_SIZE] sets the size (in bytes) of the thread-local memory
  pool. To limit the synchronization overhead, each thread created by Jem
  has a small memory pool from which it allocates small objects. The
  default size of this pool is 128KB. You may have to decrease the size
  if your program creates a lot of threads on a low-memory machine.

\item[JEM_MP_DRIVER] \label{envar:JEM-MP-DRIVER} specifies the message
  passing driver to be used by the \Code{mp} package. If this variable is
  empty or not set, the default driver is used. Otherwise, it should have
  one of the following values:
  \begin{Description}[\Code]

  \item[default] use the default driver. This is \Code{mpi} when the
    Message Passing Interface is available, and \Code{mt} otherwise.

  \item[mpi] use the MPI-based driver. MPI will be used to create
    parallel processes and to exchange messages between these
    processes.

  \item[mt]  use the multi-threaded driver. Threads will be used to
    simulate parallel processes; messages are copied directly between
    threads.

  \item[uni] use the single-threaded driver. This is essentially a dummy
    driver that does not create any parallel processes.

  \end{Description}
  Note that the driver specified on the command line (with the \Code{-md}
  or \Code{--mp-driver} option) overrides the driver specified by
  \Code{JEM\_MP\_DRIVER}.

\item[JEM_SPIN_COUNT] sets the number of cycles that a thread waits
  before giving up the processor it is running on. When a thread has to
  wait on a locked synchronization object like a mutex, it can retry
  locking that object a number of times before making a system call and
  give up the processor. The environment variable \Code{JEM\_SPIN\_COUNT}
  specifies how many times a thread should retry to lock an object. The
  default value is zero. By setting it to a non-zero value you can
  decrease the synchronization overhead in multi-threaded programs,
  provided that the number of running threads does not exceed the number
  of physical processors in the system.

\end{Description}

%========================================================================

\section{Make variables
  \label{section:make-variables}}

This section describes the variables that are defined in the
package-specific makefiles. It starts with a summary that you can use to
quickly identify the variables of interest. After that, it provides an
alphabetically sorted list with more information about each variable.

\subsection{Make variable summary}

\begin{MakevarTable}{Customization variables}

  \hyperref[makevar:MY-LIBS]{MY_LIBS} &
  Extra libraries to be linked. \\

  \hyperref[makevar:MY-LIBDIRS]{MY_LIBDIRS} &
  Extra directories to be searched for libraries. \\

  \hyperref[makevar:MY-LIBDIRS-OPT]{MY_LIBDIRS_OPT} &
  Extra directories to be searched for optimized libraries. \\

  \hyperref[makevar:MY-LIBDIRS-DBG]{MY_LIBDIRS_DBG} &
  Extra directories to be searched for libraries with debug
  information. \\

  \hyperref[makevar:MY-INCDIRS]{MY_INCDIRS} &
  Extra directories to be searched for header files. \\

  \hyperref[makevar:MY-CXX-STD-FLAGS]{MY_CXX_STD_FLAGS} &
  Extra compiler options. \\

  \hyperref[makevar:MY-CXX-LNK-FLAGS]{MY_CXX_LNK_FLAGS} &
  Extra linking options. \\

  \hyperref[makevar:MY-CXX-DLL-FLAGS]{MY_CXX_DLL_FLAGS} &
  Extra shared library options. \\

  \hyperref[makevar:MY-CXX-DBG-FLAGS]{MY_CXX_DBG_FLAGS} &
  Extra debugging options. \\

  \hyperref[makevar:MY-CXX-OPT-FLAGS]{MY_CXX_OPT_FLAGS} &
  Extra optimization options. \\

  \hyperref[makevar:MY-CXX-OPT-FLAGS1]{MY_CXX_OPT_FLAGS1} &
  Extra level-1 optimization options. \\

  \hyperref[makevar:MY-CXX-OPT-FLAGS2]{MY_CXX_OPT_FLAGS2} &
  Extra level-2 optimization options. \\

  \hyperref[makevar:MY-CXX-OPT-FLAGS3]{MY_CXX_OPT_FLAGS3} &
  Extra level-3 optimization options. \\

\end{MakevarTable}

\begin{MakevarTable}{Compiler-related variables}

  \hyperref[makevar:LIBS]{LIBS} &
  All libraries to be linked. \\

  \hyperref[makevar:LIBDIRS]{LIBDIRS} &
  All directories to be searched for libraries. \\

  \hyperref[makevar:LIBDIRS-OPT]{LIBDIRS_OPT} &
  All directories to be searched for optimized libraries. \\

  \hyperref[makevar:LIBDIRS-DBG]{LIBDIRS_DBG} &
  All directories to be searched for libraries with debug
  information. \\

  \hyperref[makevar:INCDIRS]{INCDIRS} &
  All directories to be searched for header files. \\

  \hyperref[makevar:CXX]{CXX} &
  The name of the C++ compiler executable. \\

  \hyperref[makevar:CXX-NAME]{CXX_NAME} &
  The (symbolic) name of the C++ compiler. \\

  \hyperref[makevar:CXX-VERSION]{CXX_VERSION} &
  The C++ compiler version. \\

  \hyperref[makevar:CXX-FLAGS]{CXX_FLAGS} &
  The compiler options for producing normal object files. \\

  \hyperref[makevar:CXX-FLAGS-DBG]{CXX_FLAGS_DBG} &
  The compiler options for producing object files with debug
  information. \\

  \hyperref[makevar:CXX-FLAGS-OPT]{CXX_FLAGS_OPT} &
  The compiler options for producing optimized object files. \\

  \hyperref[makevar:CXX-STD-FLAGS]{CXX_STD_FLAGS} &
  The standard compiler options. \\

  \hyperref[makevar:CXX-LNK-FLAGS]{CXX_LNK_FLAGS} &
  The compiler options for linking. \\

  \hyperref[makevar:CXX-DLL-FLAGS]{CXX_DLL_FLAGS} &
  The compiler options for creating code that can be used in
  shared libraries. \\

  \hyperref[makevar:CXX-DBG-FLAGS]{CXX_DBG_FLAGS} &
  The debugging options. \\

  \hyperref[makevar:CXX-OPT-FLAGS1]{CXX_OPT_FLAGS1} &
  The level-1 optimization options. \\

  \hyperref[makevar:CXX-OPT-FLAGS2]{CXX_OPT_FLAGS2} &
  The level-2 optimization options. \\

  \hyperref[makevar:CXX-OPT-FLAGS3]{CXX_OPT_FLAGS3} &
  The level-3 optimization options. \\

  \hyperref[makevar:OPT-LEVEL]{OPT_LEVEL} &
  The compiler optimization level. \\

  \hyperref[makevar:CMODE]{CMODE} &
  The compilation mode. \\

  \hyperref[makevar:Compile]{Compile} &
  Commands for compiling a source file. \\

  \hyperref[makevar:CompileExe]{CompileExe} &
  Commands for compiling an executable program. \\

  \hyperref[makevar:LinkDLL]{LinkDLL} &
  Commands for linking a shared library. \\

  \hyperref[makevar:LinkExe]{LinkExe} &
  Commands for linking an executable program. \\

\end{MakevarTable}

\begin{MakevarTable}{System-related variables}

  \hyperref[makevar:SYS-NAME]{SYS_NAME} &
  The name of the current system. \\

  \hyperref[makevar:SYS-ARCH]{SYS_ARCH} &
  The system's hardware architecture. \\

  \hyperref[makevar:SYS-LIBS]{SYS_LIBS} &
  The system libraries to be linked. \\

  \hyperref[makevar:SYS-LIBDIRS]{SYS_LIBDIRS} &
  The directories to be searched for system libraries. \\

  \hyperref[makevar:SYS-INCDIRS]{SYS_INCDIRS} &
  The directories to be searched for system header files. \\

  \hyperref[makevar:OBJ-SUFFIX]{OBJ_SUFFIX} &
  The suffix of an object file name. \\

  \hyperref[makevar:EXE-SUFFIX]{EXE_SUFFIX} &
  The suffix of an executable name. \\

\end{MakevarTable}

\begin{MakevarTable}{Jem-related variables}

  \hyperref[makevar:JEM-LIBS]{JEM_LIBS} &
  The Jem libraries to be linked. \\

  \hyperref[makevar:JEM-LIBDIRS]{JEM_LIBDIRS} &
  The directories to be searched for the Jem libraries. \\

  \hyperref[makevar:JEM-LIBDIRS-OPT]{JEM_LIBDIRS_OPT} &
  The directories to be searched for the optimized Jem libraries. \\

  \hyperref[makevar:JEM-LIBDIRS-DBG]{JEM_LIBDIRS_DBG} &
  The directories to be searched for the Jem libraries with debug
  information. \\

  \hyperref[makevar:JEM-PACKAGES]{JEM_PACKAGES} &
  The names of all used Jem packages. \\

  \hyperref[makevar:JEM-CXX-STD-FLAGS]{JEM_CXX_STD_FLAGS} &
  The standard compiler options. \\

  \hyperref[makevar:JEM-CXX-LNK-FLAGS]{JEM_CXX_LNK_FLAGS} &
  The compiler options for linking. \\

  \hyperref[makevar:JEM-CXX-DLL-FLAGS]{JEM_CXX_DLL_FLAGS} &
  The shared library options. \\

  \hyperref[makevar:JEM-CXX-DBG-FLAGS]{JEM_CXX_DBG_FLAGS} &
  The debugging options. \\

  \hyperref[makevar:JEM-CXX-OPT-FLAGS1]{JEM_CXX_OPT_FLAGS1} &
  The level-1 optimization options. \\

  \hyperref[makevar:JEM-CXX-OPT-FLAGS2]{JEM_CXX_OPT_FLAGS2} &
  The level-2 optimization options. \\

  \hyperref[makevar:JEM-CXX-OPT-FLAGS3]{JEM_CXX_OPT_FLAGS3} &
  The level-3 optimization options. \\

\end{MakevarTable}

\subsection{Make variable details}

\begin{Description}[\Code]

\item[CMODE] \label{makevar:CMODE} the compilation mode: normal,
  optimized or with debugging information. This variable should be
  either empty, or equal to \Makexp{CMODE_OPT} or \Makexp{CMODE_DBG}.
  In the first case, only the standard compiler options will be used
  when compiling a source file. In the second case, the standard
  options will be extended with the optimization options corresponding
  with the current optimization level. In the third case, the standard
  options will be extended with the debug options. Note that
  \begin{Source}[gobble=4]{make}
    CMODE_OPT = _OPT
    CMODE_DBG = _DBG
  \end{Source}
  See also \hyperref[makevar:Compile]{\Code{Compile}} and
  \hyperref[makevar:OPT-LEVEL]{\Code{OPT\_LEVEL}}.

\item[CXX_DBG_FLAGS] \label{makevar:CXX-DBG-FLAGS} the compiler
  options for adding debugging information to object files. This
  variable is defined as:
  \begin{Source}[gobble=4]{make}
    CXX_DBG_FLAGS = $(MY_CXX_DBG_FLAGS) \
                    $(JEM_CXX_DBG_FLAGS)
  \end{Source}

\item[CXX_DLL_FLAGS] \label{makevar:CXX-DLL-FLAGS} the compiler
  options for creating position-independent object files that can be
  used in shared libraries. This variable is defined as:
  \begin{Source}[gobble=4]{make}
    CXX_DLL_FLAGS = $(MY_CXX_DLL_FLAGS) \
                    $(JEM_CXX_DLL_FLAGS)
  \end{Source}

\item[CXX_FLAGS] \label{makevar:CXX-FLAGS} the compiler options for
  producing normal object files. This variable is defined as:
  \begin{Source}[gobble=4]{make}
    CXX_FLAGS = $(CXX_STD_FLAGS)
  \end{Source} %$
  It exists so that the variable \Code{CMODE} can be used to select
  the compiler options. That is, the options to be passed to the
  compiler are given by the expression:
  \begin{Source}[gobble=4]{make}
    $(CXX_FLAGS$(CMODE))
  \end{Source}
  See also \hyperref[makevar:Compile]{\Code{Compile}}.

\item[CXX_FLAGS_DBG] \label{makevar:CXX-FLAGS-DBG} the compiler
  options for producing object files with debugging information. This
  variable is defined as:
  \begin{Source}[gobble=4]{make}
    CXX_FLAGS_DBG = $(CXX_FLAGS) $(CXX_DBG_FLAGS)
  \end{Source}
  See also \hyperref[makevar:CXX-FLAGS]{\Code{CXX\_FLAGS}} and
  \hyperref[makevar:Compile]{\Code{Compile}}.

\item[CXX_FLAGS_OPT] \label{makevar:CXX-FLAGS-OPT} the compiler
  options for producing optimized object files. This variable is
  defined as:
  \begin{Source}[gobble=4]{make}
    CXX_FLAGS_OPT = $(CXX_FLAGS) \
                    $(CXX_OPT_FLAGS$(OPT_LEVEL))
  \end{Source} %$
  See also \hyperref[makevar:CXX-FLAGS]{\Code{CXX\_FLAGS}},
  \hyperref[makevar:OPT-LEVEL]{\Code{OPT\_LEVEL}} and
  \hyperref[makevar:Compile]{\Code{Compile}}.

\item[CXX_LNK_FLAGS] \label{makevar:CXX-LNK-FLAGS} the compiler
  options for linking a set of object files into an executable
  program. This variable is defined as:
  \begin{Source}[gobble=4]{make}
    CXX_LNK_FLAGS = $(MY_CXX_LNK_FLAGS) \
                    $(JEM_CXX_LNK_FLAGS)
  \end{Source}

\item[CXX_NAME] \label{makevar:CXX-NAME} the symbolic name of the C++
  compiler. It has one of the following values:
  \begin{Description}[\Code]

  \item[gcc] the GNU C/C++ compiler.

  \item[icc] the Intel C++ compiler.

  \item[msc] the Microsoft Visual C++ compiler.

  \end{Description}
  You can use this variable to define compiler-specific options, as in
  the following example:
  \begin{Source}[gobble=4]{make}
    ifeq ($(CXX_NAME),icc)
      MY_CXX_STD_FLAGS = -ip
    endif
  \end{Source} %$

\item[CXX_OPT_FLAGS1] \label{makevar:CXX-OPT-FLAGS1} the level-one
  optimization options for the compiler. This variable is defined as:
  \begin{Source}[gobble=4]{make}
    CXX_OPT_FLAGS1 = $(MY_CXX_OPT_FLAGS1) \
                     $(JEM_CXX_OPT_FLAGS1)
  \end{Source}
  See also \hyperref[makevar:OPT-LEVEL]{\Code{OPT\_LEVEL}}.

\item[CXX_OPT_FLAGS2] \label{makevar:CXX-OPT-FLAGS2} the level-two
  optimization options for the compiler. This variable is defined as:
  \begin{Source}[gobble=4]{make}
    CXX_OPT_FLAGS2 = $(MY_CXX_OPT_FLAGS2) \
                     $(JEM_CXX_OPT_FLAGS2)
  \end{Source}
  See also \hyperref[makevar:OPT-LEVEL]{\Code{OPT\_LEVEL}}.

\item[CXX_OPT_FLAGS3] \label{makevar:CXX-OPT-FLAGS3} the level-three
  optimization options for the compiler. This variable is defined as:
  \begin{Source}[gobble=4]{make}
    CXX_OPT_FLAGS3 = $(MY_CXX_OPT_FLAGS3) \
                     $(JEM_CXX_OPT_FLAGS3)
  \end{Source}
  See also \hyperref[makevar:OPT-LEVEL]{\Code{OPT\_LEVEL}}.

\item[CXX_STD_FLAGS] \label{makevar:CXX-STD-FLAGS} the standard
  compiler flags that are always passed to the compiler. This variable
  is defined as:
  \begin{Source}[gobble=4]{make}
    CXX_STD_FLAGS = $(MY_CXX_STD_FLAGS) \
                    $(JEM_CXX_STD_FLAGS)
  \end{Source}

\item[CXX_VERSION] \label{makevar:CXX-VERSION} the compiler version as
  an integer.

\item[CXX] \label{makevar:CXX} the name of the compiler executable.
  This is usually not the same as \Makexp{CXX\_NAME}.

\item[CompileExe] \label{makevar:CompileExe} a sequence of commands
  for creating an executable program (the current target, \Code{\$@})
  from a single C++ source file and a set of object files (the current
  prerequisites, \Code{\$\^}). The compiler/linker is invoked with the
  following flags:
  \begin{Source}[gobble=4]{make}
    $(CXX_FLAGS$(CMODE)) $(CXX_LNK_FLAGS) $(INCDIRS) \
    $(LIBDIRS$(CMODE)) $(LIBS)
  \end{Source} %$
  See also \hyperref[makevar:Compile]{\Code{Compile}} and
  \hyperref[makevar:LinkExe]{\Code{LinkExe}}.

\item[Compile] \label{makevar:Compile} a sequence of commands for
  creating an executable program (the current target, \Code{\$@}) from
  a single C++ source file (the current prerequisite, \Code{\$<}). The
  compiler is invoked with the folllowing flags:
  \begin{Source}[gobble=4]{make}
    $(CXX_FLAGS$(CMODE)) $(INCDIRS)
  \end{Source} %$
  The compiler wrapper script \Code{\$JEMDIR/\-bin/\-CC} is used to
  invoke the compiler; the make variable \Code{CXX} is ignored.

\item[EXE_SUFFIX] \label{makevar:EXE-SUFFIX} the suffix of an
  executable program. On Unix-like systems, this variable is empty; on
  Windows systems this variable expands to \Code{.exe}.

\item[INCDIRS] \label{makevar:INCDIRS} the list of directories to be
  searched for header files. This variable is defined as:
  \begin{Source}[gobble=4]{make}
    INCDIRS = $(MY_INCDIRS) $(SYS_INCDIRS) $(JEM_INCDIRS)
  \end{Source} %$
  The value of \Code{INCDIRS} is \emph{not} passed directly to the
  compiler. It is first expanded by the shell, and then an appropriate
  compiler flag is prepended to each word in the resulting list,
  unless the word starts with the string \Code{-NODIR:}.

\item[JEM_CXX_DBG_FLAGS] \label{makevar:JEM-CXX-DBG-FLAGS} the
  debugging compiler options. This variable exists so that you can
  re-define \hyperref[makevar:CXX-DBG-FLAGS]{\Code{CXX\_DBG\_FLAGS}}.

\item[JEM_CXX_DLL_FLAGS] \label{makevar:JEM-CXX-DLL-FLAGS} the
  compiler options for creating position-independent object files.
  This variable exists so that you can re-define
  \hyperref[makevar:CXX-DLL-FLAGS]{\Code{CXX\_DLL\_FLAGS}}.

\item[JEM_CXX_LNK_FLAGS] \label{makevar:JEM-CXX-LNK-FLAGS} the
  compiler options for linking a set of object files into an
  executable program. This variable exists so that you can re-define
  \hyperref[makevar:CXX-LNK-FLAGS]{\Code{CXX\_LNK\_FLAGS}}.

\item[JEM_CXX_OPT_FLAGS1] \label{makevar:JEM-CXX-OPT-FLAGS1} the
  level-one optimization options for the compiler. This variable
  exists so that you can re-define
  \hyperref[makevar:CXX-OPT-FLAGS1]{\Code{CXX\_OPT\_FLAGS1}}.

\item[JEM_CXX_OPT_FLAGS2] \label{makevar:JEM-CXX-OPT-FLAGS2} the
  level-two optimization options for the compiler. This variable
  exists so that you can re-define
  \hyperref[makevar:CXX-OPT-FLAGS2]{\Code{CXX\_OPT\_FLAGS2}}.

\item[JEM_CXX_OPT_FLAGS3] \label{makevar:JEM-CXX-OPT-FLAGS3} the
  level-three optimization options for the compiler. This variable
  exists so that you can re-define
  \hyperref[makevar:CXX-OPT-FLAGS3]{\Code{CXX\_OPT\_FLAGS3}}.

\item[JEM_CXX_STD_FLAGS] \label{makevar:JEM-CXX-STD-FLAGS} the
  standard compiler flags that are always passed to the compiler. This
  variable exists so that you can re-define
  \hyperref[makevar:CXX-STD-FLAGS]{\Code{CXX\_STD\_FLAGS}}.

\item[JEM_LIBDIRS_DBG] \label{makevar:JEM-LIBDIRS-DBG} the directories
  containing the Jem libraries with debugging information.

\item[JEM_LIBDIRS_OPT] \label{makevar:JEM-LIBDIRS-OPT} the directories
  containing the optimized Jem libraries.

\item[JEM_LIBDIRS] \label{makevar:JEM-LIBDIRS} the directories
  containing the standard Jem libraries.

\item[JEM_LIBS] \label{makevar:JEM-LIBS} a list of all required Jem
  libraries.

\item[JEM_PACKAGES] \label{makevar:JEM-PACKAGES} the names of all used
  Jem packages.

\item[LIBDIRS_DBG] \label{makevar:LIBDIRS-DBG} the list of directories
  to be searched for libraries with debugging information. This
  variable is defined as:
  \begin{Source}[gobble=4]{make}
    LIBDIRS_DBG = $(MY_LIBDIRS_DBG) $(SYS_LIBDIRS) \
                  $(JEM_LIBDIRS_DBG)
  \end{Source} %$
  See also \hyperref[makevar:LIBDIRS]{\Code{LIBDIRS}}.

\item[LIBDIRS_OPT] \label{makevar:LIBDIRS-OPT} the list of directories
  to be searched for optimized libraries. This variable is defined as:
  \begin{Source}[gobble=4]{make}
    LIBDIRS_OPT = $(MY_LIBDIRS_OPT) $(SYS_LIBDIRS) \
                  $(JEM_LIBDIRS_OPT)
  \end{Source} %$
  See also \hyperref[makevar:LIBDIRS]{\Code{LIBDIRS}}.

\item[LIBDIRS] \label{makevar:LIBDIRS} the list of directories to be
  searched for libraries. This variable is defined as:
  \begin{Source}[gobble=4]{make}
    LIBDIRS = $(MY_LIBDIRS) $(SYS_LIBDIRS) $(JEM_LIBDIRS)
  \end{Source} %$
  The value of \Code{LIBDIRS} is \emph{not} passed directly to the
  compiler/linker. It is first expanded by the shell, and then an
  appropriate compiler flag is prepended to each word in the resulting
  list, unless the word starts with the string \Code{-NODIR:}. See
  also \hyperref[makevar:LinkExe]{\Code{LinkExe}}.

\item[LIBS] \label{makevar:LIBS} the list of libraries to be linked
  with a program. This variable is defined as:
  \begin{Source}[gobble=4]{make}
    LIBS = $(MY_LIBS) $(SYS_LIBS) $(JEM_LIBS)
  \end{Source} %$
  The value of \Code{LIBDIRS} is \emph{not} passed directly to the
  compiler/linker. It is first expanded by the shell, and then an
  appropriate compiler flag is prepended to each word in the resulting
  list, unless the word starts with the string \Code{-NOLIB:}. See
  also \hyperref[makevar:LinkExe]{\Code{LinkExe}}.

\item[LinkDLL] \label{makevar:LinkDLL} a sequence of commands for
  linking a set of object files (the current prerequisites,
  \Code{\$\^}) into a shared library or DLL (the current target,
  \Code{\$@}). The compiler/linker is invoked with the following
  flags:
  \begin{Source}[gobble=4]{make}
    $(CXX_FLAGS$(CMODE)) $(CXX_LNK_FLAGS) \
    $(LIBDIRS$(CMODE)) $(LIBS)
  \end{Source} %$
  The compiler wrapper script \Code{\$JEMDIR/\-bin/\-CC} is used to
  invoke the compiler/linker; the make variable \Code{CXX} is ignored.

\item[LinkExe] \label{makevar:LinkExe} a sequence of commands for
  linking a set of object files (the current prerequisites,
  \Code{\$\^}) into an executable program (the current target,
  \Code{\$@}). The compiler/linker is invoked with the following
  flags:
  \begin{Source}[gobble=4]{make}
    $(CXX_FLAGS$(CMODE)) $(CXX_LNK_FLAGS) \
    $(LIBDIRS$(CMODE)) $(LIBS)
  \end{Source} %$
  The compiler wrapper script \Code{\$JEMDIR/\-bin/\-CC} is used to
  invoke the compiler/linker; the make variable \Code{CXX} is ignored.

\item[MY_CXX_DBG_FLAGS] \label{makevar:MY-CXX-DBG-FLAGS} extra
  debugging options to be passed to the compiler. By default, this
  variable is empty. See also
  \hyperref[makevar:CXX-DBG-FLAGS]{\Code{CXX\_DBG\_FLAGS}}.

\item[MY_CXX_DLL_FLAGS] \label{makevar:MY-CXX-DLL-FLAGS} extra
  compiler options for creating a shared library. By default, this
  variable is empty. See also
  \hyperref[makevar:CXX-DLL-FLAGS]{\Code{CXX\_DLL\_FLAGS}}.

\item[MY_CXX_LNK_FLAGS] \label{makevar:MY-CXX-LNK-FLAGS} extra
  compiler options for linking an executable program. By default, this
  variable is empty. See also
  \hyperref[makevar:CXX-LNK-FLAGS]{\Code{CXX\_LNK\_FLAGS}}.

\item[MY_CXX_OPT_FLAGS1] \label{makevar:MY-CXX-OPT-FLAGS1} extra
  level-one optimization options to be passed to the compiler. By
  default, this variable has the value \Makexp{MY\_CXX\_OPT\_FLAGS}.
  See also \hyperref[makevar:CXX-OPT-FLAGS1]{\Code{CXX\_OPT\_FLAGS1}}.

\item[MY_CXX_OPT_FLAGS2] \label{makevar:MY-CXX-OPT-FLAGS2} extra
  level-two optimization options to be passed to the compiler. By
  default, this variable has the value \Makexp{MY\_CXX\_OPT\_FLAGS}.
  See also \hyperref[makevar:CXX-OPT-FLAGS2]{\Code{CXX\_OPT\_FLAGS2}}.

\item[MY_CXX_OPT_FLAGS3] \label{makevar:MY-CXX-OPT-FLAGS3} extra
  level-three optimization options to be passed to the compiler. By
  default, this variable has the value \Makexp{MY\_CXX\_OPT\_FLAGS}.
  See also \hyperref[makevar:CXX-OPT-FLAGS3]{\Code{CXX\_OPT\_FLAGS3}}.

\item[MY_CXX_OPT_FLAGS] \label{makevar:MY-CXX-OPT-FLAGS} extra
  optimizations options to be passed to the compiler at any
  optimization level. By default, this variable is empty.

\item[MY_CXX_STD_FLAGS] \label{makevar:MY-CXX-STD-FLAGS} extra options
  that should always be passed to the compiler. See also
  \hyperref[makevar:CXX-STD-FLAGS]{\Code{CXX\_STD\_FLAGS}}.

\item[MY_INCDIRS] \label{makevar:MY-INCDIRS} extra directories to be
  searched for header files. See also
  \hyperref[makevar:INCDIRS]{\Code{INCDIRS}}.

\item[MY_LIBDIRS_DBG] \label{makevar:MY-LIBDIRS-DBG} extra directories
  to be searched for libraries with debugging information. By default,
  this variabled has the value \Makexp{MY\_LIBDIRS}. See also
  \hyperref[makevar:LIBDIRS-DBG]{\Code{LIBDIRS\_DBG}}.

\item[MY_LIBDIRS_OPT] \label{makevar:MY-LIBDIRS-OPT} extra directories
  to be searched for optimized libraries. By default, this variable
  has the value \Makexp{MY\_LIBDIRS}. See also
  \hyperref[makevar:LIBDIRS-OPT]{\Code{LIBDIRS\_OPT}}.

\item[MY_LIBDIRS] \label{makevar:MY-LIBDIRS} extra directories to be
  searched for libraries. By default, this variable is empty. See also
  \hyperref[makevar:LIBDIRS]{\Code{LIBDIRS}}.

\item[MY_LIBS] \label{makevar:MY-LIBS} extra directories to be linked
  with a program. By default, this variable is empty. See also
  \hyperref[makevar:LIBS]{\Code{LIBS}}.

\item[OBJ_SUFFIX] \label{makevar:OBJ-SUFFIX} the suffix of an object
  file name. It is equal to \Code{.o} on Unix-like systems, and
  \Code{.obj} on Windows systems.

\item[OPT_LEVEL] \label{makevar:OPT-LEVEL} the optimization level to
  be used. It should be either \Code{1}, \Code{2} or \Code{3}. The
  default value depends on the system and the compiler.

\item[SYS_ARCH] \label{makevar:SYS-ARCH} the (symbolic) name of the
  system's hardware architecture. It has one of the following values:
  \begin{Description}[\Code]

  \item[ia32]   Intel 32-bit architecture (x86 processors).

  \item[ia64]   Intel 64-bit architecture (Itanium processors).

  \item[x86_64] AMD 64-bit architecture (also known as Intel Extended
    Memory Technology).

  \item[ppc]    PowerPC architecture.

  \end{Description}

\item[SYS_INCDIRS] \label{makevar:SYS-INCDIRS} the directories to be
  searched for system header files. See also
  \hyperref[makevar:INCDIRS]{\Code{INCDIRS}}.

\item[SYS_LIBDIRS] \label{makevar:SYS-LIBDIRS} the directories to be
  searched for system libraries. See also
  \hyperref[makevar:LIBDIRS]{\Code{LIBDIRS}}.

\item[SYS_LIBS] \label{makevar:SYS-LIBS} the system directories to be
  linked with a program. See also
  \hyperref[makevar:LIBS]{\Code{LIBS}}.

\item[SYS_NAME] \label{makevar:SYS-NAME} the (symbolic) name of the
  operating system. It has one of the following values:
  \begin{Description}[\Code]

  \item[linux]   GNU/Linux.

  \item[darwin]  Apple MacOS X.

  \item[windows] Microsoft Windows.

  \end{Description}

\end{Description}


\end{document}
