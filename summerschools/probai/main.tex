% Author: Ozgur Taylan TURAN
% Delft University of Technology
\documentclass{article}

\usepackage{/home/taylanot/texmf/tex/arxivtemplate}

\usepackage[utf8]{inputenc} % allow utf-8 input
\usepackage[T1]{fontenc}    % use 8-bit T1 fonts

\title{Motivation Letter}
\begin{document}
%\maketitle


I am Taylan a Ph.D. candidate from Delft University of Technology, Pattern Recognition Lab. I started my higher education with the mindset of equipping myself with as broad knowledge as I can. This led me to obtain two BSc degrees namely Environmental and Civil Engineering at Middle East Technical University (METU) in Turkey. One of my Professors saw my interest in diverse topics and offered me a Research Assistantship and an M.Sc. position in Concrete Structures at METU. After dealing with a variety of fun projects on Applied Mathematics problems, I wanted to follow an M.Sc. in Structural Engineering. I attended the Delft University of Technology to realize that.

This is where my Machine Learning journey starts. I was looking for research projects to work on the side. Another Professor of mine offered an interesting project that involved the intersection of machine learning and computational mechanics. I integrated a Bayesian Optimization procedure for actively probing the domain of a computational model. The aim was to actively create a surrogate model that relies on the expensive Finite Element Method (a method to solve partial differential equations) with the least amount of data possible. This initial project was where I learned about Gaussian Processes. This led me to my M.Sc. thesis project in which I utilized multi-fidelity Gaussian Processes (which is a method that utilizes data coming from different fidelities to improve the data-scarce highest fidelity) to create surrogate constitutive models for plastic materials. 

After, getting two applied projects under my belt, I decided to pursue the Ph.D. that I am enrolled in at the moment. It was offered as a position that creates an interface between machine learning and material science applications with increased educational responsibilities. I am mostly working on the machine learning part where my findings and the methods that I develop are projected to be utilized in material science problems in collaboration with other Ph.D. candidates. This Ph.D. allows me to work on more specific machine learning areas via meta-learning and continual learning. My current research project focuses on understanding current shallow meta-learning/continual learning methods and developing new ones and evaluating them with appropriate measures to have a fair comparison basis with acknowledged models.

My most recent works include an extensive study on a well-known meta-learning algorithm called Model-Agnostic Meta-Learning. In this paper, I investigated the regularization caused by the gradient step during the adaptation stage for the learning task that is of interest and compare its expected performance with single-task learners, which is presented at BNAIC-BeNeLearn 2022 conference in Belgium. In my most recent work, my colleague and I utilized a continual learning method to find constitutive relationships for different types of domains under the same boundary conditions. In this work, we show the data efficiency gained for a new task and emphasize that we can learn multiple tasks and perform equally well in them in a continuous manner. We aim that this type of application can open the way to create more cooperative data-driven computational mechanics applications.
 
Although my past and current Ph.D. work are mainly concerned with deterministic methods. As a final step of my thesis, I would like to focus my energy on probabilistic meta-learning. Although I am familiar with probabilistic models due to the courses that I have attended, I am eager to learn more about the variety of topics that this summer school has to offer. Moreover, due to the nature of my Ph.D., I have increased my educational workload. This is one of the reasons I would like to improve myself on the probabilistic side of machine learning.

I believe that Variational Inference and Causal Inference sessions will help me immensely. One of my side projects includes stitching Gaussian Processes together in a multi-fidelity setting where variational inference might come in handy. Moreover, my general interest lies in the meta-learning setting and my officemate and I are working on causal inference in meta-learning settings. He suggested I attend, sessions with Fredrik D. Johansson since he was a former student of his. Finally, with the final assignment, I believe that I can stimulate myself to think more about the things that I have learned and get the chance to practice probabilistic workflows ahead of the final two years of my Ph.D. program.

All in all, my supervisors and I are on the same page about attending this summer school. We believe that attending this summer school will help me in the final stretch of my Ph.D. journey. With the potential knowledge and experience gained by attending this summer school, I hope to find interesting research questions to work on in a setting that I do not encounter daily. 

\begin{flushright}                                                   
Ozgur Taylan Turan \\
23.02.2023 \\ 
Delft, The Netherlands\\
\end{flushright}                                                   
\thispagestyle{empty}
\end{document}



